\documentclass[aspectratio=169,mathserif]{beamer}
\usepackage{amsmath}
\usepackage{amssymb}
\usepackage{kotex}
\usepackage{pgf}
\setsanshangulfont[ItalicFont={*},ItalicFeatures={FakeSlant=.22}]{Noto Sans CJK KR}
\title{유한체에서의 타원곡선과 암호학에서의 응용}
\author[Sohn]{22학번 손량}

\usetheme{default}
\setbeamertemplate{itemize items}[circle]
\setlength\abovecaptionskip{-20pt}

\newcommand{\Fp}{\mathbb{F}_p}
\newcommand{\Ftwom}{\mathbb{F}_{2^m}}

\begin{document}
  \begin{frame}
    \titlepage
  \end{frame}

  \begin{frame}
    \frametitle{타원곡선}
    \framesubtitle{Definition}

    \begin{columns}
      \column{0.4\textwidth}
      \begin{figure}
        \centering
        \scalebox{0.5}{%% Creator: Matplotlib, PGF backend
%%
%% To include the figure in your LaTeX document, write
%%   \input{<filename>.pgf}
%%
%% Make sure the required packages are loaded in your preamble
%%   \usepackage{pgf}
%%
%% Also ensure that all the required font packages are loaded; for instance,
%% the lmodern package is sometimes necessary when using math font.
%%   \usepackage{lmodern}
%%
%% Figures using additional raster images can only be included by \input if
%% they are in the same directory as the main LaTeX file. For loading figures
%% from other directories you can use the `import` package
%%   \usepackage{import}
%%
%% and then include the figures with
%%   \import{<path to file>}{<filename>.pgf}
%%
%% Matplotlib used the following preamble
%%   \usepackage{fontspec}
%%   \setmainfont{DejaVuSerif.ttf}[Path=\detokenize{/usr/lib/python3.10/site-packages/matplotlib/mpl-data/fonts/ttf/}]
%%   \setsansfont{DejaVuSans.ttf}[Path=\detokenize{/usr/lib/python3.10/site-packages/matplotlib/mpl-data/fonts/ttf/}]
%%   \setmonofont{DejaVuSansMono.ttf}[Path=\detokenize{/usr/lib/python3.10/site-packages/matplotlib/mpl-data/fonts/ttf/}]
%%
\begingroup%
\makeatletter%
\begin{pgfpicture}%
\pgfpathrectangle{\pgfpointorigin}{\pgfqpoint{3.583066in}{4.689426in}}%
\pgfusepath{use as bounding box, clip}%
\begin{pgfscope}%
\pgfsetbuttcap%
\pgfsetmiterjoin%
\definecolor{currentfill}{rgb}{1.000000,1.000000,1.000000}%
\pgfsetfillcolor{currentfill}%
\pgfsetlinewidth{0.000000pt}%
\definecolor{currentstroke}{rgb}{1.000000,1.000000,1.000000}%
\pgfsetstrokecolor{currentstroke}%
\pgfsetdash{}{0pt}%
\pgfpathmoveto{\pgfqpoint{0.000000in}{0.000000in}}%
\pgfpathlineto{\pgfqpoint{3.583066in}{0.000000in}}%
\pgfpathlineto{\pgfqpoint{3.583066in}{4.689426in}}%
\pgfpathlineto{\pgfqpoint{0.000000in}{4.689426in}}%
\pgfpathlineto{\pgfqpoint{0.000000in}{0.000000in}}%
\pgfpathclose%
\pgfusepath{fill}%
\end{pgfscope}%
\begin{pgfscope}%
\pgfsetbuttcap%
\pgfsetmiterjoin%
\definecolor{currentfill}{rgb}{1.000000,1.000000,1.000000}%
\pgfsetfillcolor{currentfill}%
\pgfsetlinewidth{0.000000pt}%
\definecolor{currentstroke}{rgb}{0.000000,0.000000,0.000000}%
\pgfsetstrokecolor{currentstroke}%
\pgfsetstrokeopacity{0.000000}%
\pgfsetdash{}{0pt}%
\pgfpathmoveto{\pgfqpoint{0.100000in}{0.100000in}}%
\pgfpathlineto{\pgfqpoint{3.483066in}{0.100000in}}%
\pgfpathlineto{\pgfqpoint{3.483066in}{4.589426in}}%
\pgfpathlineto{\pgfqpoint{0.100000in}{4.589426in}}%
\pgfpathlineto{\pgfqpoint{0.100000in}{0.100000in}}%
\pgfpathclose%
\pgfusepath{fill}%
\end{pgfscope}%
\begin{pgfscope}%
\pgfsetbuttcap%
\pgfsetroundjoin%
\definecolor{currentfill}{rgb}{0.000000,0.000000,0.000000}%
\pgfsetfillcolor{currentfill}%
\pgfsetlinewidth{0.803000pt}%
\definecolor{currentstroke}{rgb}{0.000000,0.000000,0.000000}%
\pgfsetstrokecolor{currentstroke}%
\pgfsetdash{}{0pt}%
\pgfsys@defobject{currentmarker}{\pgfqpoint{0.000000in}{-0.048611in}}{\pgfqpoint{0.000000in}{0.000000in}}{%
\pgfpathmoveto{\pgfqpoint{0.000000in}{0.000000in}}%
\pgfpathlineto{\pgfqpoint{0.000000in}{-0.048611in}}%
\pgfusepath{stroke,fill}%
}%
\begin{pgfscope}%
\pgfsys@transformshift{0.540170in}{2.344713in}%
\pgfsys@useobject{currentmarker}{}%
\end{pgfscope}%
\end{pgfscope}%
\begin{pgfscope}%
\definecolor{textcolor}{rgb}{0.000000,0.000000,0.000000}%
\pgfsetstrokecolor{textcolor}%
\pgfsetfillcolor{textcolor}%
\pgftext[x=0.540170in,y=2.247491in,,top]{\color{textcolor}\sffamily\fontsize{10.000000}{12.000000}\selectfont \ensuremath{-}1}%
\end{pgfscope}%
\begin{pgfscope}%
\pgfsetbuttcap%
\pgfsetroundjoin%
\definecolor{currentfill}{rgb}{0.000000,0.000000,0.000000}%
\pgfsetfillcolor{currentfill}%
\pgfsetlinewidth{0.803000pt}%
\definecolor{currentstroke}{rgb}{0.000000,0.000000,0.000000}%
\pgfsetstrokecolor{currentstroke}%
\pgfsetdash{}{0pt}%
\pgfsys@defobject{currentmarker}{\pgfqpoint{0.000000in}{-0.048611in}}{\pgfqpoint{0.000000in}{0.000000in}}{%
\pgfpathmoveto{\pgfqpoint{0.000000in}{0.000000in}}%
\pgfpathlineto{\pgfqpoint{0.000000in}{-0.048611in}}%
\pgfusepath{stroke,fill}%
}%
\begin{pgfscope}%
\pgfsys@transformshift{1.259630in}{2.344713in}%
\pgfsys@useobject{currentmarker}{}%
\end{pgfscope}%
\end{pgfscope}%
\begin{pgfscope}%
\pgfsetbuttcap%
\pgfsetroundjoin%
\definecolor{currentfill}{rgb}{0.000000,0.000000,0.000000}%
\pgfsetfillcolor{currentfill}%
\pgfsetlinewidth{0.803000pt}%
\definecolor{currentstroke}{rgb}{0.000000,0.000000,0.000000}%
\pgfsetstrokecolor{currentstroke}%
\pgfsetdash{}{0pt}%
\pgfsys@defobject{currentmarker}{\pgfqpoint{0.000000in}{-0.048611in}}{\pgfqpoint{0.000000in}{0.000000in}}{%
\pgfpathmoveto{\pgfqpoint{0.000000in}{0.000000in}}%
\pgfpathlineto{\pgfqpoint{0.000000in}{-0.048611in}}%
\pgfusepath{stroke,fill}%
}%
\begin{pgfscope}%
\pgfsys@transformshift{1.979089in}{2.344713in}%
\pgfsys@useobject{currentmarker}{}%
\end{pgfscope}%
\end{pgfscope}%
\begin{pgfscope}%
\definecolor{textcolor}{rgb}{0.000000,0.000000,0.000000}%
\pgfsetstrokecolor{textcolor}%
\pgfsetfillcolor{textcolor}%
\pgftext[x=1.979089in,y=2.247491in,,top]{\color{textcolor}\sffamily\fontsize{10.000000}{12.000000}\selectfont 1}%
\end{pgfscope}%
\begin{pgfscope}%
\pgfsetbuttcap%
\pgfsetroundjoin%
\definecolor{currentfill}{rgb}{0.000000,0.000000,0.000000}%
\pgfsetfillcolor{currentfill}%
\pgfsetlinewidth{0.803000pt}%
\definecolor{currentstroke}{rgb}{0.000000,0.000000,0.000000}%
\pgfsetstrokecolor{currentstroke}%
\pgfsetdash{}{0pt}%
\pgfsys@defobject{currentmarker}{\pgfqpoint{0.000000in}{-0.048611in}}{\pgfqpoint{0.000000in}{0.000000in}}{%
\pgfpathmoveto{\pgfqpoint{0.000000in}{0.000000in}}%
\pgfpathlineto{\pgfqpoint{0.000000in}{-0.048611in}}%
\pgfusepath{stroke,fill}%
}%
\begin{pgfscope}%
\pgfsys@transformshift{2.698548in}{2.344713in}%
\pgfsys@useobject{currentmarker}{}%
\end{pgfscope}%
\end{pgfscope}%
\begin{pgfscope}%
\definecolor{textcolor}{rgb}{0.000000,0.000000,0.000000}%
\pgfsetstrokecolor{textcolor}%
\pgfsetfillcolor{textcolor}%
\pgftext[x=2.698548in,y=2.247491in,,top]{\color{textcolor}\sffamily\fontsize{10.000000}{12.000000}\selectfont 2}%
\end{pgfscope}%
\begin{pgfscope}%
\pgfsetbuttcap%
\pgfsetroundjoin%
\definecolor{currentfill}{rgb}{0.000000,0.000000,0.000000}%
\pgfsetfillcolor{currentfill}%
\pgfsetlinewidth{0.803000pt}%
\definecolor{currentstroke}{rgb}{0.000000,0.000000,0.000000}%
\pgfsetstrokecolor{currentstroke}%
\pgfsetdash{}{0pt}%
\pgfsys@defobject{currentmarker}{\pgfqpoint{0.000000in}{-0.048611in}}{\pgfqpoint{0.000000in}{0.000000in}}{%
\pgfpathmoveto{\pgfqpoint{0.000000in}{0.000000in}}%
\pgfpathlineto{\pgfqpoint{0.000000in}{-0.048611in}}%
\pgfusepath{stroke,fill}%
}%
\begin{pgfscope}%
\pgfsys@transformshift{3.418007in}{2.344713in}%
\pgfsys@useobject{currentmarker}{}%
\end{pgfscope}%
\end{pgfscope}%
\begin{pgfscope}%
\definecolor{textcolor}{rgb}{0.000000,0.000000,0.000000}%
\pgfsetstrokecolor{textcolor}%
\pgfsetfillcolor{textcolor}%
\pgftext[x=3.418007in,y=2.247491in,,top]{\color{textcolor}\sffamily\fontsize{10.000000}{12.000000}\selectfont 3}%
\end{pgfscope}%
\begin{pgfscope}%
\pgfsetbuttcap%
\pgfsetroundjoin%
\definecolor{currentfill}{rgb}{0.000000,0.000000,0.000000}%
\pgfsetfillcolor{currentfill}%
\pgfsetlinewidth{0.602250pt}%
\definecolor{currentstroke}{rgb}{0.000000,0.000000,0.000000}%
\pgfsetstrokecolor{currentstroke}%
\pgfsetdash{}{0pt}%
\pgfsys@defobject{currentmarker}{\pgfqpoint{0.000000in}{-0.027778in}}{\pgfqpoint{0.000000in}{0.000000in}}{%
\pgfpathmoveto{\pgfqpoint{0.000000in}{0.000000in}}%
\pgfpathlineto{\pgfqpoint{0.000000in}{-0.027778in}}%
\pgfusepath{stroke,fill}%
}%
\begin{pgfscope}%
\pgfsys@transformshift{0.108495in}{2.344713in}%
\pgfsys@useobject{currentmarker}{}%
\end{pgfscope}%
\end{pgfscope}%
\begin{pgfscope}%
\pgfsetbuttcap%
\pgfsetroundjoin%
\definecolor{currentfill}{rgb}{0.000000,0.000000,0.000000}%
\pgfsetfillcolor{currentfill}%
\pgfsetlinewidth{0.602250pt}%
\definecolor{currentstroke}{rgb}{0.000000,0.000000,0.000000}%
\pgfsetstrokecolor{currentstroke}%
\pgfsetdash{}{0pt}%
\pgfsys@defobject{currentmarker}{\pgfqpoint{0.000000in}{-0.027778in}}{\pgfqpoint{0.000000in}{0.000000in}}{%
\pgfpathmoveto{\pgfqpoint{0.000000in}{0.000000in}}%
\pgfpathlineto{\pgfqpoint{0.000000in}{-0.027778in}}%
\pgfusepath{stroke,fill}%
}%
\begin{pgfscope}%
\pgfsys@transformshift{0.252387in}{2.344713in}%
\pgfsys@useobject{currentmarker}{}%
\end{pgfscope}%
\end{pgfscope}%
\begin{pgfscope}%
\pgfsetbuttcap%
\pgfsetroundjoin%
\definecolor{currentfill}{rgb}{0.000000,0.000000,0.000000}%
\pgfsetfillcolor{currentfill}%
\pgfsetlinewidth{0.602250pt}%
\definecolor{currentstroke}{rgb}{0.000000,0.000000,0.000000}%
\pgfsetstrokecolor{currentstroke}%
\pgfsetdash{}{0pt}%
\pgfsys@defobject{currentmarker}{\pgfqpoint{0.000000in}{-0.027778in}}{\pgfqpoint{0.000000in}{0.000000in}}{%
\pgfpathmoveto{\pgfqpoint{0.000000in}{0.000000in}}%
\pgfpathlineto{\pgfqpoint{0.000000in}{-0.027778in}}%
\pgfusepath{stroke,fill}%
}%
\begin{pgfscope}%
\pgfsys@transformshift{0.396279in}{2.344713in}%
\pgfsys@useobject{currentmarker}{}%
\end{pgfscope}%
\end{pgfscope}%
\begin{pgfscope}%
\pgfsetbuttcap%
\pgfsetroundjoin%
\definecolor{currentfill}{rgb}{0.000000,0.000000,0.000000}%
\pgfsetfillcolor{currentfill}%
\pgfsetlinewidth{0.602250pt}%
\definecolor{currentstroke}{rgb}{0.000000,0.000000,0.000000}%
\pgfsetstrokecolor{currentstroke}%
\pgfsetdash{}{0pt}%
\pgfsys@defobject{currentmarker}{\pgfqpoint{0.000000in}{-0.027778in}}{\pgfqpoint{0.000000in}{0.000000in}}{%
\pgfpathmoveto{\pgfqpoint{0.000000in}{0.000000in}}%
\pgfpathlineto{\pgfqpoint{0.000000in}{-0.027778in}}%
\pgfusepath{stroke,fill}%
}%
\begin{pgfscope}%
\pgfsys@transformshift{0.684062in}{2.344713in}%
\pgfsys@useobject{currentmarker}{}%
\end{pgfscope}%
\end{pgfscope}%
\begin{pgfscope}%
\pgfsetbuttcap%
\pgfsetroundjoin%
\definecolor{currentfill}{rgb}{0.000000,0.000000,0.000000}%
\pgfsetfillcolor{currentfill}%
\pgfsetlinewidth{0.602250pt}%
\definecolor{currentstroke}{rgb}{0.000000,0.000000,0.000000}%
\pgfsetstrokecolor{currentstroke}%
\pgfsetdash{}{0pt}%
\pgfsys@defobject{currentmarker}{\pgfqpoint{0.000000in}{-0.027778in}}{\pgfqpoint{0.000000in}{0.000000in}}{%
\pgfpathmoveto{\pgfqpoint{0.000000in}{0.000000in}}%
\pgfpathlineto{\pgfqpoint{0.000000in}{-0.027778in}}%
\pgfusepath{stroke,fill}%
}%
\begin{pgfscope}%
\pgfsys@transformshift{0.827954in}{2.344713in}%
\pgfsys@useobject{currentmarker}{}%
\end{pgfscope}%
\end{pgfscope}%
\begin{pgfscope}%
\pgfsetbuttcap%
\pgfsetroundjoin%
\definecolor{currentfill}{rgb}{0.000000,0.000000,0.000000}%
\pgfsetfillcolor{currentfill}%
\pgfsetlinewidth{0.602250pt}%
\definecolor{currentstroke}{rgb}{0.000000,0.000000,0.000000}%
\pgfsetstrokecolor{currentstroke}%
\pgfsetdash{}{0pt}%
\pgfsys@defobject{currentmarker}{\pgfqpoint{0.000000in}{-0.027778in}}{\pgfqpoint{0.000000in}{0.000000in}}{%
\pgfpathmoveto{\pgfqpoint{0.000000in}{0.000000in}}%
\pgfpathlineto{\pgfqpoint{0.000000in}{-0.027778in}}%
\pgfusepath{stroke,fill}%
}%
\begin{pgfscope}%
\pgfsys@transformshift{0.971846in}{2.344713in}%
\pgfsys@useobject{currentmarker}{}%
\end{pgfscope}%
\end{pgfscope}%
\begin{pgfscope}%
\pgfsetbuttcap%
\pgfsetroundjoin%
\definecolor{currentfill}{rgb}{0.000000,0.000000,0.000000}%
\pgfsetfillcolor{currentfill}%
\pgfsetlinewidth{0.602250pt}%
\definecolor{currentstroke}{rgb}{0.000000,0.000000,0.000000}%
\pgfsetstrokecolor{currentstroke}%
\pgfsetdash{}{0pt}%
\pgfsys@defobject{currentmarker}{\pgfqpoint{0.000000in}{-0.027778in}}{\pgfqpoint{0.000000in}{0.000000in}}{%
\pgfpathmoveto{\pgfqpoint{0.000000in}{0.000000in}}%
\pgfpathlineto{\pgfqpoint{0.000000in}{-0.027778in}}%
\pgfusepath{stroke,fill}%
}%
\begin{pgfscope}%
\pgfsys@transformshift{1.115738in}{2.344713in}%
\pgfsys@useobject{currentmarker}{}%
\end{pgfscope}%
\end{pgfscope}%
\begin{pgfscope}%
\pgfsetbuttcap%
\pgfsetroundjoin%
\definecolor{currentfill}{rgb}{0.000000,0.000000,0.000000}%
\pgfsetfillcolor{currentfill}%
\pgfsetlinewidth{0.602250pt}%
\definecolor{currentstroke}{rgb}{0.000000,0.000000,0.000000}%
\pgfsetstrokecolor{currentstroke}%
\pgfsetdash{}{0pt}%
\pgfsys@defobject{currentmarker}{\pgfqpoint{0.000000in}{-0.027778in}}{\pgfqpoint{0.000000in}{0.000000in}}{%
\pgfpathmoveto{\pgfqpoint{0.000000in}{0.000000in}}%
\pgfpathlineto{\pgfqpoint{0.000000in}{-0.027778in}}%
\pgfusepath{stroke,fill}%
}%
\begin{pgfscope}%
\pgfsys@transformshift{1.403522in}{2.344713in}%
\pgfsys@useobject{currentmarker}{}%
\end{pgfscope}%
\end{pgfscope}%
\begin{pgfscope}%
\pgfsetbuttcap%
\pgfsetroundjoin%
\definecolor{currentfill}{rgb}{0.000000,0.000000,0.000000}%
\pgfsetfillcolor{currentfill}%
\pgfsetlinewidth{0.602250pt}%
\definecolor{currentstroke}{rgb}{0.000000,0.000000,0.000000}%
\pgfsetstrokecolor{currentstroke}%
\pgfsetdash{}{0pt}%
\pgfsys@defobject{currentmarker}{\pgfqpoint{0.000000in}{-0.027778in}}{\pgfqpoint{0.000000in}{0.000000in}}{%
\pgfpathmoveto{\pgfqpoint{0.000000in}{0.000000in}}%
\pgfpathlineto{\pgfqpoint{0.000000in}{-0.027778in}}%
\pgfusepath{stroke,fill}%
}%
\begin{pgfscope}%
\pgfsys@transformshift{1.547413in}{2.344713in}%
\pgfsys@useobject{currentmarker}{}%
\end{pgfscope}%
\end{pgfscope}%
\begin{pgfscope}%
\pgfsetbuttcap%
\pgfsetroundjoin%
\definecolor{currentfill}{rgb}{0.000000,0.000000,0.000000}%
\pgfsetfillcolor{currentfill}%
\pgfsetlinewidth{0.602250pt}%
\definecolor{currentstroke}{rgb}{0.000000,0.000000,0.000000}%
\pgfsetstrokecolor{currentstroke}%
\pgfsetdash{}{0pt}%
\pgfsys@defobject{currentmarker}{\pgfqpoint{0.000000in}{-0.027778in}}{\pgfqpoint{0.000000in}{0.000000in}}{%
\pgfpathmoveto{\pgfqpoint{0.000000in}{0.000000in}}%
\pgfpathlineto{\pgfqpoint{0.000000in}{-0.027778in}}%
\pgfusepath{stroke,fill}%
}%
\begin{pgfscope}%
\pgfsys@transformshift{1.691305in}{2.344713in}%
\pgfsys@useobject{currentmarker}{}%
\end{pgfscope}%
\end{pgfscope}%
\begin{pgfscope}%
\pgfsetbuttcap%
\pgfsetroundjoin%
\definecolor{currentfill}{rgb}{0.000000,0.000000,0.000000}%
\pgfsetfillcolor{currentfill}%
\pgfsetlinewidth{0.602250pt}%
\definecolor{currentstroke}{rgb}{0.000000,0.000000,0.000000}%
\pgfsetstrokecolor{currentstroke}%
\pgfsetdash{}{0pt}%
\pgfsys@defobject{currentmarker}{\pgfqpoint{0.000000in}{-0.027778in}}{\pgfqpoint{0.000000in}{0.000000in}}{%
\pgfpathmoveto{\pgfqpoint{0.000000in}{0.000000in}}%
\pgfpathlineto{\pgfqpoint{0.000000in}{-0.027778in}}%
\pgfusepath{stroke,fill}%
}%
\begin{pgfscope}%
\pgfsys@transformshift{1.835197in}{2.344713in}%
\pgfsys@useobject{currentmarker}{}%
\end{pgfscope}%
\end{pgfscope}%
\begin{pgfscope}%
\pgfsetbuttcap%
\pgfsetroundjoin%
\definecolor{currentfill}{rgb}{0.000000,0.000000,0.000000}%
\pgfsetfillcolor{currentfill}%
\pgfsetlinewidth{0.602250pt}%
\definecolor{currentstroke}{rgb}{0.000000,0.000000,0.000000}%
\pgfsetstrokecolor{currentstroke}%
\pgfsetdash{}{0pt}%
\pgfsys@defobject{currentmarker}{\pgfqpoint{0.000000in}{-0.027778in}}{\pgfqpoint{0.000000in}{0.000000in}}{%
\pgfpathmoveto{\pgfqpoint{0.000000in}{0.000000in}}%
\pgfpathlineto{\pgfqpoint{0.000000in}{-0.027778in}}%
\pgfusepath{stroke,fill}%
}%
\begin{pgfscope}%
\pgfsys@transformshift{2.122981in}{2.344713in}%
\pgfsys@useobject{currentmarker}{}%
\end{pgfscope}%
\end{pgfscope}%
\begin{pgfscope}%
\pgfsetbuttcap%
\pgfsetroundjoin%
\definecolor{currentfill}{rgb}{0.000000,0.000000,0.000000}%
\pgfsetfillcolor{currentfill}%
\pgfsetlinewidth{0.602250pt}%
\definecolor{currentstroke}{rgb}{0.000000,0.000000,0.000000}%
\pgfsetstrokecolor{currentstroke}%
\pgfsetdash{}{0pt}%
\pgfsys@defobject{currentmarker}{\pgfqpoint{0.000000in}{-0.027778in}}{\pgfqpoint{0.000000in}{0.000000in}}{%
\pgfpathmoveto{\pgfqpoint{0.000000in}{0.000000in}}%
\pgfpathlineto{\pgfqpoint{0.000000in}{-0.027778in}}%
\pgfusepath{stroke,fill}%
}%
\begin{pgfscope}%
\pgfsys@transformshift{2.266873in}{2.344713in}%
\pgfsys@useobject{currentmarker}{}%
\end{pgfscope}%
\end{pgfscope}%
\begin{pgfscope}%
\pgfsetbuttcap%
\pgfsetroundjoin%
\definecolor{currentfill}{rgb}{0.000000,0.000000,0.000000}%
\pgfsetfillcolor{currentfill}%
\pgfsetlinewidth{0.602250pt}%
\definecolor{currentstroke}{rgb}{0.000000,0.000000,0.000000}%
\pgfsetstrokecolor{currentstroke}%
\pgfsetdash{}{0pt}%
\pgfsys@defobject{currentmarker}{\pgfqpoint{0.000000in}{-0.027778in}}{\pgfqpoint{0.000000in}{0.000000in}}{%
\pgfpathmoveto{\pgfqpoint{0.000000in}{0.000000in}}%
\pgfpathlineto{\pgfqpoint{0.000000in}{-0.027778in}}%
\pgfusepath{stroke,fill}%
}%
\begin{pgfscope}%
\pgfsys@transformshift{2.410764in}{2.344713in}%
\pgfsys@useobject{currentmarker}{}%
\end{pgfscope}%
\end{pgfscope}%
\begin{pgfscope}%
\pgfsetbuttcap%
\pgfsetroundjoin%
\definecolor{currentfill}{rgb}{0.000000,0.000000,0.000000}%
\pgfsetfillcolor{currentfill}%
\pgfsetlinewidth{0.602250pt}%
\definecolor{currentstroke}{rgb}{0.000000,0.000000,0.000000}%
\pgfsetstrokecolor{currentstroke}%
\pgfsetdash{}{0pt}%
\pgfsys@defobject{currentmarker}{\pgfqpoint{0.000000in}{-0.027778in}}{\pgfqpoint{0.000000in}{0.000000in}}{%
\pgfpathmoveto{\pgfqpoint{0.000000in}{0.000000in}}%
\pgfpathlineto{\pgfqpoint{0.000000in}{-0.027778in}}%
\pgfusepath{stroke,fill}%
}%
\begin{pgfscope}%
\pgfsys@transformshift{2.554656in}{2.344713in}%
\pgfsys@useobject{currentmarker}{}%
\end{pgfscope}%
\end{pgfscope}%
\begin{pgfscope}%
\pgfsetbuttcap%
\pgfsetroundjoin%
\definecolor{currentfill}{rgb}{0.000000,0.000000,0.000000}%
\pgfsetfillcolor{currentfill}%
\pgfsetlinewidth{0.602250pt}%
\definecolor{currentstroke}{rgb}{0.000000,0.000000,0.000000}%
\pgfsetstrokecolor{currentstroke}%
\pgfsetdash{}{0pt}%
\pgfsys@defobject{currentmarker}{\pgfqpoint{0.000000in}{-0.027778in}}{\pgfqpoint{0.000000in}{0.000000in}}{%
\pgfpathmoveto{\pgfqpoint{0.000000in}{0.000000in}}%
\pgfpathlineto{\pgfqpoint{0.000000in}{-0.027778in}}%
\pgfusepath{stroke,fill}%
}%
\begin{pgfscope}%
\pgfsys@transformshift{2.842440in}{2.344713in}%
\pgfsys@useobject{currentmarker}{}%
\end{pgfscope}%
\end{pgfscope}%
\begin{pgfscope}%
\pgfsetbuttcap%
\pgfsetroundjoin%
\definecolor{currentfill}{rgb}{0.000000,0.000000,0.000000}%
\pgfsetfillcolor{currentfill}%
\pgfsetlinewidth{0.602250pt}%
\definecolor{currentstroke}{rgb}{0.000000,0.000000,0.000000}%
\pgfsetstrokecolor{currentstroke}%
\pgfsetdash{}{0pt}%
\pgfsys@defobject{currentmarker}{\pgfqpoint{0.000000in}{-0.027778in}}{\pgfqpoint{0.000000in}{0.000000in}}{%
\pgfpathmoveto{\pgfqpoint{0.000000in}{0.000000in}}%
\pgfpathlineto{\pgfqpoint{0.000000in}{-0.027778in}}%
\pgfusepath{stroke,fill}%
}%
\begin{pgfscope}%
\pgfsys@transformshift{2.986332in}{2.344713in}%
\pgfsys@useobject{currentmarker}{}%
\end{pgfscope}%
\end{pgfscope}%
\begin{pgfscope}%
\pgfsetbuttcap%
\pgfsetroundjoin%
\definecolor{currentfill}{rgb}{0.000000,0.000000,0.000000}%
\pgfsetfillcolor{currentfill}%
\pgfsetlinewidth{0.602250pt}%
\definecolor{currentstroke}{rgb}{0.000000,0.000000,0.000000}%
\pgfsetstrokecolor{currentstroke}%
\pgfsetdash{}{0pt}%
\pgfsys@defobject{currentmarker}{\pgfqpoint{0.000000in}{-0.027778in}}{\pgfqpoint{0.000000in}{0.000000in}}{%
\pgfpathmoveto{\pgfqpoint{0.000000in}{0.000000in}}%
\pgfpathlineto{\pgfqpoint{0.000000in}{-0.027778in}}%
\pgfusepath{stroke,fill}%
}%
\begin{pgfscope}%
\pgfsys@transformshift{3.130224in}{2.344713in}%
\pgfsys@useobject{currentmarker}{}%
\end{pgfscope}%
\end{pgfscope}%
\begin{pgfscope}%
\pgfsetbuttcap%
\pgfsetroundjoin%
\definecolor{currentfill}{rgb}{0.000000,0.000000,0.000000}%
\pgfsetfillcolor{currentfill}%
\pgfsetlinewidth{0.602250pt}%
\definecolor{currentstroke}{rgb}{0.000000,0.000000,0.000000}%
\pgfsetstrokecolor{currentstroke}%
\pgfsetdash{}{0pt}%
\pgfsys@defobject{currentmarker}{\pgfqpoint{0.000000in}{-0.027778in}}{\pgfqpoint{0.000000in}{0.000000in}}{%
\pgfpathmoveto{\pgfqpoint{0.000000in}{0.000000in}}%
\pgfpathlineto{\pgfqpoint{0.000000in}{-0.027778in}}%
\pgfusepath{stroke,fill}%
}%
\begin{pgfscope}%
\pgfsys@transformshift{3.274116in}{2.344713in}%
\pgfsys@useobject{currentmarker}{}%
\end{pgfscope}%
\end{pgfscope}%
\begin{pgfscope}%
\pgfsetbuttcap%
\pgfsetroundjoin%
\definecolor{currentfill}{rgb}{0.000000,0.000000,0.000000}%
\pgfsetfillcolor{currentfill}%
\pgfsetlinewidth{0.803000pt}%
\definecolor{currentstroke}{rgb}{0.000000,0.000000,0.000000}%
\pgfsetstrokecolor{currentstroke}%
\pgfsetdash{}{0pt}%
\pgfsys@defobject{currentmarker}{\pgfqpoint{-0.048611in}{0.000000in}}{\pgfqpoint{-0.000000in}{0.000000in}}{%
\pgfpathmoveto{\pgfqpoint{-0.000000in}{0.000000in}}%
\pgfpathlineto{\pgfqpoint{-0.048611in}{0.000000in}}%
\pgfusepath{stroke,fill}%
}%
\begin{pgfscope}%
\pgfsys@transformshift{1.259630in}{0.186335in}%
\pgfsys@useobject{currentmarker}{}%
\end{pgfscope}%
\end{pgfscope}%
\begin{pgfscope}%
\definecolor{textcolor}{rgb}{0.000000,0.000000,0.000000}%
\pgfsetstrokecolor{textcolor}%
\pgfsetfillcolor{textcolor}%
\pgftext[x=0.966017in, y=0.133574in, left, base]{\color{textcolor}\sffamily\fontsize{10.000000}{12.000000}\selectfont \ensuremath{-}3}%
\end{pgfscope}%
\begin{pgfscope}%
\pgfsetbuttcap%
\pgfsetroundjoin%
\definecolor{currentfill}{rgb}{0.000000,0.000000,0.000000}%
\pgfsetfillcolor{currentfill}%
\pgfsetlinewidth{0.803000pt}%
\definecolor{currentstroke}{rgb}{0.000000,0.000000,0.000000}%
\pgfsetstrokecolor{currentstroke}%
\pgfsetdash{}{0pt}%
\pgfsys@defobject{currentmarker}{\pgfqpoint{-0.048611in}{0.000000in}}{\pgfqpoint{-0.000000in}{0.000000in}}{%
\pgfpathmoveto{\pgfqpoint{-0.000000in}{0.000000in}}%
\pgfpathlineto{\pgfqpoint{-0.048611in}{0.000000in}}%
\pgfusepath{stroke,fill}%
}%
\begin{pgfscope}%
\pgfsys@transformshift{1.259630in}{0.905794in}%
\pgfsys@useobject{currentmarker}{}%
\end{pgfscope}%
\end{pgfscope}%
\begin{pgfscope}%
\definecolor{textcolor}{rgb}{0.000000,0.000000,0.000000}%
\pgfsetstrokecolor{textcolor}%
\pgfsetfillcolor{textcolor}%
\pgftext[x=0.966017in, y=0.853033in, left, base]{\color{textcolor}\sffamily\fontsize{10.000000}{12.000000}\selectfont \ensuremath{-}2}%
\end{pgfscope}%
\begin{pgfscope}%
\pgfsetbuttcap%
\pgfsetroundjoin%
\definecolor{currentfill}{rgb}{0.000000,0.000000,0.000000}%
\pgfsetfillcolor{currentfill}%
\pgfsetlinewidth{0.803000pt}%
\definecolor{currentstroke}{rgb}{0.000000,0.000000,0.000000}%
\pgfsetstrokecolor{currentstroke}%
\pgfsetdash{}{0pt}%
\pgfsys@defobject{currentmarker}{\pgfqpoint{-0.048611in}{0.000000in}}{\pgfqpoint{-0.000000in}{0.000000in}}{%
\pgfpathmoveto{\pgfqpoint{-0.000000in}{0.000000in}}%
\pgfpathlineto{\pgfqpoint{-0.048611in}{0.000000in}}%
\pgfusepath{stroke,fill}%
}%
\begin{pgfscope}%
\pgfsys@transformshift{1.259630in}{1.625254in}%
\pgfsys@useobject{currentmarker}{}%
\end{pgfscope}%
\end{pgfscope}%
\begin{pgfscope}%
\definecolor{textcolor}{rgb}{0.000000,0.000000,0.000000}%
\pgfsetstrokecolor{textcolor}%
\pgfsetfillcolor{textcolor}%
\pgftext[x=0.966017in, y=1.572492in, left, base]{\color{textcolor}\sffamily\fontsize{10.000000}{12.000000}\selectfont \ensuremath{-}1}%
\end{pgfscope}%
\begin{pgfscope}%
\pgfsetbuttcap%
\pgfsetroundjoin%
\definecolor{currentfill}{rgb}{0.000000,0.000000,0.000000}%
\pgfsetfillcolor{currentfill}%
\pgfsetlinewidth{0.803000pt}%
\definecolor{currentstroke}{rgb}{0.000000,0.000000,0.000000}%
\pgfsetstrokecolor{currentstroke}%
\pgfsetdash{}{0pt}%
\pgfsys@defobject{currentmarker}{\pgfqpoint{-0.048611in}{0.000000in}}{\pgfqpoint{-0.000000in}{0.000000in}}{%
\pgfpathmoveto{\pgfqpoint{-0.000000in}{0.000000in}}%
\pgfpathlineto{\pgfqpoint{-0.048611in}{0.000000in}}%
\pgfusepath{stroke,fill}%
}%
\begin{pgfscope}%
\pgfsys@transformshift{1.259630in}{2.344713in}%
\pgfsys@useobject{currentmarker}{}%
\end{pgfscope}%
\end{pgfscope}%
\begin{pgfscope}%
\pgfsetbuttcap%
\pgfsetroundjoin%
\definecolor{currentfill}{rgb}{0.000000,0.000000,0.000000}%
\pgfsetfillcolor{currentfill}%
\pgfsetlinewidth{0.803000pt}%
\definecolor{currentstroke}{rgb}{0.000000,0.000000,0.000000}%
\pgfsetstrokecolor{currentstroke}%
\pgfsetdash{}{0pt}%
\pgfsys@defobject{currentmarker}{\pgfqpoint{-0.048611in}{0.000000in}}{\pgfqpoint{-0.000000in}{0.000000in}}{%
\pgfpathmoveto{\pgfqpoint{-0.000000in}{0.000000in}}%
\pgfpathlineto{\pgfqpoint{-0.048611in}{0.000000in}}%
\pgfusepath{stroke,fill}%
}%
\begin{pgfscope}%
\pgfsys@transformshift{1.259630in}{3.064172in}%
\pgfsys@useobject{currentmarker}{}%
\end{pgfscope}%
\end{pgfscope}%
\begin{pgfscope}%
\definecolor{textcolor}{rgb}{0.000000,0.000000,0.000000}%
\pgfsetstrokecolor{textcolor}%
\pgfsetfillcolor{textcolor}%
\pgftext[x=1.074042in, y=3.011411in, left, base]{\color{textcolor}\sffamily\fontsize{10.000000}{12.000000}\selectfont 1}%
\end{pgfscope}%
\begin{pgfscope}%
\pgfsetbuttcap%
\pgfsetroundjoin%
\definecolor{currentfill}{rgb}{0.000000,0.000000,0.000000}%
\pgfsetfillcolor{currentfill}%
\pgfsetlinewidth{0.803000pt}%
\definecolor{currentstroke}{rgb}{0.000000,0.000000,0.000000}%
\pgfsetstrokecolor{currentstroke}%
\pgfsetdash{}{0pt}%
\pgfsys@defobject{currentmarker}{\pgfqpoint{-0.048611in}{0.000000in}}{\pgfqpoint{-0.000000in}{0.000000in}}{%
\pgfpathmoveto{\pgfqpoint{-0.000000in}{0.000000in}}%
\pgfpathlineto{\pgfqpoint{-0.048611in}{0.000000in}}%
\pgfusepath{stroke,fill}%
}%
\begin{pgfscope}%
\pgfsys@transformshift{1.259630in}{3.783631in}%
\pgfsys@useobject{currentmarker}{}%
\end{pgfscope}%
\end{pgfscope}%
\begin{pgfscope}%
\definecolor{textcolor}{rgb}{0.000000,0.000000,0.000000}%
\pgfsetstrokecolor{textcolor}%
\pgfsetfillcolor{textcolor}%
\pgftext[x=1.074042in, y=3.730870in, left, base]{\color{textcolor}\sffamily\fontsize{10.000000}{12.000000}\selectfont 2}%
\end{pgfscope}%
\begin{pgfscope}%
\pgfsetbuttcap%
\pgfsetroundjoin%
\definecolor{currentfill}{rgb}{0.000000,0.000000,0.000000}%
\pgfsetfillcolor{currentfill}%
\pgfsetlinewidth{0.803000pt}%
\definecolor{currentstroke}{rgb}{0.000000,0.000000,0.000000}%
\pgfsetstrokecolor{currentstroke}%
\pgfsetdash{}{0pt}%
\pgfsys@defobject{currentmarker}{\pgfqpoint{-0.048611in}{0.000000in}}{\pgfqpoint{-0.000000in}{0.000000in}}{%
\pgfpathmoveto{\pgfqpoint{-0.000000in}{0.000000in}}%
\pgfpathlineto{\pgfqpoint{-0.048611in}{0.000000in}}%
\pgfusepath{stroke,fill}%
}%
\begin{pgfscope}%
\pgfsys@transformshift{1.259630in}{4.503091in}%
\pgfsys@useobject{currentmarker}{}%
\end{pgfscope}%
\end{pgfscope}%
\begin{pgfscope}%
\definecolor{textcolor}{rgb}{0.000000,0.000000,0.000000}%
\pgfsetstrokecolor{textcolor}%
\pgfsetfillcolor{textcolor}%
\pgftext[x=1.074042in, y=4.450329in, left, base]{\color{textcolor}\sffamily\fontsize{10.000000}{12.000000}\selectfont 3}%
\end{pgfscope}%
\begin{pgfscope}%
\pgfsetbuttcap%
\pgfsetroundjoin%
\definecolor{currentfill}{rgb}{0.000000,0.000000,0.000000}%
\pgfsetfillcolor{currentfill}%
\pgfsetlinewidth{0.602250pt}%
\definecolor{currentstroke}{rgb}{0.000000,0.000000,0.000000}%
\pgfsetstrokecolor{currentstroke}%
\pgfsetdash{}{0pt}%
\pgfsys@defobject{currentmarker}{\pgfqpoint{-0.027778in}{0.000000in}}{\pgfqpoint{-0.000000in}{0.000000in}}{%
\pgfpathmoveto{\pgfqpoint{-0.000000in}{0.000000in}}%
\pgfpathlineto{\pgfqpoint{-0.027778in}{0.000000in}}%
\pgfusepath{stroke,fill}%
}%
\begin{pgfscope}%
\pgfsys@transformshift{1.259630in}{0.330227in}%
\pgfsys@useobject{currentmarker}{}%
\end{pgfscope}%
\end{pgfscope}%
\begin{pgfscope}%
\pgfsetbuttcap%
\pgfsetroundjoin%
\definecolor{currentfill}{rgb}{0.000000,0.000000,0.000000}%
\pgfsetfillcolor{currentfill}%
\pgfsetlinewidth{0.602250pt}%
\definecolor{currentstroke}{rgb}{0.000000,0.000000,0.000000}%
\pgfsetstrokecolor{currentstroke}%
\pgfsetdash{}{0pt}%
\pgfsys@defobject{currentmarker}{\pgfqpoint{-0.027778in}{0.000000in}}{\pgfqpoint{-0.000000in}{0.000000in}}{%
\pgfpathmoveto{\pgfqpoint{-0.000000in}{0.000000in}}%
\pgfpathlineto{\pgfqpoint{-0.027778in}{0.000000in}}%
\pgfusepath{stroke,fill}%
}%
\begin{pgfscope}%
\pgfsys@transformshift{1.259630in}{0.474119in}%
\pgfsys@useobject{currentmarker}{}%
\end{pgfscope}%
\end{pgfscope}%
\begin{pgfscope}%
\pgfsetbuttcap%
\pgfsetroundjoin%
\definecolor{currentfill}{rgb}{0.000000,0.000000,0.000000}%
\pgfsetfillcolor{currentfill}%
\pgfsetlinewidth{0.602250pt}%
\definecolor{currentstroke}{rgb}{0.000000,0.000000,0.000000}%
\pgfsetstrokecolor{currentstroke}%
\pgfsetdash{}{0pt}%
\pgfsys@defobject{currentmarker}{\pgfqpoint{-0.027778in}{0.000000in}}{\pgfqpoint{-0.000000in}{0.000000in}}{%
\pgfpathmoveto{\pgfqpoint{-0.000000in}{0.000000in}}%
\pgfpathlineto{\pgfqpoint{-0.027778in}{0.000000in}}%
\pgfusepath{stroke,fill}%
}%
\begin{pgfscope}%
\pgfsys@transformshift{1.259630in}{0.618011in}%
\pgfsys@useobject{currentmarker}{}%
\end{pgfscope}%
\end{pgfscope}%
\begin{pgfscope}%
\pgfsetbuttcap%
\pgfsetroundjoin%
\definecolor{currentfill}{rgb}{0.000000,0.000000,0.000000}%
\pgfsetfillcolor{currentfill}%
\pgfsetlinewidth{0.602250pt}%
\definecolor{currentstroke}{rgb}{0.000000,0.000000,0.000000}%
\pgfsetstrokecolor{currentstroke}%
\pgfsetdash{}{0pt}%
\pgfsys@defobject{currentmarker}{\pgfqpoint{-0.027778in}{0.000000in}}{\pgfqpoint{-0.000000in}{0.000000in}}{%
\pgfpathmoveto{\pgfqpoint{-0.000000in}{0.000000in}}%
\pgfpathlineto{\pgfqpoint{-0.027778in}{0.000000in}}%
\pgfusepath{stroke,fill}%
}%
\begin{pgfscope}%
\pgfsys@transformshift{1.259630in}{0.761903in}%
\pgfsys@useobject{currentmarker}{}%
\end{pgfscope}%
\end{pgfscope}%
\begin{pgfscope}%
\pgfsetbuttcap%
\pgfsetroundjoin%
\definecolor{currentfill}{rgb}{0.000000,0.000000,0.000000}%
\pgfsetfillcolor{currentfill}%
\pgfsetlinewidth{0.602250pt}%
\definecolor{currentstroke}{rgb}{0.000000,0.000000,0.000000}%
\pgfsetstrokecolor{currentstroke}%
\pgfsetdash{}{0pt}%
\pgfsys@defobject{currentmarker}{\pgfqpoint{-0.027778in}{0.000000in}}{\pgfqpoint{-0.000000in}{0.000000in}}{%
\pgfpathmoveto{\pgfqpoint{-0.000000in}{0.000000in}}%
\pgfpathlineto{\pgfqpoint{-0.027778in}{0.000000in}}%
\pgfusepath{stroke,fill}%
}%
\begin{pgfscope}%
\pgfsys@transformshift{1.259630in}{1.049686in}%
\pgfsys@useobject{currentmarker}{}%
\end{pgfscope}%
\end{pgfscope}%
\begin{pgfscope}%
\pgfsetbuttcap%
\pgfsetroundjoin%
\definecolor{currentfill}{rgb}{0.000000,0.000000,0.000000}%
\pgfsetfillcolor{currentfill}%
\pgfsetlinewidth{0.602250pt}%
\definecolor{currentstroke}{rgb}{0.000000,0.000000,0.000000}%
\pgfsetstrokecolor{currentstroke}%
\pgfsetdash{}{0pt}%
\pgfsys@defobject{currentmarker}{\pgfqpoint{-0.027778in}{0.000000in}}{\pgfqpoint{-0.000000in}{0.000000in}}{%
\pgfpathmoveto{\pgfqpoint{-0.000000in}{0.000000in}}%
\pgfpathlineto{\pgfqpoint{-0.027778in}{0.000000in}}%
\pgfusepath{stroke,fill}%
}%
\begin{pgfscope}%
\pgfsys@transformshift{1.259630in}{1.193578in}%
\pgfsys@useobject{currentmarker}{}%
\end{pgfscope}%
\end{pgfscope}%
\begin{pgfscope}%
\pgfsetbuttcap%
\pgfsetroundjoin%
\definecolor{currentfill}{rgb}{0.000000,0.000000,0.000000}%
\pgfsetfillcolor{currentfill}%
\pgfsetlinewidth{0.602250pt}%
\definecolor{currentstroke}{rgb}{0.000000,0.000000,0.000000}%
\pgfsetstrokecolor{currentstroke}%
\pgfsetdash{}{0pt}%
\pgfsys@defobject{currentmarker}{\pgfqpoint{-0.027778in}{0.000000in}}{\pgfqpoint{-0.000000in}{0.000000in}}{%
\pgfpathmoveto{\pgfqpoint{-0.000000in}{0.000000in}}%
\pgfpathlineto{\pgfqpoint{-0.027778in}{0.000000in}}%
\pgfusepath{stroke,fill}%
}%
\begin{pgfscope}%
\pgfsys@transformshift{1.259630in}{1.337470in}%
\pgfsys@useobject{currentmarker}{}%
\end{pgfscope}%
\end{pgfscope}%
\begin{pgfscope}%
\pgfsetbuttcap%
\pgfsetroundjoin%
\definecolor{currentfill}{rgb}{0.000000,0.000000,0.000000}%
\pgfsetfillcolor{currentfill}%
\pgfsetlinewidth{0.602250pt}%
\definecolor{currentstroke}{rgb}{0.000000,0.000000,0.000000}%
\pgfsetstrokecolor{currentstroke}%
\pgfsetdash{}{0pt}%
\pgfsys@defobject{currentmarker}{\pgfqpoint{-0.027778in}{0.000000in}}{\pgfqpoint{-0.000000in}{0.000000in}}{%
\pgfpathmoveto{\pgfqpoint{-0.000000in}{0.000000in}}%
\pgfpathlineto{\pgfqpoint{-0.027778in}{0.000000in}}%
\pgfusepath{stroke,fill}%
}%
\begin{pgfscope}%
\pgfsys@transformshift{1.259630in}{1.481362in}%
\pgfsys@useobject{currentmarker}{}%
\end{pgfscope}%
\end{pgfscope}%
\begin{pgfscope}%
\pgfsetbuttcap%
\pgfsetroundjoin%
\definecolor{currentfill}{rgb}{0.000000,0.000000,0.000000}%
\pgfsetfillcolor{currentfill}%
\pgfsetlinewidth{0.602250pt}%
\definecolor{currentstroke}{rgb}{0.000000,0.000000,0.000000}%
\pgfsetstrokecolor{currentstroke}%
\pgfsetdash{}{0pt}%
\pgfsys@defobject{currentmarker}{\pgfqpoint{-0.027778in}{0.000000in}}{\pgfqpoint{-0.000000in}{0.000000in}}{%
\pgfpathmoveto{\pgfqpoint{-0.000000in}{0.000000in}}%
\pgfpathlineto{\pgfqpoint{-0.027778in}{0.000000in}}%
\pgfusepath{stroke,fill}%
}%
\begin{pgfscope}%
\pgfsys@transformshift{1.259630in}{1.769145in}%
\pgfsys@useobject{currentmarker}{}%
\end{pgfscope}%
\end{pgfscope}%
\begin{pgfscope}%
\pgfsetbuttcap%
\pgfsetroundjoin%
\definecolor{currentfill}{rgb}{0.000000,0.000000,0.000000}%
\pgfsetfillcolor{currentfill}%
\pgfsetlinewidth{0.602250pt}%
\definecolor{currentstroke}{rgb}{0.000000,0.000000,0.000000}%
\pgfsetstrokecolor{currentstroke}%
\pgfsetdash{}{0pt}%
\pgfsys@defobject{currentmarker}{\pgfqpoint{-0.027778in}{0.000000in}}{\pgfqpoint{-0.000000in}{0.000000in}}{%
\pgfpathmoveto{\pgfqpoint{-0.000000in}{0.000000in}}%
\pgfpathlineto{\pgfqpoint{-0.027778in}{0.000000in}}%
\pgfusepath{stroke,fill}%
}%
\begin{pgfscope}%
\pgfsys@transformshift{1.259630in}{1.913037in}%
\pgfsys@useobject{currentmarker}{}%
\end{pgfscope}%
\end{pgfscope}%
\begin{pgfscope}%
\pgfsetbuttcap%
\pgfsetroundjoin%
\definecolor{currentfill}{rgb}{0.000000,0.000000,0.000000}%
\pgfsetfillcolor{currentfill}%
\pgfsetlinewidth{0.602250pt}%
\definecolor{currentstroke}{rgb}{0.000000,0.000000,0.000000}%
\pgfsetstrokecolor{currentstroke}%
\pgfsetdash{}{0pt}%
\pgfsys@defobject{currentmarker}{\pgfqpoint{-0.027778in}{0.000000in}}{\pgfqpoint{-0.000000in}{0.000000in}}{%
\pgfpathmoveto{\pgfqpoint{-0.000000in}{0.000000in}}%
\pgfpathlineto{\pgfqpoint{-0.027778in}{0.000000in}}%
\pgfusepath{stroke,fill}%
}%
\begin{pgfscope}%
\pgfsys@transformshift{1.259630in}{2.056929in}%
\pgfsys@useobject{currentmarker}{}%
\end{pgfscope}%
\end{pgfscope}%
\begin{pgfscope}%
\pgfsetbuttcap%
\pgfsetroundjoin%
\definecolor{currentfill}{rgb}{0.000000,0.000000,0.000000}%
\pgfsetfillcolor{currentfill}%
\pgfsetlinewidth{0.602250pt}%
\definecolor{currentstroke}{rgb}{0.000000,0.000000,0.000000}%
\pgfsetstrokecolor{currentstroke}%
\pgfsetdash{}{0pt}%
\pgfsys@defobject{currentmarker}{\pgfqpoint{-0.027778in}{0.000000in}}{\pgfqpoint{-0.000000in}{0.000000in}}{%
\pgfpathmoveto{\pgfqpoint{-0.000000in}{0.000000in}}%
\pgfpathlineto{\pgfqpoint{-0.027778in}{0.000000in}}%
\pgfusepath{stroke,fill}%
}%
\begin{pgfscope}%
\pgfsys@transformshift{1.259630in}{2.200821in}%
\pgfsys@useobject{currentmarker}{}%
\end{pgfscope}%
\end{pgfscope}%
\begin{pgfscope}%
\pgfsetbuttcap%
\pgfsetroundjoin%
\definecolor{currentfill}{rgb}{0.000000,0.000000,0.000000}%
\pgfsetfillcolor{currentfill}%
\pgfsetlinewidth{0.602250pt}%
\definecolor{currentstroke}{rgb}{0.000000,0.000000,0.000000}%
\pgfsetstrokecolor{currentstroke}%
\pgfsetdash{}{0pt}%
\pgfsys@defobject{currentmarker}{\pgfqpoint{-0.027778in}{0.000000in}}{\pgfqpoint{-0.000000in}{0.000000in}}{%
\pgfpathmoveto{\pgfqpoint{-0.000000in}{0.000000in}}%
\pgfpathlineto{\pgfqpoint{-0.027778in}{0.000000in}}%
\pgfusepath{stroke,fill}%
}%
\begin{pgfscope}%
\pgfsys@transformshift{1.259630in}{2.488605in}%
\pgfsys@useobject{currentmarker}{}%
\end{pgfscope}%
\end{pgfscope}%
\begin{pgfscope}%
\pgfsetbuttcap%
\pgfsetroundjoin%
\definecolor{currentfill}{rgb}{0.000000,0.000000,0.000000}%
\pgfsetfillcolor{currentfill}%
\pgfsetlinewidth{0.602250pt}%
\definecolor{currentstroke}{rgb}{0.000000,0.000000,0.000000}%
\pgfsetstrokecolor{currentstroke}%
\pgfsetdash{}{0pt}%
\pgfsys@defobject{currentmarker}{\pgfqpoint{-0.027778in}{0.000000in}}{\pgfqpoint{-0.000000in}{0.000000in}}{%
\pgfpathmoveto{\pgfqpoint{-0.000000in}{0.000000in}}%
\pgfpathlineto{\pgfqpoint{-0.027778in}{0.000000in}}%
\pgfusepath{stroke,fill}%
}%
\begin{pgfscope}%
\pgfsys@transformshift{1.259630in}{2.632497in}%
\pgfsys@useobject{currentmarker}{}%
\end{pgfscope}%
\end{pgfscope}%
\begin{pgfscope}%
\pgfsetbuttcap%
\pgfsetroundjoin%
\definecolor{currentfill}{rgb}{0.000000,0.000000,0.000000}%
\pgfsetfillcolor{currentfill}%
\pgfsetlinewidth{0.602250pt}%
\definecolor{currentstroke}{rgb}{0.000000,0.000000,0.000000}%
\pgfsetstrokecolor{currentstroke}%
\pgfsetdash{}{0pt}%
\pgfsys@defobject{currentmarker}{\pgfqpoint{-0.027778in}{0.000000in}}{\pgfqpoint{-0.000000in}{0.000000in}}{%
\pgfpathmoveto{\pgfqpoint{-0.000000in}{0.000000in}}%
\pgfpathlineto{\pgfqpoint{-0.027778in}{0.000000in}}%
\pgfusepath{stroke,fill}%
}%
\begin{pgfscope}%
\pgfsys@transformshift{1.259630in}{2.776388in}%
\pgfsys@useobject{currentmarker}{}%
\end{pgfscope}%
\end{pgfscope}%
\begin{pgfscope}%
\pgfsetbuttcap%
\pgfsetroundjoin%
\definecolor{currentfill}{rgb}{0.000000,0.000000,0.000000}%
\pgfsetfillcolor{currentfill}%
\pgfsetlinewidth{0.602250pt}%
\definecolor{currentstroke}{rgb}{0.000000,0.000000,0.000000}%
\pgfsetstrokecolor{currentstroke}%
\pgfsetdash{}{0pt}%
\pgfsys@defobject{currentmarker}{\pgfqpoint{-0.027778in}{0.000000in}}{\pgfqpoint{-0.000000in}{0.000000in}}{%
\pgfpathmoveto{\pgfqpoint{-0.000000in}{0.000000in}}%
\pgfpathlineto{\pgfqpoint{-0.027778in}{0.000000in}}%
\pgfusepath{stroke,fill}%
}%
\begin{pgfscope}%
\pgfsys@transformshift{1.259630in}{2.920280in}%
\pgfsys@useobject{currentmarker}{}%
\end{pgfscope}%
\end{pgfscope}%
\begin{pgfscope}%
\pgfsetbuttcap%
\pgfsetroundjoin%
\definecolor{currentfill}{rgb}{0.000000,0.000000,0.000000}%
\pgfsetfillcolor{currentfill}%
\pgfsetlinewidth{0.602250pt}%
\definecolor{currentstroke}{rgb}{0.000000,0.000000,0.000000}%
\pgfsetstrokecolor{currentstroke}%
\pgfsetdash{}{0pt}%
\pgfsys@defobject{currentmarker}{\pgfqpoint{-0.027778in}{0.000000in}}{\pgfqpoint{-0.000000in}{0.000000in}}{%
\pgfpathmoveto{\pgfqpoint{-0.000000in}{0.000000in}}%
\pgfpathlineto{\pgfqpoint{-0.027778in}{0.000000in}}%
\pgfusepath{stroke,fill}%
}%
\begin{pgfscope}%
\pgfsys@transformshift{1.259630in}{3.208064in}%
\pgfsys@useobject{currentmarker}{}%
\end{pgfscope}%
\end{pgfscope}%
\begin{pgfscope}%
\pgfsetbuttcap%
\pgfsetroundjoin%
\definecolor{currentfill}{rgb}{0.000000,0.000000,0.000000}%
\pgfsetfillcolor{currentfill}%
\pgfsetlinewidth{0.602250pt}%
\definecolor{currentstroke}{rgb}{0.000000,0.000000,0.000000}%
\pgfsetstrokecolor{currentstroke}%
\pgfsetdash{}{0pt}%
\pgfsys@defobject{currentmarker}{\pgfqpoint{-0.027778in}{0.000000in}}{\pgfqpoint{-0.000000in}{0.000000in}}{%
\pgfpathmoveto{\pgfqpoint{-0.000000in}{0.000000in}}%
\pgfpathlineto{\pgfqpoint{-0.027778in}{0.000000in}}%
\pgfusepath{stroke,fill}%
}%
\begin{pgfscope}%
\pgfsys@transformshift{1.259630in}{3.351956in}%
\pgfsys@useobject{currentmarker}{}%
\end{pgfscope}%
\end{pgfscope}%
\begin{pgfscope}%
\pgfsetbuttcap%
\pgfsetroundjoin%
\definecolor{currentfill}{rgb}{0.000000,0.000000,0.000000}%
\pgfsetfillcolor{currentfill}%
\pgfsetlinewidth{0.602250pt}%
\definecolor{currentstroke}{rgb}{0.000000,0.000000,0.000000}%
\pgfsetstrokecolor{currentstroke}%
\pgfsetdash{}{0pt}%
\pgfsys@defobject{currentmarker}{\pgfqpoint{-0.027778in}{0.000000in}}{\pgfqpoint{-0.000000in}{0.000000in}}{%
\pgfpathmoveto{\pgfqpoint{-0.000000in}{0.000000in}}%
\pgfpathlineto{\pgfqpoint{-0.027778in}{0.000000in}}%
\pgfusepath{stroke,fill}%
}%
\begin{pgfscope}%
\pgfsys@transformshift{1.259630in}{3.495848in}%
\pgfsys@useobject{currentmarker}{}%
\end{pgfscope}%
\end{pgfscope}%
\begin{pgfscope}%
\pgfsetbuttcap%
\pgfsetroundjoin%
\definecolor{currentfill}{rgb}{0.000000,0.000000,0.000000}%
\pgfsetfillcolor{currentfill}%
\pgfsetlinewidth{0.602250pt}%
\definecolor{currentstroke}{rgb}{0.000000,0.000000,0.000000}%
\pgfsetstrokecolor{currentstroke}%
\pgfsetdash{}{0pt}%
\pgfsys@defobject{currentmarker}{\pgfqpoint{-0.027778in}{0.000000in}}{\pgfqpoint{-0.000000in}{0.000000in}}{%
\pgfpathmoveto{\pgfqpoint{-0.000000in}{0.000000in}}%
\pgfpathlineto{\pgfqpoint{-0.027778in}{0.000000in}}%
\pgfusepath{stroke,fill}%
}%
\begin{pgfscope}%
\pgfsys@transformshift{1.259630in}{3.639739in}%
\pgfsys@useobject{currentmarker}{}%
\end{pgfscope}%
\end{pgfscope}%
\begin{pgfscope}%
\pgfsetbuttcap%
\pgfsetroundjoin%
\definecolor{currentfill}{rgb}{0.000000,0.000000,0.000000}%
\pgfsetfillcolor{currentfill}%
\pgfsetlinewidth{0.602250pt}%
\definecolor{currentstroke}{rgb}{0.000000,0.000000,0.000000}%
\pgfsetstrokecolor{currentstroke}%
\pgfsetdash{}{0pt}%
\pgfsys@defobject{currentmarker}{\pgfqpoint{-0.027778in}{0.000000in}}{\pgfqpoint{-0.000000in}{0.000000in}}{%
\pgfpathmoveto{\pgfqpoint{-0.000000in}{0.000000in}}%
\pgfpathlineto{\pgfqpoint{-0.027778in}{0.000000in}}%
\pgfusepath{stroke,fill}%
}%
\begin{pgfscope}%
\pgfsys@transformshift{1.259630in}{3.927523in}%
\pgfsys@useobject{currentmarker}{}%
\end{pgfscope}%
\end{pgfscope}%
\begin{pgfscope}%
\pgfsetbuttcap%
\pgfsetroundjoin%
\definecolor{currentfill}{rgb}{0.000000,0.000000,0.000000}%
\pgfsetfillcolor{currentfill}%
\pgfsetlinewidth{0.602250pt}%
\definecolor{currentstroke}{rgb}{0.000000,0.000000,0.000000}%
\pgfsetstrokecolor{currentstroke}%
\pgfsetdash{}{0pt}%
\pgfsys@defobject{currentmarker}{\pgfqpoint{-0.027778in}{0.000000in}}{\pgfqpoint{-0.000000in}{0.000000in}}{%
\pgfpathmoveto{\pgfqpoint{-0.000000in}{0.000000in}}%
\pgfpathlineto{\pgfqpoint{-0.027778in}{0.000000in}}%
\pgfusepath{stroke,fill}%
}%
\begin{pgfscope}%
\pgfsys@transformshift{1.259630in}{4.071415in}%
\pgfsys@useobject{currentmarker}{}%
\end{pgfscope}%
\end{pgfscope}%
\begin{pgfscope}%
\pgfsetbuttcap%
\pgfsetroundjoin%
\definecolor{currentfill}{rgb}{0.000000,0.000000,0.000000}%
\pgfsetfillcolor{currentfill}%
\pgfsetlinewidth{0.602250pt}%
\definecolor{currentstroke}{rgb}{0.000000,0.000000,0.000000}%
\pgfsetstrokecolor{currentstroke}%
\pgfsetdash{}{0pt}%
\pgfsys@defobject{currentmarker}{\pgfqpoint{-0.027778in}{0.000000in}}{\pgfqpoint{-0.000000in}{0.000000in}}{%
\pgfpathmoveto{\pgfqpoint{-0.000000in}{0.000000in}}%
\pgfpathlineto{\pgfqpoint{-0.027778in}{0.000000in}}%
\pgfusepath{stroke,fill}%
}%
\begin{pgfscope}%
\pgfsys@transformshift{1.259630in}{4.215307in}%
\pgfsys@useobject{currentmarker}{}%
\end{pgfscope}%
\end{pgfscope}%
\begin{pgfscope}%
\pgfsetbuttcap%
\pgfsetroundjoin%
\definecolor{currentfill}{rgb}{0.000000,0.000000,0.000000}%
\pgfsetfillcolor{currentfill}%
\pgfsetlinewidth{0.602250pt}%
\definecolor{currentstroke}{rgb}{0.000000,0.000000,0.000000}%
\pgfsetstrokecolor{currentstroke}%
\pgfsetdash{}{0pt}%
\pgfsys@defobject{currentmarker}{\pgfqpoint{-0.027778in}{0.000000in}}{\pgfqpoint{-0.000000in}{0.000000in}}{%
\pgfpathmoveto{\pgfqpoint{-0.000000in}{0.000000in}}%
\pgfpathlineto{\pgfqpoint{-0.027778in}{0.000000in}}%
\pgfusepath{stroke,fill}%
}%
\begin{pgfscope}%
\pgfsys@transformshift{1.259630in}{4.359199in}%
\pgfsys@useobject{currentmarker}{}%
\end{pgfscope}%
\end{pgfscope}%
\begin{pgfscope}%
\pgfpathrectangle{\pgfqpoint{0.100000in}{0.100000in}}{\pgfqpoint{3.383066in}{4.489426in}}%
\pgfusepath{clip}%
\pgfsetrectcap%
\pgfsetroundjoin%
\pgfsetlinewidth{1.003750pt}%
\definecolor{currentstroke}{rgb}{0.000000,0.000000,1.000000}%
\pgfsetstrokecolor{currentstroke}%
\pgfsetdash{}{0pt}%
\pgfpathmoveto{\pgfqpoint{2.808299in}{4.599426in}}%
\pgfpathlineto{\pgfqpoint{2.747643in}{4.476411in}}%
\pgfpathlineto{\pgfqpoint{2.672578in}{4.329521in}}%
\pgfpathlineto{\pgfqpoint{2.607072in}{4.206477in}}%
\pgfpathlineto{\pgfqpoint{2.531274in}{4.070557in}}%
\pgfpathlineto{\pgfqpoint{2.477117in}{3.978008in}}%
\pgfpathlineto{\pgfqpoint{2.420529in}{3.885692in}}%
\pgfpathlineto{\pgfqpoint{2.359684in}{3.791804in}}%
\pgfpathlineto{\pgfqpoint{2.313044in}{3.723888in}}%
\pgfpathlineto{\pgfqpoint{2.277124in}{3.674135in}}%
\pgfpathlineto{\pgfqpoint{2.228682in}{3.610766in}}%
\pgfpathlineto{\pgfqpoint{2.184891in}{3.557377in}}%
\pgfpathlineto{\pgfqpoint{2.144288in}{3.511371in}}%
\pgfpathlineto{\pgfqpoint{2.100360in}{3.465579in}}%
\pgfpathlineto{\pgfqpoint{2.075211in}{3.441302in}}%
\pgfpathlineto{\pgfqpoint{2.031000in}{3.402171in}}%
\pgfpathlineto{\pgfqpoint{1.992070in}{3.371568in}}%
\pgfpathlineto{\pgfqpoint{1.970005in}{3.355860in}}%
\pgfpathlineto{\pgfqpoint{1.935129in}{3.333483in}}%
\pgfpathlineto{\pgfqpoint{1.897293in}{3.312627in}}%
\pgfpathlineto{\pgfqpoint{1.858109in}{3.294771in}}%
\pgfpathlineto{\pgfqpoint{1.826998in}{3.283276in}}%
\pgfpathlineto{\pgfqpoint{1.790537in}{3.272759in}}%
\pgfpathlineto{\pgfqpoint{1.758206in}{3.266016in}}%
\pgfpathlineto{\pgfqpoint{1.736638in}{3.262816in}}%
\pgfpathlineto{\pgfqpoint{1.699299in}{3.259618in}}%
\pgfpathlineto{\pgfqpoint{1.666397in}{3.259119in}}%
\pgfpathlineto{\pgfqpoint{1.625301in}{3.261293in}}%
\pgfpathlineto{\pgfqpoint{1.580484in}{3.266837in}}%
\pgfpathlineto{\pgfqpoint{1.542442in}{3.273805in}}%
\pgfpathlineto{\pgfqpoint{1.494048in}{3.285188in}}%
\pgfpathlineto{\pgfqpoint{1.439097in}{3.300829in}}%
\pgfpathlineto{\pgfqpoint{1.372681in}{3.322422in}}%
\pgfpathlineto{\pgfqpoint{1.125428in}{3.406992in}}%
\pgfpathlineto{\pgfqpoint{1.068485in}{3.423164in}}%
\pgfpathlineto{\pgfqpoint{1.005892in}{3.437915in}}%
\pgfpathlineto{\pgfqpoint{0.971725in}{3.444353in}}%
\pgfpathlineto{\pgfqpoint{0.936519in}{3.449621in}}%
\pgfpathlineto{\pgfqpoint{0.893185in}{3.453988in}}%
\pgfpathlineto{\pgfqpoint{0.852095in}{3.455735in}}%
\pgfpathlineto{\pgfqpoint{0.809523in}{3.454816in}}%
\pgfpathlineto{\pgfqpoint{0.790421in}{3.453424in}}%
\pgfpathlineto{\pgfqpoint{0.754451in}{3.449024in}}%
\pgfpathlineto{\pgfqpoint{0.717034in}{3.441792in}}%
\pgfpathlineto{\pgfqpoint{0.686219in}{3.433627in}}%
\pgfpathlineto{\pgfqpoint{0.655540in}{3.423349in}}%
\pgfpathlineto{\pgfqpoint{0.615676in}{3.406497in}}%
\pgfpathlineto{\pgfqpoint{0.591175in}{3.394005in}}%
\pgfpathlineto{\pgfqpoint{0.562872in}{3.377369in}}%
\pgfpathlineto{\pgfqpoint{0.529649in}{3.354550in}}%
\pgfpathlineto{\pgfqpoint{0.497312in}{3.328559in}}%
\pgfpathlineto{\pgfqpoint{0.479815in}{3.312784in}}%
\pgfpathlineto{\pgfqpoint{0.450289in}{3.283151in}}%
\pgfpathlineto{\pgfqpoint{0.431647in}{3.262306in}}%
\pgfpathlineto{\pgfqpoint{0.401839in}{3.225115in}}%
\pgfpathlineto{\pgfqpoint{0.379957in}{3.194412in}}%
\pgfpathlineto{\pgfqpoint{0.354454in}{3.154427in}}%
\pgfpathlineto{\pgfqpoint{0.335446in}{3.121243in}}%
\pgfpathlineto{\pgfqpoint{0.309305in}{3.070003in}}%
\pgfpathlineto{\pgfqpoint{0.286791in}{3.019574in}}%
\pgfpathlineto{\pgfqpoint{0.269461in}{2.975815in}}%
\pgfpathlineto{\pgfqpoint{0.251043in}{2.923362in}}%
\pgfpathlineto{\pgfqpoint{0.233933in}{2.867524in}}%
\pgfpathlineto{\pgfqpoint{0.220400in}{2.816851in}}%
\pgfpathlineto{\pgfqpoint{0.206867in}{2.758125in}}%
\pgfpathlineto{\pgfqpoint{0.194608in}{2.694586in}}%
\pgfpathlineto{\pgfqpoint{0.183460in}{2.622474in}}%
\pgfpathlineto{\pgfqpoint{0.176283in}{2.562480in}}%
\pgfpathlineto{\pgfqpoint{0.169869in}{2.487765in}}%
\pgfpathlineto{\pgfqpoint{0.166261in}{2.416377in}}%
\pgfpathlineto{\pgfqpoint{0.165059in}{2.344713in}}%
\pgfpathlineto{\pgfqpoint{0.165059in}{2.344713in}}%
\pgfpathlineto{\pgfqpoint{0.166767in}{2.259339in}}%
\pgfpathlineto{\pgfqpoint{0.170182in}{2.197111in}}%
\pgfpathlineto{\pgfqpoint{0.176158in}{2.128143in}}%
\pgfpathlineto{\pgfqpoint{0.183508in}{2.066594in}}%
\pgfpathlineto{\pgfqpoint{0.191490in}{2.013255in}}%
\pgfpathlineto{\pgfqpoint{0.204261in}{1.943829in}}%
\pgfpathlineto{\pgfqpoint{0.216945in}{1.886687in}}%
\pgfpathlineto{\pgfqpoint{0.234530in}{1.819815in}}%
\pgfpathlineto{\pgfqpoint{0.252944in}{1.760321in}}%
\pgfpathlineto{\pgfqpoint{0.272969in}{1.704357in}}%
\pgfpathlineto{\pgfqpoint{0.295065in}{1.650553in}}%
\pgfpathlineto{\pgfqpoint{0.318546in}{1.600494in}}%
\pgfpathlineto{\pgfqpoint{0.339371in}{1.561070in}}%
\pgfpathlineto{\pgfqpoint{0.366045in}{1.516224in}}%
\pgfpathlineto{\pgfqpoint{0.386531in}{1.485462in}}%
\pgfpathlineto{\pgfqpoint{0.416512in}{1.445376in}}%
\pgfpathlineto{\pgfqpoint{0.437942in}{1.419885in}}%
\pgfpathlineto{\pgfqpoint{0.464607in}{1.391411in}}%
\pgfpathlineto{\pgfqpoint{0.496463in}{1.361603in}}%
\pgfpathlineto{\pgfqpoint{0.521935in}{1.340722in}}%
\pgfpathlineto{\pgfqpoint{0.547559in}{1.322113in}}%
\pgfpathlineto{\pgfqpoint{0.569237in}{1.308099in}}%
\pgfpathlineto{\pgfqpoint{0.600554in}{1.290437in}}%
\pgfpathlineto{\pgfqpoint{0.635995in}{1.273824in}}%
\pgfpathlineto{\pgfqpoint{0.675111in}{1.259263in}}%
\pgfpathlineto{\pgfqpoint{0.711881in}{1.248855in}}%
\pgfpathlineto{\pgfqpoint{0.740954in}{1.242686in}}%
\pgfpathlineto{\pgfqpoint{0.769617in}{1.238254in}}%
\pgfpathlineto{\pgfqpoint{0.803773in}{1.234962in}}%
\pgfpathlineto{\pgfqpoint{0.845914in}{1.233645in}}%
\pgfpathlineto{\pgfqpoint{0.891893in}{1.235346in}}%
\pgfpathlineto{\pgfqpoint{0.933737in}{1.239452in}}%
\pgfpathlineto{\pgfqpoint{0.976310in}{1.245863in}}%
\pgfpathlineto{\pgfqpoint{1.030154in}{1.256804in}}%
\pgfpathlineto{\pgfqpoint{1.078582in}{1.268957in}}%
\pgfpathlineto{\pgfqpoint{1.147371in}{1.289231in}}%
\pgfpathlineto{\pgfqpoint{1.232854in}{1.317834in}}%
\pgfpathlineto{\pgfqpoint{1.424424in}{1.384036in}}%
\pgfpathlineto{\pgfqpoint{1.470305in}{1.397788in}}%
\pgfpathlineto{\pgfqpoint{1.529694in}{1.412876in}}%
\pgfpathlineto{\pgfqpoint{1.564867in}{1.419961in}}%
\pgfpathlineto{\pgfqpoint{1.613666in}{1.426994in}}%
\pgfpathlineto{\pgfqpoint{1.659418in}{1.430149in}}%
\pgfpathlineto{\pgfqpoint{1.679277in}{1.430359in}}%
\pgfpathlineto{\pgfqpoint{1.720263in}{1.428369in}}%
\pgfpathlineto{\pgfqpoint{1.754047in}{1.424106in}}%
\pgfpathlineto{\pgfqpoint{1.783402in}{1.418360in}}%
\pgfpathlineto{\pgfqpoint{1.816477in}{1.409507in}}%
\pgfpathlineto{\pgfqpoint{1.842277in}{1.400798in}}%
\pgfpathlineto{\pgfqpoint{1.875018in}{1.387415in}}%
\pgfpathlineto{\pgfqpoint{1.916433in}{1.366693in}}%
\pgfpathlineto{\pgfqpoint{1.939008in}{1.353603in}}%
\pgfpathlineto{\pgfqpoint{1.969603in}{1.333841in}}%
\pgfpathlineto{\pgfqpoint{2.003182in}{1.309497in}}%
\pgfpathlineto{\pgfqpoint{2.043086in}{1.277014in}}%
\pgfpathlineto{\pgfqpoint{2.088754in}{1.235229in}}%
\pgfpathlineto{\pgfqpoint{2.122225in}{1.201583in}}%
\pgfpathlineto{\pgfqpoint{2.167873in}{1.151751in}}%
\pgfpathlineto{\pgfqpoint{2.215938in}{1.094588in}}%
\pgfpathlineto{\pgfqpoint{2.259053in}{1.039443in}}%
\pgfpathlineto{\pgfqpoint{2.292597in}{0.994140in}}%
\pgfpathlineto{\pgfqpoint{2.347977in}{0.915011in}}%
\pgfpathlineto{\pgfqpoint{2.383078in}{0.862210in}}%
\pgfpathlineto{\pgfqpoint{2.448710in}{0.758340in}}%
\pgfpathlineto{\pgfqpoint{2.503269in}{0.667221in}}%
\pgfpathlineto{\pgfqpoint{2.569766in}{0.550744in}}%
\pgfpathlineto{\pgfqpoint{2.626799in}{0.446419in}}%
\pgfpathlineto{\pgfqpoint{2.693446in}{0.319681in}}%
\pgfpathlineto{\pgfqpoint{2.770473in}{0.167157in}}%
\pgfpathlineto{\pgfqpoint{2.808291in}{0.090000in}}%
\pgfpathlineto{\pgfqpoint{2.808291in}{0.090000in}}%
\pgfusepath{stroke}%
\end{pgfscope}%
\begin{pgfscope}%
\pgfsetrectcap%
\pgfsetmiterjoin%
\pgfsetlinewidth{0.803000pt}%
\definecolor{currentstroke}{rgb}{0.000000,0.000000,0.000000}%
\pgfsetstrokecolor{currentstroke}%
\pgfsetdash{}{0pt}%
\pgfpathmoveto{\pgfqpoint{1.259630in}{0.100000in}}%
\pgfpathlineto{\pgfqpoint{1.259630in}{4.589426in}}%
\pgfusepath{stroke}%
\end{pgfscope}%
\begin{pgfscope}%
\pgfsetrectcap%
\pgfsetmiterjoin%
\pgfsetlinewidth{0.803000pt}%
\definecolor{currentstroke}{rgb}{0.000000,0.000000,0.000000}%
\pgfsetstrokecolor{currentstroke}%
\pgfsetdash{}{0pt}%
\pgfpathmoveto{\pgfqpoint{0.100000in}{2.344713in}}%
\pgfpathlineto{\pgfqpoint{3.483066in}{2.344713in}}%
\pgfusepath{stroke}%
\end{pgfscope}%
\end{pgfpicture}%
\makeatother%
\endgroup%
}
      \end{figure}
      \column{0.6\textwidth}
      체 \(K\)에 대해, \(\mathrm{char}(K)\not\in\{2,3\}\)이면 \(K\) 위에서 정의된 타원 곡선은 다음 방정식의 해집합.
      \[y^2 = x^3 + ax + b\]
      여기서 \(\Delta = -16(4a^3 + 27b^2) \not= 0\)이어야 함.
    \end{columns}
  \end{frame}

  \begin{frame}
    \frametitle{유한체}
    \framesubtitle{Definition}
    덧셈과 곱셈 두 개의 이항 연산을 가지는 집합 \(F\)
    \pause
    \begin{itemize}
      \item 이항 연산은 \(F\times F\to F\)여야 함
      \pause
      \item \(a+(b+c)=(a+b)+c,\,a\cdot(b\cdot c)=(a\cdot b)\cdot c\) (결합법칙)
      \item \(a+b=b+a,\,a\cdot b=b\cdot a\) (교환법칙)
      \item \(a+0=a,\,a\cdot 1=a\)인 \(0, 1\in F\) 존재 (항등원)
      \item \(a+(-a)=0\)인 \(-a\in F\) 존재 (덧셈의 역원)
      \item \(a\not=0\)인 모든 \(a\)에 대해, \(a\cdot a^{-1}=1\)인 \(a^{-1}\in F\) 존재 (곱셈의 역원)
      \item \(a\cdot(b+c)=(a\cdot b)+(a\cdot c)\) (분배법칙)
    \end{itemize}
    \pause
    대표적인 체의 예: 유리수 \(\mathbb{Q}\), 실수 \(\mathbb{R}\)
  \end{frame}

  \begin{frame}
    \frametitle{유한체}
    \framesubtitle{Definition of Field}

    유한체: order(원소의 개수)가 유한한 체
    \begin{itemize}
      \item 유한체의 order는 \(q=p^k\) 형태를 가짐
      \item 유한체의 예: \(\Fp, \Ftwom\) 등
    \end{itemize}
  \end{frame}

  \begin{frame}
    \frametitle{유한체}
    \framesubtitle{유한체 \(\Fp\)}

    \begin{itemize}
      \item \(\Fp\)는 원소를 나열해 \(\{0,\,1,\,\cdots,\,p-1\}\)으로 나타낼 수 있음
      \item 덧셈과 곱셈은 직접 곱하거나 더한 결과의 \(p\)로 나눈 나머지를 취한 것으로 정의.
      \pause
      \item 예를 들어, \(\mathbb{F}_2\)에서 다음이 성립.
      \begin{itemize}
        \item \(1+1=0\)
        \item \((x+y)^2=x^2+y^2\) (`1학년의 꿈')
      \end{itemize}
      \pause
      \item 덧셈의 역원: \(a+x\equiv 0\;(\mathrm{mod}\:p)\)의 유일한 해로 정의
      \item 곱셈의 역원: \(a\cdot x\equiv 1\;(\mathrm{mod}\:p)\)의 유일한 해로 정의
      \pause
      \item 이 \(\Fp\)에서 타원곡선을 정의할 수 있음
      \begin{itemize}
        \item 타원곡선의 형태를 결정하는 매개변수 \(a,\,b\)가 \(\Fp\)에 존재.
      \end{itemize}
    \end{itemize}
  \end{frame}

  \begin{frame}
    \frametitle{유한체}
    \framesubtitle{\(\mathbb{F}_{71}\)위에서 정의된 타원곡선 \(y^2 = x^3 - x + 2\)}

    \begin{columns}
      \column{0.4\textwidth}
      \begin{figure}
        \raggedleft
        \scalebox{0.5}{%% Creator: Matplotlib, PGF backend
%%
%% To include the figure in your LaTeX document, write
%%   \input{<filename>.pgf}
%%
%% Make sure the required packages are loaded in your preamble
%%   \usepackage{pgf}
%%
%% Also ensure that all the required font packages are loaded; for instance,
%% the lmodern package is sometimes necessary when using math font.
%%   \usepackage{lmodern}
%%
%% Figures using additional raster images can only be included by \input if
%% they are in the same directory as the main LaTeX file. For loading figures
%% from other directories you can use the `import` package
%%   \usepackage{import}
%%
%% and then include the figures with
%%   \import{<path to file>}{<filename>.pgf}
%%
%% Matplotlib used the following preamble
%%   \usepackage{fontspec}
%%   \setmainfont{DejaVuSerif.ttf}[Path=\detokenize{/usr/lib/python3.10/site-packages/matplotlib/mpl-data/fonts/ttf/}]
%%   \setsansfont{DejaVuSans.ttf}[Path=\detokenize{/usr/lib/python3.10/site-packages/matplotlib/mpl-data/fonts/ttf/}]
%%   \setmonofont{DejaVuSansMono.ttf}[Path=\detokenize{/usr/lib/python3.10/site-packages/matplotlib/mpl-data/fonts/ttf/}]
%%
\begingroup%
\makeatletter%
\begin{pgfpicture}%
\pgfpathrectangle{\pgfpointorigin}{\pgfqpoint{3.583066in}{4.689426in}}%
\pgfusepath{use as bounding box, clip}%
\begin{pgfscope}%
\pgfsetbuttcap%
\pgfsetmiterjoin%
\definecolor{currentfill}{rgb}{1.000000,1.000000,1.000000}%
\pgfsetfillcolor{currentfill}%
\pgfsetlinewidth{0.000000pt}%
\definecolor{currentstroke}{rgb}{1.000000,1.000000,1.000000}%
\pgfsetstrokecolor{currentstroke}%
\pgfsetdash{}{0pt}%
\pgfpathmoveto{\pgfqpoint{0.000000in}{0.000000in}}%
\pgfpathlineto{\pgfqpoint{3.583066in}{0.000000in}}%
\pgfpathlineto{\pgfqpoint{3.583066in}{4.689426in}}%
\pgfpathlineto{\pgfqpoint{0.000000in}{4.689426in}}%
\pgfpathlineto{\pgfqpoint{0.000000in}{0.000000in}}%
\pgfpathclose%
\pgfusepath{fill}%
\end{pgfscope}%
\begin{pgfscope}%
\pgfsetbuttcap%
\pgfsetmiterjoin%
\definecolor{currentfill}{rgb}{1.000000,1.000000,1.000000}%
\pgfsetfillcolor{currentfill}%
\pgfsetlinewidth{0.000000pt}%
\definecolor{currentstroke}{rgb}{0.000000,0.000000,0.000000}%
\pgfsetstrokecolor{currentstroke}%
\pgfsetstrokeopacity{0.000000}%
\pgfsetdash{}{0pt}%
\pgfpathmoveto{\pgfqpoint{0.100000in}{0.100000in}}%
\pgfpathlineto{\pgfqpoint{3.483066in}{0.100000in}}%
\pgfpathlineto{\pgfqpoint{3.483066in}{4.589426in}}%
\pgfpathlineto{\pgfqpoint{0.100000in}{4.589426in}}%
\pgfpathlineto{\pgfqpoint{0.100000in}{0.100000in}}%
\pgfpathclose%
\pgfusepath{fill}%
\end{pgfscope}%
\begin{pgfscope}%
\pgfsetbuttcap%
\pgfsetroundjoin%
\definecolor{currentfill}{rgb}{0.000000,0.000000,0.000000}%
\pgfsetfillcolor{currentfill}%
\pgfsetlinewidth{0.803000pt}%
\definecolor{currentstroke}{rgb}{0.000000,0.000000,0.000000}%
\pgfsetstrokecolor{currentstroke}%
\pgfsetdash{}{0pt}%
\pgfsys@defobject{currentmarker}{\pgfqpoint{0.000000in}{-0.048611in}}{\pgfqpoint{0.000000in}{0.000000in}}{%
\pgfpathmoveto{\pgfqpoint{0.000000in}{0.000000in}}%
\pgfpathlineto{\pgfqpoint{0.000000in}{-0.048611in}}%
\pgfusepath{stroke,fill}%
}%
\begin{pgfscope}%
\pgfsys@transformshift{0.540170in}{2.344713in}%
\pgfsys@useobject{currentmarker}{}%
\end{pgfscope}%
\end{pgfscope}%
\begin{pgfscope}%
\definecolor{textcolor}{rgb}{0.000000,0.000000,0.000000}%
\pgfsetstrokecolor{textcolor}%
\pgfsetfillcolor{textcolor}%
\pgftext[x=0.540170in,y=2.247491in,,top]{\color{textcolor}\sffamily\fontsize{10.000000}{12.000000}\selectfont \ensuremath{-}1}%
\end{pgfscope}%
\begin{pgfscope}%
\pgfsetbuttcap%
\pgfsetroundjoin%
\definecolor{currentfill}{rgb}{0.000000,0.000000,0.000000}%
\pgfsetfillcolor{currentfill}%
\pgfsetlinewidth{0.803000pt}%
\definecolor{currentstroke}{rgb}{0.000000,0.000000,0.000000}%
\pgfsetstrokecolor{currentstroke}%
\pgfsetdash{}{0pt}%
\pgfsys@defobject{currentmarker}{\pgfqpoint{0.000000in}{-0.048611in}}{\pgfqpoint{0.000000in}{0.000000in}}{%
\pgfpathmoveto{\pgfqpoint{0.000000in}{0.000000in}}%
\pgfpathlineto{\pgfqpoint{0.000000in}{-0.048611in}}%
\pgfusepath{stroke,fill}%
}%
\begin{pgfscope}%
\pgfsys@transformshift{1.259630in}{2.344713in}%
\pgfsys@useobject{currentmarker}{}%
\end{pgfscope}%
\end{pgfscope}%
\begin{pgfscope}%
\pgfsetbuttcap%
\pgfsetroundjoin%
\definecolor{currentfill}{rgb}{0.000000,0.000000,0.000000}%
\pgfsetfillcolor{currentfill}%
\pgfsetlinewidth{0.803000pt}%
\definecolor{currentstroke}{rgb}{0.000000,0.000000,0.000000}%
\pgfsetstrokecolor{currentstroke}%
\pgfsetdash{}{0pt}%
\pgfsys@defobject{currentmarker}{\pgfqpoint{0.000000in}{-0.048611in}}{\pgfqpoint{0.000000in}{0.000000in}}{%
\pgfpathmoveto{\pgfqpoint{0.000000in}{0.000000in}}%
\pgfpathlineto{\pgfqpoint{0.000000in}{-0.048611in}}%
\pgfusepath{stroke,fill}%
}%
\begin{pgfscope}%
\pgfsys@transformshift{1.979089in}{2.344713in}%
\pgfsys@useobject{currentmarker}{}%
\end{pgfscope}%
\end{pgfscope}%
\begin{pgfscope}%
\definecolor{textcolor}{rgb}{0.000000,0.000000,0.000000}%
\pgfsetstrokecolor{textcolor}%
\pgfsetfillcolor{textcolor}%
\pgftext[x=1.979089in,y=2.247491in,,top]{\color{textcolor}\sffamily\fontsize{10.000000}{12.000000}\selectfont 1}%
\end{pgfscope}%
\begin{pgfscope}%
\pgfsetbuttcap%
\pgfsetroundjoin%
\definecolor{currentfill}{rgb}{0.000000,0.000000,0.000000}%
\pgfsetfillcolor{currentfill}%
\pgfsetlinewidth{0.803000pt}%
\definecolor{currentstroke}{rgb}{0.000000,0.000000,0.000000}%
\pgfsetstrokecolor{currentstroke}%
\pgfsetdash{}{0pt}%
\pgfsys@defobject{currentmarker}{\pgfqpoint{0.000000in}{-0.048611in}}{\pgfqpoint{0.000000in}{0.000000in}}{%
\pgfpathmoveto{\pgfqpoint{0.000000in}{0.000000in}}%
\pgfpathlineto{\pgfqpoint{0.000000in}{-0.048611in}}%
\pgfusepath{stroke,fill}%
}%
\begin{pgfscope}%
\pgfsys@transformshift{2.698548in}{2.344713in}%
\pgfsys@useobject{currentmarker}{}%
\end{pgfscope}%
\end{pgfscope}%
\begin{pgfscope}%
\definecolor{textcolor}{rgb}{0.000000,0.000000,0.000000}%
\pgfsetstrokecolor{textcolor}%
\pgfsetfillcolor{textcolor}%
\pgftext[x=2.698548in,y=2.247491in,,top]{\color{textcolor}\sffamily\fontsize{10.000000}{12.000000}\selectfont 2}%
\end{pgfscope}%
\begin{pgfscope}%
\pgfsetbuttcap%
\pgfsetroundjoin%
\definecolor{currentfill}{rgb}{0.000000,0.000000,0.000000}%
\pgfsetfillcolor{currentfill}%
\pgfsetlinewidth{0.803000pt}%
\definecolor{currentstroke}{rgb}{0.000000,0.000000,0.000000}%
\pgfsetstrokecolor{currentstroke}%
\pgfsetdash{}{0pt}%
\pgfsys@defobject{currentmarker}{\pgfqpoint{0.000000in}{-0.048611in}}{\pgfqpoint{0.000000in}{0.000000in}}{%
\pgfpathmoveto{\pgfqpoint{0.000000in}{0.000000in}}%
\pgfpathlineto{\pgfqpoint{0.000000in}{-0.048611in}}%
\pgfusepath{stroke,fill}%
}%
\begin{pgfscope}%
\pgfsys@transformshift{3.418007in}{2.344713in}%
\pgfsys@useobject{currentmarker}{}%
\end{pgfscope}%
\end{pgfscope}%
\begin{pgfscope}%
\definecolor{textcolor}{rgb}{0.000000,0.000000,0.000000}%
\pgfsetstrokecolor{textcolor}%
\pgfsetfillcolor{textcolor}%
\pgftext[x=3.418007in,y=2.247491in,,top]{\color{textcolor}\sffamily\fontsize{10.000000}{12.000000}\selectfont 3}%
\end{pgfscope}%
\begin{pgfscope}%
\pgfsetbuttcap%
\pgfsetroundjoin%
\definecolor{currentfill}{rgb}{0.000000,0.000000,0.000000}%
\pgfsetfillcolor{currentfill}%
\pgfsetlinewidth{0.602250pt}%
\definecolor{currentstroke}{rgb}{0.000000,0.000000,0.000000}%
\pgfsetstrokecolor{currentstroke}%
\pgfsetdash{}{0pt}%
\pgfsys@defobject{currentmarker}{\pgfqpoint{0.000000in}{-0.027778in}}{\pgfqpoint{0.000000in}{0.000000in}}{%
\pgfpathmoveto{\pgfqpoint{0.000000in}{0.000000in}}%
\pgfpathlineto{\pgfqpoint{0.000000in}{-0.027778in}}%
\pgfusepath{stroke,fill}%
}%
\begin{pgfscope}%
\pgfsys@transformshift{0.108495in}{2.344713in}%
\pgfsys@useobject{currentmarker}{}%
\end{pgfscope}%
\end{pgfscope}%
\begin{pgfscope}%
\pgfsetbuttcap%
\pgfsetroundjoin%
\definecolor{currentfill}{rgb}{0.000000,0.000000,0.000000}%
\pgfsetfillcolor{currentfill}%
\pgfsetlinewidth{0.602250pt}%
\definecolor{currentstroke}{rgb}{0.000000,0.000000,0.000000}%
\pgfsetstrokecolor{currentstroke}%
\pgfsetdash{}{0pt}%
\pgfsys@defobject{currentmarker}{\pgfqpoint{0.000000in}{-0.027778in}}{\pgfqpoint{0.000000in}{0.000000in}}{%
\pgfpathmoveto{\pgfqpoint{0.000000in}{0.000000in}}%
\pgfpathlineto{\pgfqpoint{0.000000in}{-0.027778in}}%
\pgfusepath{stroke,fill}%
}%
\begin{pgfscope}%
\pgfsys@transformshift{0.252387in}{2.344713in}%
\pgfsys@useobject{currentmarker}{}%
\end{pgfscope}%
\end{pgfscope}%
\begin{pgfscope}%
\pgfsetbuttcap%
\pgfsetroundjoin%
\definecolor{currentfill}{rgb}{0.000000,0.000000,0.000000}%
\pgfsetfillcolor{currentfill}%
\pgfsetlinewidth{0.602250pt}%
\definecolor{currentstroke}{rgb}{0.000000,0.000000,0.000000}%
\pgfsetstrokecolor{currentstroke}%
\pgfsetdash{}{0pt}%
\pgfsys@defobject{currentmarker}{\pgfqpoint{0.000000in}{-0.027778in}}{\pgfqpoint{0.000000in}{0.000000in}}{%
\pgfpathmoveto{\pgfqpoint{0.000000in}{0.000000in}}%
\pgfpathlineto{\pgfqpoint{0.000000in}{-0.027778in}}%
\pgfusepath{stroke,fill}%
}%
\begin{pgfscope}%
\pgfsys@transformshift{0.396279in}{2.344713in}%
\pgfsys@useobject{currentmarker}{}%
\end{pgfscope}%
\end{pgfscope}%
\begin{pgfscope}%
\pgfsetbuttcap%
\pgfsetroundjoin%
\definecolor{currentfill}{rgb}{0.000000,0.000000,0.000000}%
\pgfsetfillcolor{currentfill}%
\pgfsetlinewidth{0.602250pt}%
\definecolor{currentstroke}{rgb}{0.000000,0.000000,0.000000}%
\pgfsetstrokecolor{currentstroke}%
\pgfsetdash{}{0pt}%
\pgfsys@defobject{currentmarker}{\pgfqpoint{0.000000in}{-0.027778in}}{\pgfqpoint{0.000000in}{0.000000in}}{%
\pgfpathmoveto{\pgfqpoint{0.000000in}{0.000000in}}%
\pgfpathlineto{\pgfqpoint{0.000000in}{-0.027778in}}%
\pgfusepath{stroke,fill}%
}%
\begin{pgfscope}%
\pgfsys@transformshift{0.684062in}{2.344713in}%
\pgfsys@useobject{currentmarker}{}%
\end{pgfscope}%
\end{pgfscope}%
\begin{pgfscope}%
\pgfsetbuttcap%
\pgfsetroundjoin%
\definecolor{currentfill}{rgb}{0.000000,0.000000,0.000000}%
\pgfsetfillcolor{currentfill}%
\pgfsetlinewidth{0.602250pt}%
\definecolor{currentstroke}{rgb}{0.000000,0.000000,0.000000}%
\pgfsetstrokecolor{currentstroke}%
\pgfsetdash{}{0pt}%
\pgfsys@defobject{currentmarker}{\pgfqpoint{0.000000in}{-0.027778in}}{\pgfqpoint{0.000000in}{0.000000in}}{%
\pgfpathmoveto{\pgfqpoint{0.000000in}{0.000000in}}%
\pgfpathlineto{\pgfqpoint{0.000000in}{-0.027778in}}%
\pgfusepath{stroke,fill}%
}%
\begin{pgfscope}%
\pgfsys@transformshift{0.827954in}{2.344713in}%
\pgfsys@useobject{currentmarker}{}%
\end{pgfscope}%
\end{pgfscope}%
\begin{pgfscope}%
\pgfsetbuttcap%
\pgfsetroundjoin%
\definecolor{currentfill}{rgb}{0.000000,0.000000,0.000000}%
\pgfsetfillcolor{currentfill}%
\pgfsetlinewidth{0.602250pt}%
\definecolor{currentstroke}{rgb}{0.000000,0.000000,0.000000}%
\pgfsetstrokecolor{currentstroke}%
\pgfsetdash{}{0pt}%
\pgfsys@defobject{currentmarker}{\pgfqpoint{0.000000in}{-0.027778in}}{\pgfqpoint{0.000000in}{0.000000in}}{%
\pgfpathmoveto{\pgfqpoint{0.000000in}{0.000000in}}%
\pgfpathlineto{\pgfqpoint{0.000000in}{-0.027778in}}%
\pgfusepath{stroke,fill}%
}%
\begin{pgfscope}%
\pgfsys@transformshift{0.971846in}{2.344713in}%
\pgfsys@useobject{currentmarker}{}%
\end{pgfscope}%
\end{pgfscope}%
\begin{pgfscope}%
\pgfsetbuttcap%
\pgfsetroundjoin%
\definecolor{currentfill}{rgb}{0.000000,0.000000,0.000000}%
\pgfsetfillcolor{currentfill}%
\pgfsetlinewidth{0.602250pt}%
\definecolor{currentstroke}{rgb}{0.000000,0.000000,0.000000}%
\pgfsetstrokecolor{currentstroke}%
\pgfsetdash{}{0pt}%
\pgfsys@defobject{currentmarker}{\pgfqpoint{0.000000in}{-0.027778in}}{\pgfqpoint{0.000000in}{0.000000in}}{%
\pgfpathmoveto{\pgfqpoint{0.000000in}{0.000000in}}%
\pgfpathlineto{\pgfqpoint{0.000000in}{-0.027778in}}%
\pgfusepath{stroke,fill}%
}%
\begin{pgfscope}%
\pgfsys@transformshift{1.115738in}{2.344713in}%
\pgfsys@useobject{currentmarker}{}%
\end{pgfscope}%
\end{pgfscope}%
\begin{pgfscope}%
\pgfsetbuttcap%
\pgfsetroundjoin%
\definecolor{currentfill}{rgb}{0.000000,0.000000,0.000000}%
\pgfsetfillcolor{currentfill}%
\pgfsetlinewidth{0.602250pt}%
\definecolor{currentstroke}{rgb}{0.000000,0.000000,0.000000}%
\pgfsetstrokecolor{currentstroke}%
\pgfsetdash{}{0pt}%
\pgfsys@defobject{currentmarker}{\pgfqpoint{0.000000in}{-0.027778in}}{\pgfqpoint{0.000000in}{0.000000in}}{%
\pgfpathmoveto{\pgfqpoint{0.000000in}{0.000000in}}%
\pgfpathlineto{\pgfqpoint{0.000000in}{-0.027778in}}%
\pgfusepath{stroke,fill}%
}%
\begin{pgfscope}%
\pgfsys@transformshift{1.403522in}{2.344713in}%
\pgfsys@useobject{currentmarker}{}%
\end{pgfscope}%
\end{pgfscope}%
\begin{pgfscope}%
\pgfsetbuttcap%
\pgfsetroundjoin%
\definecolor{currentfill}{rgb}{0.000000,0.000000,0.000000}%
\pgfsetfillcolor{currentfill}%
\pgfsetlinewidth{0.602250pt}%
\definecolor{currentstroke}{rgb}{0.000000,0.000000,0.000000}%
\pgfsetstrokecolor{currentstroke}%
\pgfsetdash{}{0pt}%
\pgfsys@defobject{currentmarker}{\pgfqpoint{0.000000in}{-0.027778in}}{\pgfqpoint{0.000000in}{0.000000in}}{%
\pgfpathmoveto{\pgfqpoint{0.000000in}{0.000000in}}%
\pgfpathlineto{\pgfqpoint{0.000000in}{-0.027778in}}%
\pgfusepath{stroke,fill}%
}%
\begin{pgfscope}%
\pgfsys@transformshift{1.547413in}{2.344713in}%
\pgfsys@useobject{currentmarker}{}%
\end{pgfscope}%
\end{pgfscope}%
\begin{pgfscope}%
\pgfsetbuttcap%
\pgfsetroundjoin%
\definecolor{currentfill}{rgb}{0.000000,0.000000,0.000000}%
\pgfsetfillcolor{currentfill}%
\pgfsetlinewidth{0.602250pt}%
\definecolor{currentstroke}{rgb}{0.000000,0.000000,0.000000}%
\pgfsetstrokecolor{currentstroke}%
\pgfsetdash{}{0pt}%
\pgfsys@defobject{currentmarker}{\pgfqpoint{0.000000in}{-0.027778in}}{\pgfqpoint{0.000000in}{0.000000in}}{%
\pgfpathmoveto{\pgfqpoint{0.000000in}{0.000000in}}%
\pgfpathlineto{\pgfqpoint{0.000000in}{-0.027778in}}%
\pgfusepath{stroke,fill}%
}%
\begin{pgfscope}%
\pgfsys@transformshift{1.691305in}{2.344713in}%
\pgfsys@useobject{currentmarker}{}%
\end{pgfscope}%
\end{pgfscope}%
\begin{pgfscope}%
\pgfsetbuttcap%
\pgfsetroundjoin%
\definecolor{currentfill}{rgb}{0.000000,0.000000,0.000000}%
\pgfsetfillcolor{currentfill}%
\pgfsetlinewidth{0.602250pt}%
\definecolor{currentstroke}{rgb}{0.000000,0.000000,0.000000}%
\pgfsetstrokecolor{currentstroke}%
\pgfsetdash{}{0pt}%
\pgfsys@defobject{currentmarker}{\pgfqpoint{0.000000in}{-0.027778in}}{\pgfqpoint{0.000000in}{0.000000in}}{%
\pgfpathmoveto{\pgfqpoint{0.000000in}{0.000000in}}%
\pgfpathlineto{\pgfqpoint{0.000000in}{-0.027778in}}%
\pgfusepath{stroke,fill}%
}%
\begin{pgfscope}%
\pgfsys@transformshift{1.835197in}{2.344713in}%
\pgfsys@useobject{currentmarker}{}%
\end{pgfscope}%
\end{pgfscope}%
\begin{pgfscope}%
\pgfsetbuttcap%
\pgfsetroundjoin%
\definecolor{currentfill}{rgb}{0.000000,0.000000,0.000000}%
\pgfsetfillcolor{currentfill}%
\pgfsetlinewidth{0.602250pt}%
\definecolor{currentstroke}{rgb}{0.000000,0.000000,0.000000}%
\pgfsetstrokecolor{currentstroke}%
\pgfsetdash{}{0pt}%
\pgfsys@defobject{currentmarker}{\pgfqpoint{0.000000in}{-0.027778in}}{\pgfqpoint{0.000000in}{0.000000in}}{%
\pgfpathmoveto{\pgfqpoint{0.000000in}{0.000000in}}%
\pgfpathlineto{\pgfqpoint{0.000000in}{-0.027778in}}%
\pgfusepath{stroke,fill}%
}%
\begin{pgfscope}%
\pgfsys@transformshift{2.122981in}{2.344713in}%
\pgfsys@useobject{currentmarker}{}%
\end{pgfscope}%
\end{pgfscope}%
\begin{pgfscope}%
\pgfsetbuttcap%
\pgfsetroundjoin%
\definecolor{currentfill}{rgb}{0.000000,0.000000,0.000000}%
\pgfsetfillcolor{currentfill}%
\pgfsetlinewidth{0.602250pt}%
\definecolor{currentstroke}{rgb}{0.000000,0.000000,0.000000}%
\pgfsetstrokecolor{currentstroke}%
\pgfsetdash{}{0pt}%
\pgfsys@defobject{currentmarker}{\pgfqpoint{0.000000in}{-0.027778in}}{\pgfqpoint{0.000000in}{0.000000in}}{%
\pgfpathmoveto{\pgfqpoint{0.000000in}{0.000000in}}%
\pgfpathlineto{\pgfqpoint{0.000000in}{-0.027778in}}%
\pgfusepath{stroke,fill}%
}%
\begin{pgfscope}%
\pgfsys@transformshift{2.266873in}{2.344713in}%
\pgfsys@useobject{currentmarker}{}%
\end{pgfscope}%
\end{pgfscope}%
\begin{pgfscope}%
\pgfsetbuttcap%
\pgfsetroundjoin%
\definecolor{currentfill}{rgb}{0.000000,0.000000,0.000000}%
\pgfsetfillcolor{currentfill}%
\pgfsetlinewidth{0.602250pt}%
\definecolor{currentstroke}{rgb}{0.000000,0.000000,0.000000}%
\pgfsetstrokecolor{currentstroke}%
\pgfsetdash{}{0pt}%
\pgfsys@defobject{currentmarker}{\pgfqpoint{0.000000in}{-0.027778in}}{\pgfqpoint{0.000000in}{0.000000in}}{%
\pgfpathmoveto{\pgfqpoint{0.000000in}{0.000000in}}%
\pgfpathlineto{\pgfqpoint{0.000000in}{-0.027778in}}%
\pgfusepath{stroke,fill}%
}%
\begin{pgfscope}%
\pgfsys@transformshift{2.410764in}{2.344713in}%
\pgfsys@useobject{currentmarker}{}%
\end{pgfscope}%
\end{pgfscope}%
\begin{pgfscope}%
\pgfsetbuttcap%
\pgfsetroundjoin%
\definecolor{currentfill}{rgb}{0.000000,0.000000,0.000000}%
\pgfsetfillcolor{currentfill}%
\pgfsetlinewidth{0.602250pt}%
\definecolor{currentstroke}{rgb}{0.000000,0.000000,0.000000}%
\pgfsetstrokecolor{currentstroke}%
\pgfsetdash{}{0pt}%
\pgfsys@defobject{currentmarker}{\pgfqpoint{0.000000in}{-0.027778in}}{\pgfqpoint{0.000000in}{0.000000in}}{%
\pgfpathmoveto{\pgfqpoint{0.000000in}{0.000000in}}%
\pgfpathlineto{\pgfqpoint{0.000000in}{-0.027778in}}%
\pgfusepath{stroke,fill}%
}%
\begin{pgfscope}%
\pgfsys@transformshift{2.554656in}{2.344713in}%
\pgfsys@useobject{currentmarker}{}%
\end{pgfscope}%
\end{pgfscope}%
\begin{pgfscope}%
\pgfsetbuttcap%
\pgfsetroundjoin%
\definecolor{currentfill}{rgb}{0.000000,0.000000,0.000000}%
\pgfsetfillcolor{currentfill}%
\pgfsetlinewidth{0.602250pt}%
\definecolor{currentstroke}{rgb}{0.000000,0.000000,0.000000}%
\pgfsetstrokecolor{currentstroke}%
\pgfsetdash{}{0pt}%
\pgfsys@defobject{currentmarker}{\pgfqpoint{0.000000in}{-0.027778in}}{\pgfqpoint{0.000000in}{0.000000in}}{%
\pgfpathmoveto{\pgfqpoint{0.000000in}{0.000000in}}%
\pgfpathlineto{\pgfqpoint{0.000000in}{-0.027778in}}%
\pgfusepath{stroke,fill}%
}%
\begin{pgfscope}%
\pgfsys@transformshift{2.842440in}{2.344713in}%
\pgfsys@useobject{currentmarker}{}%
\end{pgfscope}%
\end{pgfscope}%
\begin{pgfscope}%
\pgfsetbuttcap%
\pgfsetroundjoin%
\definecolor{currentfill}{rgb}{0.000000,0.000000,0.000000}%
\pgfsetfillcolor{currentfill}%
\pgfsetlinewidth{0.602250pt}%
\definecolor{currentstroke}{rgb}{0.000000,0.000000,0.000000}%
\pgfsetstrokecolor{currentstroke}%
\pgfsetdash{}{0pt}%
\pgfsys@defobject{currentmarker}{\pgfqpoint{0.000000in}{-0.027778in}}{\pgfqpoint{0.000000in}{0.000000in}}{%
\pgfpathmoveto{\pgfqpoint{0.000000in}{0.000000in}}%
\pgfpathlineto{\pgfqpoint{0.000000in}{-0.027778in}}%
\pgfusepath{stroke,fill}%
}%
\begin{pgfscope}%
\pgfsys@transformshift{2.986332in}{2.344713in}%
\pgfsys@useobject{currentmarker}{}%
\end{pgfscope}%
\end{pgfscope}%
\begin{pgfscope}%
\pgfsetbuttcap%
\pgfsetroundjoin%
\definecolor{currentfill}{rgb}{0.000000,0.000000,0.000000}%
\pgfsetfillcolor{currentfill}%
\pgfsetlinewidth{0.602250pt}%
\definecolor{currentstroke}{rgb}{0.000000,0.000000,0.000000}%
\pgfsetstrokecolor{currentstroke}%
\pgfsetdash{}{0pt}%
\pgfsys@defobject{currentmarker}{\pgfqpoint{0.000000in}{-0.027778in}}{\pgfqpoint{0.000000in}{0.000000in}}{%
\pgfpathmoveto{\pgfqpoint{0.000000in}{0.000000in}}%
\pgfpathlineto{\pgfqpoint{0.000000in}{-0.027778in}}%
\pgfusepath{stroke,fill}%
}%
\begin{pgfscope}%
\pgfsys@transformshift{3.130224in}{2.344713in}%
\pgfsys@useobject{currentmarker}{}%
\end{pgfscope}%
\end{pgfscope}%
\begin{pgfscope}%
\pgfsetbuttcap%
\pgfsetroundjoin%
\definecolor{currentfill}{rgb}{0.000000,0.000000,0.000000}%
\pgfsetfillcolor{currentfill}%
\pgfsetlinewidth{0.602250pt}%
\definecolor{currentstroke}{rgb}{0.000000,0.000000,0.000000}%
\pgfsetstrokecolor{currentstroke}%
\pgfsetdash{}{0pt}%
\pgfsys@defobject{currentmarker}{\pgfqpoint{0.000000in}{-0.027778in}}{\pgfqpoint{0.000000in}{0.000000in}}{%
\pgfpathmoveto{\pgfqpoint{0.000000in}{0.000000in}}%
\pgfpathlineto{\pgfqpoint{0.000000in}{-0.027778in}}%
\pgfusepath{stroke,fill}%
}%
\begin{pgfscope}%
\pgfsys@transformshift{3.274116in}{2.344713in}%
\pgfsys@useobject{currentmarker}{}%
\end{pgfscope}%
\end{pgfscope}%
\begin{pgfscope}%
\pgfsetbuttcap%
\pgfsetroundjoin%
\definecolor{currentfill}{rgb}{0.000000,0.000000,0.000000}%
\pgfsetfillcolor{currentfill}%
\pgfsetlinewidth{0.803000pt}%
\definecolor{currentstroke}{rgb}{0.000000,0.000000,0.000000}%
\pgfsetstrokecolor{currentstroke}%
\pgfsetdash{}{0pt}%
\pgfsys@defobject{currentmarker}{\pgfqpoint{-0.048611in}{0.000000in}}{\pgfqpoint{-0.000000in}{0.000000in}}{%
\pgfpathmoveto{\pgfqpoint{-0.000000in}{0.000000in}}%
\pgfpathlineto{\pgfqpoint{-0.048611in}{0.000000in}}%
\pgfusepath{stroke,fill}%
}%
\begin{pgfscope}%
\pgfsys@transformshift{1.259630in}{0.186335in}%
\pgfsys@useobject{currentmarker}{}%
\end{pgfscope}%
\end{pgfscope}%
\begin{pgfscope}%
\definecolor{textcolor}{rgb}{0.000000,0.000000,0.000000}%
\pgfsetstrokecolor{textcolor}%
\pgfsetfillcolor{textcolor}%
\pgftext[x=0.966017in, y=0.133574in, left, base]{\color{textcolor}\sffamily\fontsize{10.000000}{12.000000}\selectfont \ensuremath{-}3}%
\end{pgfscope}%
\begin{pgfscope}%
\pgfsetbuttcap%
\pgfsetroundjoin%
\definecolor{currentfill}{rgb}{0.000000,0.000000,0.000000}%
\pgfsetfillcolor{currentfill}%
\pgfsetlinewidth{0.803000pt}%
\definecolor{currentstroke}{rgb}{0.000000,0.000000,0.000000}%
\pgfsetstrokecolor{currentstroke}%
\pgfsetdash{}{0pt}%
\pgfsys@defobject{currentmarker}{\pgfqpoint{-0.048611in}{0.000000in}}{\pgfqpoint{-0.000000in}{0.000000in}}{%
\pgfpathmoveto{\pgfqpoint{-0.000000in}{0.000000in}}%
\pgfpathlineto{\pgfqpoint{-0.048611in}{0.000000in}}%
\pgfusepath{stroke,fill}%
}%
\begin{pgfscope}%
\pgfsys@transformshift{1.259630in}{0.905794in}%
\pgfsys@useobject{currentmarker}{}%
\end{pgfscope}%
\end{pgfscope}%
\begin{pgfscope}%
\definecolor{textcolor}{rgb}{0.000000,0.000000,0.000000}%
\pgfsetstrokecolor{textcolor}%
\pgfsetfillcolor{textcolor}%
\pgftext[x=0.966017in, y=0.853033in, left, base]{\color{textcolor}\sffamily\fontsize{10.000000}{12.000000}\selectfont \ensuremath{-}2}%
\end{pgfscope}%
\begin{pgfscope}%
\pgfsetbuttcap%
\pgfsetroundjoin%
\definecolor{currentfill}{rgb}{0.000000,0.000000,0.000000}%
\pgfsetfillcolor{currentfill}%
\pgfsetlinewidth{0.803000pt}%
\definecolor{currentstroke}{rgb}{0.000000,0.000000,0.000000}%
\pgfsetstrokecolor{currentstroke}%
\pgfsetdash{}{0pt}%
\pgfsys@defobject{currentmarker}{\pgfqpoint{-0.048611in}{0.000000in}}{\pgfqpoint{-0.000000in}{0.000000in}}{%
\pgfpathmoveto{\pgfqpoint{-0.000000in}{0.000000in}}%
\pgfpathlineto{\pgfqpoint{-0.048611in}{0.000000in}}%
\pgfusepath{stroke,fill}%
}%
\begin{pgfscope}%
\pgfsys@transformshift{1.259630in}{1.625254in}%
\pgfsys@useobject{currentmarker}{}%
\end{pgfscope}%
\end{pgfscope}%
\begin{pgfscope}%
\definecolor{textcolor}{rgb}{0.000000,0.000000,0.000000}%
\pgfsetstrokecolor{textcolor}%
\pgfsetfillcolor{textcolor}%
\pgftext[x=0.966017in, y=1.572492in, left, base]{\color{textcolor}\sffamily\fontsize{10.000000}{12.000000}\selectfont \ensuremath{-}1}%
\end{pgfscope}%
\begin{pgfscope}%
\pgfsetbuttcap%
\pgfsetroundjoin%
\definecolor{currentfill}{rgb}{0.000000,0.000000,0.000000}%
\pgfsetfillcolor{currentfill}%
\pgfsetlinewidth{0.803000pt}%
\definecolor{currentstroke}{rgb}{0.000000,0.000000,0.000000}%
\pgfsetstrokecolor{currentstroke}%
\pgfsetdash{}{0pt}%
\pgfsys@defobject{currentmarker}{\pgfqpoint{-0.048611in}{0.000000in}}{\pgfqpoint{-0.000000in}{0.000000in}}{%
\pgfpathmoveto{\pgfqpoint{-0.000000in}{0.000000in}}%
\pgfpathlineto{\pgfqpoint{-0.048611in}{0.000000in}}%
\pgfusepath{stroke,fill}%
}%
\begin{pgfscope}%
\pgfsys@transformshift{1.259630in}{2.344713in}%
\pgfsys@useobject{currentmarker}{}%
\end{pgfscope}%
\end{pgfscope}%
\begin{pgfscope}%
\pgfsetbuttcap%
\pgfsetroundjoin%
\definecolor{currentfill}{rgb}{0.000000,0.000000,0.000000}%
\pgfsetfillcolor{currentfill}%
\pgfsetlinewidth{0.803000pt}%
\definecolor{currentstroke}{rgb}{0.000000,0.000000,0.000000}%
\pgfsetstrokecolor{currentstroke}%
\pgfsetdash{}{0pt}%
\pgfsys@defobject{currentmarker}{\pgfqpoint{-0.048611in}{0.000000in}}{\pgfqpoint{-0.000000in}{0.000000in}}{%
\pgfpathmoveto{\pgfqpoint{-0.000000in}{0.000000in}}%
\pgfpathlineto{\pgfqpoint{-0.048611in}{0.000000in}}%
\pgfusepath{stroke,fill}%
}%
\begin{pgfscope}%
\pgfsys@transformshift{1.259630in}{3.064172in}%
\pgfsys@useobject{currentmarker}{}%
\end{pgfscope}%
\end{pgfscope}%
\begin{pgfscope}%
\definecolor{textcolor}{rgb}{0.000000,0.000000,0.000000}%
\pgfsetstrokecolor{textcolor}%
\pgfsetfillcolor{textcolor}%
\pgftext[x=1.074042in, y=3.011411in, left, base]{\color{textcolor}\sffamily\fontsize{10.000000}{12.000000}\selectfont 1}%
\end{pgfscope}%
\begin{pgfscope}%
\pgfsetbuttcap%
\pgfsetroundjoin%
\definecolor{currentfill}{rgb}{0.000000,0.000000,0.000000}%
\pgfsetfillcolor{currentfill}%
\pgfsetlinewidth{0.803000pt}%
\definecolor{currentstroke}{rgb}{0.000000,0.000000,0.000000}%
\pgfsetstrokecolor{currentstroke}%
\pgfsetdash{}{0pt}%
\pgfsys@defobject{currentmarker}{\pgfqpoint{-0.048611in}{0.000000in}}{\pgfqpoint{-0.000000in}{0.000000in}}{%
\pgfpathmoveto{\pgfqpoint{-0.000000in}{0.000000in}}%
\pgfpathlineto{\pgfqpoint{-0.048611in}{0.000000in}}%
\pgfusepath{stroke,fill}%
}%
\begin{pgfscope}%
\pgfsys@transformshift{1.259630in}{3.783631in}%
\pgfsys@useobject{currentmarker}{}%
\end{pgfscope}%
\end{pgfscope}%
\begin{pgfscope}%
\definecolor{textcolor}{rgb}{0.000000,0.000000,0.000000}%
\pgfsetstrokecolor{textcolor}%
\pgfsetfillcolor{textcolor}%
\pgftext[x=1.074042in, y=3.730870in, left, base]{\color{textcolor}\sffamily\fontsize{10.000000}{12.000000}\selectfont 2}%
\end{pgfscope}%
\begin{pgfscope}%
\pgfsetbuttcap%
\pgfsetroundjoin%
\definecolor{currentfill}{rgb}{0.000000,0.000000,0.000000}%
\pgfsetfillcolor{currentfill}%
\pgfsetlinewidth{0.803000pt}%
\definecolor{currentstroke}{rgb}{0.000000,0.000000,0.000000}%
\pgfsetstrokecolor{currentstroke}%
\pgfsetdash{}{0pt}%
\pgfsys@defobject{currentmarker}{\pgfqpoint{-0.048611in}{0.000000in}}{\pgfqpoint{-0.000000in}{0.000000in}}{%
\pgfpathmoveto{\pgfqpoint{-0.000000in}{0.000000in}}%
\pgfpathlineto{\pgfqpoint{-0.048611in}{0.000000in}}%
\pgfusepath{stroke,fill}%
}%
\begin{pgfscope}%
\pgfsys@transformshift{1.259630in}{4.503091in}%
\pgfsys@useobject{currentmarker}{}%
\end{pgfscope}%
\end{pgfscope}%
\begin{pgfscope}%
\definecolor{textcolor}{rgb}{0.000000,0.000000,0.000000}%
\pgfsetstrokecolor{textcolor}%
\pgfsetfillcolor{textcolor}%
\pgftext[x=1.074042in, y=4.450329in, left, base]{\color{textcolor}\sffamily\fontsize{10.000000}{12.000000}\selectfont 3}%
\end{pgfscope}%
\begin{pgfscope}%
\pgfsetbuttcap%
\pgfsetroundjoin%
\definecolor{currentfill}{rgb}{0.000000,0.000000,0.000000}%
\pgfsetfillcolor{currentfill}%
\pgfsetlinewidth{0.602250pt}%
\definecolor{currentstroke}{rgb}{0.000000,0.000000,0.000000}%
\pgfsetstrokecolor{currentstroke}%
\pgfsetdash{}{0pt}%
\pgfsys@defobject{currentmarker}{\pgfqpoint{-0.027778in}{0.000000in}}{\pgfqpoint{-0.000000in}{0.000000in}}{%
\pgfpathmoveto{\pgfqpoint{-0.000000in}{0.000000in}}%
\pgfpathlineto{\pgfqpoint{-0.027778in}{0.000000in}}%
\pgfusepath{stroke,fill}%
}%
\begin{pgfscope}%
\pgfsys@transformshift{1.259630in}{0.330227in}%
\pgfsys@useobject{currentmarker}{}%
\end{pgfscope}%
\end{pgfscope}%
\begin{pgfscope}%
\pgfsetbuttcap%
\pgfsetroundjoin%
\definecolor{currentfill}{rgb}{0.000000,0.000000,0.000000}%
\pgfsetfillcolor{currentfill}%
\pgfsetlinewidth{0.602250pt}%
\definecolor{currentstroke}{rgb}{0.000000,0.000000,0.000000}%
\pgfsetstrokecolor{currentstroke}%
\pgfsetdash{}{0pt}%
\pgfsys@defobject{currentmarker}{\pgfqpoint{-0.027778in}{0.000000in}}{\pgfqpoint{-0.000000in}{0.000000in}}{%
\pgfpathmoveto{\pgfqpoint{-0.000000in}{0.000000in}}%
\pgfpathlineto{\pgfqpoint{-0.027778in}{0.000000in}}%
\pgfusepath{stroke,fill}%
}%
\begin{pgfscope}%
\pgfsys@transformshift{1.259630in}{0.474119in}%
\pgfsys@useobject{currentmarker}{}%
\end{pgfscope}%
\end{pgfscope}%
\begin{pgfscope}%
\pgfsetbuttcap%
\pgfsetroundjoin%
\definecolor{currentfill}{rgb}{0.000000,0.000000,0.000000}%
\pgfsetfillcolor{currentfill}%
\pgfsetlinewidth{0.602250pt}%
\definecolor{currentstroke}{rgb}{0.000000,0.000000,0.000000}%
\pgfsetstrokecolor{currentstroke}%
\pgfsetdash{}{0pt}%
\pgfsys@defobject{currentmarker}{\pgfqpoint{-0.027778in}{0.000000in}}{\pgfqpoint{-0.000000in}{0.000000in}}{%
\pgfpathmoveto{\pgfqpoint{-0.000000in}{0.000000in}}%
\pgfpathlineto{\pgfqpoint{-0.027778in}{0.000000in}}%
\pgfusepath{stroke,fill}%
}%
\begin{pgfscope}%
\pgfsys@transformshift{1.259630in}{0.618011in}%
\pgfsys@useobject{currentmarker}{}%
\end{pgfscope}%
\end{pgfscope}%
\begin{pgfscope}%
\pgfsetbuttcap%
\pgfsetroundjoin%
\definecolor{currentfill}{rgb}{0.000000,0.000000,0.000000}%
\pgfsetfillcolor{currentfill}%
\pgfsetlinewidth{0.602250pt}%
\definecolor{currentstroke}{rgb}{0.000000,0.000000,0.000000}%
\pgfsetstrokecolor{currentstroke}%
\pgfsetdash{}{0pt}%
\pgfsys@defobject{currentmarker}{\pgfqpoint{-0.027778in}{0.000000in}}{\pgfqpoint{-0.000000in}{0.000000in}}{%
\pgfpathmoveto{\pgfqpoint{-0.000000in}{0.000000in}}%
\pgfpathlineto{\pgfqpoint{-0.027778in}{0.000000in}}%
\pgfusepath{stroke,fill}%
}%
\begin{pgfscope}%
\pgfsys@transformshift{1.259630in}{0.761903in}%
\pgfsys@useobject{currentmarker}{}%
\end{pgfscope}%
\end{pgfscope}%
\begin{pgfscope}%
\pgfsetbuttcap%
\pgfsetroundjoin%
\definecolor{currentfill}{rgb}{0.000000,0.000000,0.000000}%
\pgfsetfillcolor{currentfill}%
\pgfsetlinewidth{0.602250pt}%
\definecolor{currentstroke}{rgb}{0.000000,0.000000,0.000000}%
\pgfsetstrokecolor{currentstroke}%
\pgfsetdash{}{0pt}%
\pgfsys@defobject{currentmarker}{\pgfqpoint{-0.027778in}{0.000000in}}{\pgfqpoint{-0.000000in}{0.000000in}}{%
\pgfpathmoveto{\pgfqpoint{-0.000000in}{0.000000in}}%
\pgfpathlineto{\pgfqpoint{-0.027778in}{0.000000in}}%
\pgfusepath{stroke,fill}%
}%
\begin{pgfscope}%
\pgfsys@transformshift{1.259630in}{1.049686in}%
\pgfsys@useobject{currentmarker}{}%
\end{pgfscope}%
\end{pgfscope}%
\begin{pgfscope}%
\pgfsetbuttcap%
\pgfsetroundjoin%
\definecolor{currentfill}{rgb}{0.000000,0.000000,0.000000}%
\pgfsetfillcolor{currentfill}%
\pgfsetlinewidth{0.602250pt}%
\definecolor{currentstroke}{rgb}{0.000000,0.000000,0.000000}%
\pgfsetstrokecolor{currentstroke}%
\pgfsetdash{}{0pt}%
\pgfsys@defobject{currentmarker}{\pgfqpoint{-0.027778in}{0.000000in}}{\pgfqpoint{-0.000000in}{0.000000in}}{%
\pgfpathmoveto{\pgfqpoint{-0.000000in}{0.000000in}}%
\pgfpathlineto{\pgfqpoint{-0.027778in}{0.000000in}}%
\pgfusepath{stroke,fill}%
}%
\begin{pgfscope}%
\pgfsys@transformshift{1.259630in}{1.193578in}%
\pgfsys@useobject{currentmarker}{}%
\end{pgfscope}%
\end{pgfscope}%
\begin{pgfscope}%
\pgfsetbuttcap%
\pgfsetroundjoin%
\definecolor{currentfill}{rgb}{0.000000,0.000000,0.000000}%
\pgfsetfillcolor{currentfill}%
\pgfsetlinewidth{0.602250pt}%
\definecolor{currentstroke}{rgb}{0.000000,0.000000,0.000000}%
\pgfsetstrokecolor{currentstroke}%
\pgfsetdash{}{0pt}%
\pgfsys@defobject{currentmarker}{\pgfqpoint{-0.027778in}{0.000000in}}{\pgfqpoint{-0.000000in}{0.000000in}}{%
\pgfpathmoveto{\pgfqpoint{-0.000000in}{0.000000in}}%
\pgfpathlineto{\pgfqpoint{-0.027778in}{0.000000in}}%
\pgfusepath{stroke,fill}%
}%
\begin{pgfscope}%
\pgfsys@transformshift{1.259630in}{1.337470in}%
\pgfsys@useobject{currentmarker}{}%
\end{pgfscope}%
\end{pgfscope}%
\begin{pgfscope}%
\pgfsetbuttcap%
\pgfsetroundjoin%
\definecolor{currentfill}{rgb}{0.000000,0.000000,0.000000}%
\pgfsetfillcolor{currentfill}%
\pgfsetlinewidth{0.602250pt}%
\definecolor{currentstroke}{rgb}{0.000000,0.000000,0.000000}%
\pgfsetstrokecolor{currentstroke}%
\pgfsetdash{}{0pt}%
\pgfsys@defobject{currentmarker}{\pgfqpoint{-0.027778in}{0.000000in}}{\pgfqpoint{-0.000000in}{0.000000in}}{%
\pgfpathmoveto{\pgfqpoint{-0.000000in}{0.000000in}}%
\pgfpathlineto{\pgfqpoint{-0.027778in}{0.000000in}}%
\pgfusepath{stroke,fill}%
}%
\begin{pgfscope}%
\pgfsys@transformshift{1.259630in}{1.481362in}%
\pgfsys@useobject{currentmarker}{}%
\end{pgfscope}%
\end{pgfscope}%
\begin{pgfscope}%
\pgfsetbuttcap%
\pgfsetroundjoin%
\definecolor{currentfill}{rgb}{0.000000,0.000000,0.000000}%
\pgfsetfillcolor{currentfill}%
\pgfsetlinewidth{0.602250pt}%
\definecolor{currentstroke}{rgb}{0.000000,0.000000,0.000000}%
\pgfsetstrokecolor{currentstroke}%
\pgfsetdash{}{0pt}%
\pgfsys@defobject{currentmarker}{\pgfqpoint{-0.027778in}{0.000000in}}{\pgfqpoint{-0.000000in}{0.000000in}}{%
\pgfpathmoveto{\pgfqpoint{-0.000000in}{0.000000in}}%
\pgfpathlineto{\pgfqpoint{-0.027778in}{0.000000in}}%
\pgfusepath{stroke,fill}%
}%
\begin{pgfscope}%
\pgfsys@transformshift{1.259630in}{1.769145in}%
\pgfsys@useobject{currentmarker}{}%
\end{pgfscope}%
\end{pgfscope}%
\begin{pgfscope}%
\pgfsetbuttcap%
\pgfsetroundjoin%
\definecolor{currentfill}{rgb}{0.000000,0.000000,0.000000}%
\pgfsetfillcolor{currentfill}%
\pgfsetlinewidth{0.602250pt}%
\definecolor{currentstroke}{rgb}{0.000000,0.000000,0.000000}%
\pgfsetstrokecolor{currentstroke}%
\pgfsetdash{}{0pt}%
\pgfsys@defobject{currentmarker}{\pgfqpoint{-0.027778in}{0.000000in}}{\pgfqpoint{-0.000000in}{0.000000in}}{%
\pgfpathmoveto{\pgfqpoint{-0.000000in}{0.000000in}}%
\pgfpathlineto{\pgfqpoint{-0.027778in}{0.000000in}}%
\pgfusepath{stroke,fill}%
}%
\begin{pgfscope}%
\pgfsys@transformshift{1.259630in}{1.913037in}%
\pgfsys@useobject{currentmarker}{}%
\end{pgfscope}%
\end{pgfscope}%
\begin{pgfscope}%
\pgfsetbuttcap%
\pgfsetroundjoin%
\definecolor{currentfill}{rgb}{0.000000,0.000000,0.000000}%
\pgfsetfillcolor{currentfill}%
\pgfsetlinewidth{0.602250pt}%
\definecolor{currentstroke}{rgb}{0.000000,0.000000,0.000000}%
\pgfsetstrokecolor{currentstroke}%
\pgfsetdash{}{0pt}%
\pgfsys@defobject{currentmarker}{\pgfqpoint{-0.027778in}{0.000000in}}{\pgfqpoint{-0.000000in}{0.000000in}}{%
\pgfpathmoveto{\pgfqpoint{-0.000000in}{0.000000in}}%
\pgfpathlineto{\pgfqpoint{-0.027778in}{0.000000in}}%
\pgfusepath{stroke,fill}%
}%
\begin{pgfscope}%
\pgfsys@transformshift{1.259630in}{2.056929in}%
\pgfsys@useobject{currentmarker}{}%
\end{pgfscope}%
\end{pgfscope}%
\begin{pgfscope}%
\pgfsetbuttcap%
\pgfsetroundjoin%
\definecolor{currentfill}{rgb}{0.000000,0.000000,0.000000}%
\pgfsetfillcolor{currentfill}%
\pgfsetlinewidth{0.602250pt}%
\definecolor{currentstroke}{rgb}{0.000000,0.000000,0.000000}%
\pgfsetstrokecolor{currentstroke}%
\pgfsetdash{}{0pt}%
\pgfsys@defobject{currentmarker}{\pgfqpoint{-0.027778in}{0.000000in}}{\pgfqpoint{-0.000000in}{0.000000in}}{%
\pgfpathmoveto{\pgfqpoint{-0.000000in}{0.000000in}}%
\pgfpathlineto{\pgfqpoint{-0.027778in}{0.000000in}}%
\pgfusepath{stroke,fill}%
}%
\begin{pgfscope}%
\pgfsys@transformshift{1.259630in}{2.200821in}%
\pgfsys@useobject{currentmarker}{}%
\end{pgfscope}%
\end{pgfscope}%
\begin{pgfscope}%
\pgfsetbuttcap%
\pgfsetroundjoin%
\definecolor{currentfill}{rgb}{0.000000,0.000000,0.000000}%
\pgfsetfillcolor{currentfill}%
\pgfsetlinewidth{0.602250pt}%
\definecolor{currentstroke}{rgb}{0.000000,0.000000,0.000000}%
\pgfsetstrokecolor{currentstroke}%
\pgfsetdash{}{0pt}%
\pgfsys@defobject{currentmarker}{\pgfqpoint{-0.027778in}{0.000000in}}{\pgfqpoint{-0.000000in}{0.000000in}}{%
\pgfpathmoveto{\pgfqpoint{-0.000000in}{0.000000in}}%
\pgfpathlineto{\pgfqpoint{-0.027778in}{0.000000in}}%
\pgfusepath{stroke,fill}%
}%
\begin{pgfscope}%
\pgfsys@transformshift{1.259630in}{2.488605in}%
\pgfsys@useobject{currentmarker}{}%
\end{pgfscope}%
\end{pgfscope}%
\begin{pgfscope}%
\pgfsetbuttcap%
\pgfsetroundjoin%
\definecolor{currentfill}{rgb}{0.000000,0.000000,0.000000}%
\pgfsetfillcolor{currentfill}%
\pgfsetlinewidth{0.602250pt}%
\definecolor{currentstroke}{rgb}{0.000000,0.000000,0.000000}%
\pgfsetstrokecolor{currentstroke}%
\pgfsetdash{}{0pt}%
\pgfsys@defobject{currentmarker}{\pgfqpoint{-0.027778in}{0.000000in}}{\pgfqpoint{-0.000000in}{0.000000in}}{%
\pgfpathmoveto{\pgfqpoint{-0.000000in}{0.000000in}}%
\pgfpathlineto{\pgfqpoint{-0.027778in}{0.000000in}}%
\pgfusepath{stroke,fill}%
}%
\begin{pgfscope}%
\pgfsys@transformshift{1.259630in}{2.632497in}%
\pgfsys@useobject{currentmarker}{}%
\end{pgfscope}%
\end{pgfscope}%
\begin{pgfscope}%
\pgfsetbuttcap%
\pgfsetroundjoin%
\definecolor{currentfill}{rgb}{0.000000,0.000000,0.000000}%
\pgfsetfillcolor{currentfill}%
\pgfsetlinewidth{0.602250pt}%
\definecolor{currentstroke}{rgb}{0.000000,0.000000,0.000000}%
\pgfsetstrokecolor{currentstroke}%
\pgfsetdash{}{0pt}%
\pgfsys@defobject{currentmarker}{\pgfqpoint{-0.027778in}{0.000000in}}{\pgfqpoint{-0.000000in}{0.000000in}}{%
\pgfpathmoveto{\pgfqpoint{-0.000000in}{0.000000in}}%
\pgfpathlineto{\pgfqpoint{-0.027778in}{0.000000in}}%
\pgfusepath{stroke,fill}%
}%
\begin{pgfscope}%
\pgfsys@transformshift{1.259630in}{2.776388in}%
\pgfsys@useobject{currentmarker}{}%
\end{pgfscope}%
\end{pgfscope}%
\begin{pgfscope}%
\pgfsetbuttcap%
\pgfsetroundjoin%
\definecolor{currentfill}{rgb}{0.000000,0.000000,0.000000}%
\pgfsetfillcolor{currentfill}%
\pgfsetlinewidth{0.602250pt}%
\definecolor{currentstroke}{rgb}{0.000000,0.000000,0.000000}%
\pgfsetstrokecolor{currentstroke}%
\pgfsetdash{}{0pt}%
\pgfsys@defobject{currentmarker}{\pgfqpoint{-0.027778in}{0.000000in}}{\pgfqpoint{-0.000000in}{0.000000in}}{%
\pgfpathmoveto{\pgfqpoint{-0.000000in}{0.000000in}}%
\pgfpathlineto{\pgfqpoint{-0.027778in}{0.000000in}}%
\pgfusepath{stroke,fill}%
}%
\begin{pgfscope}%
\pgfsys@transformshift{1.259630in}{2.920280in}%
\pgfsys@useobject{currentmarker}{}%
\end{pgfscope}%
\end{pgfscope}%
\begin{pgfscope}%
\pgfsetbuttcap%
\pgfsetroundjoin%
\definecolor{currentfill}{rgb}{0.000000,0.000000,0.000000}%
\pgfsetfillcolor{currentfill}%
\pgfsetlinewidth{0.602250pt}%
\definecolor{currentstroke}{rgb}{0.000000,0.000000,0.000000}%
\pgfsetstrokecolor{currentstroke}%
\pgfsetdash{}{0pt}%
\pgfsys@defobject{currentmarker}{\pgfqpoint{-0.027778in}{0.000000in}}{\pgfqpoint{-0.000000in}{0.000000in}}{%
\pgfpathmoveto{\pgfqpoint{-0.000000in}{0.000000in}}%
\pgfpathlineto{\pgfqpoint{-0.027778in}{0.000000in}}%
\pgfusepath{stroke,fill}%
}%
\begin{pgfscope}%
\pgfsys@transformshift{1.259630in}{3.208064in}%
\pgfsys@useobject{currentmarker}{}%
\end{pgfscope}%
\end{pgfscope}%
\begin{pgfscope}%
\pgfsetbuttcap%
\pgfsetroundjoin%
\definecolor{currentfill}{rgb}{0.000000,0.000000,0.000000}%
\pgfsetfillcolor{currentfill}%
\pgfsetlinewidth{0.602250pt}%
\definecolor{currentstroke}{rgb}{0.000000,0.000000,0.000000}%
\pgfsetstrokecolor{currentstroke}%
\pgfsetdash{}{0pt}%
\pgfsys@defobject{currentmarker}{\pgfqpoint{-0.027778in}{0.000000in}}{\pgfqpoint{-0.000000in}{0.000000in}}{%
\pgfpathmoveto{\pgfqpoint{-0.000000in}{0.000000in}}%
\pgfpathlineto{\pgfqpoint{-0.027778in}{0.000000in}}%
\pgfusepath{stroke,fill}%
}%
\begin{pgfscope}%
\pgfsys@transformshift{1.259630in}{3.351956in}%
\pgfsys@useobject{currentmarker}{}%
\end{pgfscope}%
\end{pgfscope}%
\begin{pgfscope}%
\pgfsetbuttcap%
\pgfsetroundjoin%
\definecolor{currentfill}{rgb}{0.000000,0.000000,0.000000}%
\pgfsetfillcolor{currentfill}%
\pgfsetlinewidth{0.602250pt}%
\definecolor{currentstroke}{rgb}{0.000000,0.000000,0.000000}%
\pgfsetstrokecolor{currentstroke}%
\pgfsetdash{}{0pt}%
\pgfsys@defobject{currentmarker}{\pgfqpoint{-0.027778in}{0.000000in}}{\pgfqpoint{-0.000000in}{0.000000in}}{%
\pgfpathmoveto{\pgfqpoint{-0.000000in}{0.000000in}}%
\pgfpathlineto{\pgfqpoint{-0.027778in}{0.000000in}}%
\pgfusepath{stroke,fill}%
}%
\begin{pgfscope}%
\pgfsys@transformshift{1.259630in}{3.495848in}%
\pgfsys@useobject{currentmarker}{}%
\end{pgfscope}%
\end{pgfscope}%
\begin{pgfscope}%
\pgfsetbuttcap%
\pgfsetroundjoin%
\definecolor{currentfill}{rgb}{0.000000,0.000000,0.000000}%
\pgfsetfillcolor{currentfill}%
\pgfsetlinewidth{0.602250pt}%
\definecolor{currentstroke}{rgb}{0.000000,0.000000,0.000000}%
\pgfsetstrokecolor{currentstroke}%
\pgfsetdash{}{0pt}%
\pgfsys@defobject{currentmarker}{\pgfqpoint{-0.027778in}{0.000000in}}{\pgfqpoint{-0.000000in}{0.000000in}}{%
\pgfpathmoveto{\pgfqpoint{-0.000000in}{0.000000in}}%
\pgfpathlineto{\pgfqpoint{-0.027778in}{0.000000in}}%
\pgfusepath{stroke,fill}%
}%
\begin{pgfscope}%
\pgfsys@transformshift{1.259630in}{3.639739in}%
\pgfsys@useobject{currentmarker}{}%
\end{pgfscope}%
\end{pgfscope}%
\begin{pgfscope}%
\pgfsetbuttcap%
\pgfsetroundjoin%
\definecolor{currentfill}{rgb}{0.000000,0.000000,0.000000}%
\pgfsetfillcolor{currentfill}%
\pgfsetlinewidth{0.602250pt}%
\definecolor{currentstroke}{rgb}{0.000000,0.000000,0.000000}%
\pgfsetstrokecolor{currentstroke}%
\pgfsetdash{}{0pt}%
\pgfsys@defobject{currentmarker}{\pgfqpoint{-0.027778in}{0.000000in}}{\pgfqpoint{-0.000000in}{0.000000in}}{%
\pgfpathmoveto{\pgfqpoint{-0.000000in}{0.000000in}}%
\pgfpathlineto{\pgfqpoint{-0.027778in}{0.000000in}}%
\pgfusepath{stroke,fill}%
}%
\begin{pgfscope}%
\pgfsys@transformshift{1.259630in}{3.927523in}%
\pgfsys@useobject{currentmarker}{}%
\end{pgfscope}%
\end{pgfscope}%
\begin{pgfscope}%
\pgfsetbuttcap%
\pgfsetroundjoin%
\definecolor{currentfill}{rgb}{0.000000,0.000000,0.000000}%
\pgfsetfillcolor{currentfill}%
\pgfsetlinewidth{0.602250pt}%
\definecolor{currentstroke}{rgb}{0.000000,0.000000,0.000000}%
\pgfsetstrokecolor{currentstroke}%
\pgfsetdash{}{0pt}%
\pgfsys@defobject{currentmarker}{\pgfqpoint{-0.027778in}{0.000000in}}{\pgfqpoint{-0.000000in}{0.000000in}}{%
\pgfpathmoveto{\pgfqpoint{-0.000000in}{0.000000in}}%
\pgfpathlineto{\pgfqpoint{-0.027778in}{0.000000in}}%
\pgfusepath{stroke,fill}%
}%
\begin{pgfscope}%
\pgfsys@transformshift{1.259630in}{4.071415in}%
\pgfsys@useobject{currentmarker}{}%
\end{pgfscope}%
\end{pgfscope}%
\begin{pgfscope}%
\pgfsetbuttcap%
\pgfsetroundjoin%
\definecolor{currentfill}{rgb}{0.000000,0.000000,0.000000}%
\pgfsetfillcolor{currentfill}%
\pgfsetlinewidth{0.602250pt}%
\definecolor{currentstroke}{rgb}{0.000000,0.000000,0.000000}%
\pgfsetstrokecolor{currentstroke}%
\pgfsetdash{}{0pt}%
\pgfsys@defobject{currentmarker}{\pgfqpoint{-0.027778in}{0.000000in}}{\pgfqpoint{-0.000000in}{0.000000in}}{%
\pgfpathmoveto{\pgfqpoint{-0.000000in}{0.000000in}}%
\pgfpathlineto{\pgfqpoint{-0.027778in}{0.000000in}}%
\pgfusepath{stroke,fill}%
}%
\begin{pgfscope}%
\pgfsys@transformshift{1.259630in}{4.215307in}%
\pgfsys@useobject{currentmarker}{}%
\end{pgfscope}%
\end{pgfscope}%
\begin{pgfscope}%
\pgfsetbuttcap%
\pgfsetroundjoin%
\definecolor{currentfill}{rgb}{0.000000,0.000000,0.000000}%
\pgfsetfillcolor{currentfill}%
\pgfsetlinewidth{0.602250pt}%
\definecolor{currentstroke}{rgb}{0.000000,0.000000,0.000000}%
\pgfsetstrokecolor{currentstroke}%
\pgfsetdash{}{0pt}%
\pgfsys@defobject{currentmarker}{\pgfqpoint{-0.027778in}{0.000000in}}{\pgfqpoint{-0.000000in}{0.000000in}}{%
\pgfpathmoveto{\pgfqpoint{-0.000000in}{0.000000in}}%
\pgfpathlineto{\pgfqpoint{-0.027778in}{0.000000in}}%
\pgfusepath{stroke,fill}%
}%
\begin{pgfscope}%
\pgfsys@transformshift{1.259630in}{4.359199in}%
\pgfsys@useobject{currentmarker}{}%
\end{pgfscope}%
\end{pgfscope}%
\begin{pgfscope}%
\pgfpathrectangle{\pgfqpoint{0.100000in}{0.100000in}}{\pgfqpoint{3.383066in}{4.489426in}}%
\pgfusepath{clip}%
\pgfsetrectcap%
\pgfsetroundjoin%
\pgfsetlinewidth{1.003750pt}%
\definecolor{currentstroke}{rgb}{0.000000,0.000000,1.000000}%
\pgfsetstrokecolor{currentstroke}%
\pgfsetdash{}{0pt}%
\pgfpathmoveto{\pgfqpoint{2.808299in}{4.599426in}}%
\pgfpathlineto{\pgfqpoint{2.747643in}{4.476411in}}%
\pgfpathlineto{\pgfqpoint{2.672578in}{4.329521in}}%
\pgfpathlineto{\pgfqpoint{2.607072in}{4.206477in}}%
\pgfpathlineto{\pgfqpoint{2.531274in}{4.070557in}}%
\pgfpathlineto{\pgfqpoint{2.477117in}{3.978008in}}%
\pgfpathlineto{\pgfqpoint{2.420529in}{3.885692in}}%
\pgfpathlineto{\pgfqpoint{2.359684in}{3.791804in}}%
\pgfpathlineto{\pgfqpoint{2.313044in}{3.723888in}}%
\pgfpathlineto{\pgfqpoint{2.277124in}{3.674135in}}%
\pgfpathlineto{\pgfqpoint{2.228682in}{3.610766in}}%
\pgfpathlineto{\pgfqpoint{2.184891in}{3.557377in}}%
\pgfpathlineto{\pgfqpoint{2.144288in}{3.511371in}}%
\pgfpathlineto{\pgfqpoint{2.100360in}{3.465579in}}%
\pgfpathlineto{\pgfqpoint{2.075211in}{3.441302in}}%
\pgfpathlineto{\pgfqpoint{2.031000in}{3.402171in}}%
\pgfpathlineto{\pgfqpoint{1.992070in}{3.371568in}}%
\pgfpathlineto{\pgfqpoint{1.970005in}{3.355860in}}%
\pgfpathlineto{\pgfqpoint{1.935129in}{3.333483in}}%
\pgfpathlineto{\pgfqpoint{1.897293in}{3.312627in}}%
\pgfpathlineto{\pgfqpoint{1.858109in}{3.294771in}}%
\pgfpathlineto{\pgfqpoint{1.826998in}{3.283276in}}%
\pgfpathlineto{\pgfqpoint{1.790537in}{3.272759in}}%
\pgfpathlineto{\pgfqpoint{1.758206in}{3.266016in}}%
\pgfpathlineto{\pgfqpoint{1.736638in}{3.262816in}}%
\pgfpathlineto{\pgfqpoint{1.699299in}{3.259618in}}%
\pgfpathlineto{\pgfqpoint{1.666397in}{3.259119in}}%
\pgfpathlineto{\pgfqpoint{1.625301in}{3.261293in}}%
\pgfpathlineto{\pgfqpoint{1.580484in}{3.266837in}}%
\pgfpathlineto{\pgfqpoint{1.542442in}{3.273805in}}%
\pgfpathlineto{\pgfqpoint{1.494048in}{3.285188in}}%
\pgfpathlineto{\pgfqpoint{1.439097in}{3.300829in}}%
\pgfpathlineto{\pgfqpoint{1.372681in}{3.322422in}}%
\pgfpathlineto{\pgfqpoint{1.125428in}{3.406992in}}%
\pgfpathlineto{\pgfqpoint{1.068485in}{3.423164in}}%
\pgfpathlineto{\pgfqpoint{1.005892in}{3.437915in}}%
\pgfpathlineto{\pgfqpoint{0.971725in}{3.444353in}}%
\pgfpathlineto{\pgfqpoint{0.936519in}{3.449621in}}%
\pgfpathlineto{\pgfqpoint{0.893185in}{3.453988in}}%
\pgfpathlineto{\pgfqpoint{0.852095in}{3.455735in}}%
\pgfpathlineto{\pgfqpoint{0.809523in}{3.454816in}}%
\pgfpathlineto{\pgfqpoint{0.790421in}{3.453424in}}%
\pgfpathlineto{\pgfqpoint{0.754451in}{3.449024in}}%
\pgfpathlineto{\pgfqpoint{0.717034in}{3.441792in}}%
\pgfpathlineto{\pgfqpoint{0.686219in}{3.433627in}}%
\pgfpathlineto{\pgfqpoint{0.655540in}{3.423349in}}%
\pgfpathlineto{\pgfqpoint{0.615676in}{3.406497in}}%
\pgfpathlineto{\pgfqpoint{0.591175in}{3.394005in}}%
\pgfpathlineto{\pgfqpoint{0.562872in}{3.377369in}}%
\pgfpathlineto{\pgfqpoint{0.529649in}{3.354550in}}%
\pgfpathlineto{\pgfqpoint{0.497312in}{3.328559in}}%
\pgfpathlineto{\pgfqpoint{0.479815in}{3.312784in}}%
\pgfpathlineto{\pgfqpoint{0.450289in}{3.283151in}}%
\pgfpathlineto{\pgfqpoint{0.431647in}{3.262306in}}%
\pgfpathlineto{\pgfqpoint{0.401839in}{3.225115in}}%
\pgfpathlineto{\pgfqpoint{0.379957in}{3.194412in}}%
\pgfpathlineto{\pgfqpoint{0.354454in}{3.154427in}}%
\pgfpathlineto{\pgfqpoint{0.335446in}{3.121243in}}%
\pgfpathlineto{\pgfqpoint{0.309305in}{3.070003in}}%
\pgfpathlineto{\pgfqpoint{0.286791in}{3.019574in}}%
\pgfpathlineto{\pgfqpoint{0.269461in}{2.975815in}}%
\pgfpathlineto{\pgfqpoint{0.251043in}{2.923362in}}%
\pgfpathlineto{\pgfqpoint{0.233933in}{2.867524in}}%
\pgfpathlineto{\pgfqpoint{0.220400in}{2.816851in}}%
\pgfpathlineto{\pgfqpoint{0.206867in}{2.758125in}}%
\pgfpathlineto{\pgfqpoint{0.194608in}{2.694586in}}%
\pgfpathlineto{\pgfqpoint{0.183460in}{2.622474in}}%
\pgfpathlineto{\pgfqpoint{0.176283in}{2.562480in}}%
\pgfpathlineto{\pgfqpoint{0.169869in}{2.487765in}}%
\pgfpathlineto{\pgfqpoint{0.166261in}{2.416377in}}%
\pgfpathlineto{\pgfqpoint{0.165059in}{2.344713in}}%
\pgfpathlineto{\pgfqpoint{0.165059in}{2.344713in}}%
\pgfpathlineto{\pgfqpoint{0.166767in}{2.259339in}}%
\pgfpathlineto{\pgfqpoint{0.170182in}{2.197111in}}%
\pgfpathlineto{\pgfqpoint{0.176158in}{2.128143in}}%
\pgfpathlineto{\pgfqpoint{0.183508in}{2.066594in}}%
\pgfpathlineto{\pgfqpoint{0.191490in}{2.013255in}}%
\pgfpathlineto{\pgfqpoint{0.204261in}{1.943829in}}%
\pgfpathlineto{\pgfqpoint{0.216945in}{1.886687in}}%
\pgfpathlineto{\pgfqpoint{0.234530in}{1.819815in}}%
\pgfpathlineto{\pgfqpoint{0.252944in}{1.760321in}}%
\pgfpathlineto{\pgfqpoint{0.272969in}{1.704357in}}%
\pgfpathlineto{\pgfqpoint{0.295065in}{1.650553in}}%
\pgfpathlineto{\pgfqpoint{0.318546in}{1.600494in}}%
\pgfpathlineto{\pgfqpoint{0.339371in}{1.561070in}}%
\pgfpathlineto{\pgfqpoint{0.366045in}{1.516224in}}%
\pgfpathlineto{\pgfqpoint{0.386531in}{1.485462in}}%
\pgfpathlineto{\pgfqpoint{0.416512in}{1.445376in}}%
\pgfpathlineto{\pgfqpoint{0.437942in}{1.419885in}}%
\pgfpathlineto{\pgfqpoint{0.464607in}{1.391411in}}%
\pgfpathlineto{\pgfqpoint{0.496463in}{1.361603in}}%
\pgfpathlineto{\pgfqpoint{0.521935in}{1.340722in}}%
\pgfpathlineto{\pgfqpoint{0.547559in}{1.322113in}}%
\pgfpathlineto{\pgfqpoint{0.569237in}{1.308099in}}%
\pgfpathlineto{\pgfqpoint{0.600554in}{1.290437in}}%
\pgfpathlineto{\pgfqpoint{0.635995in}{1.273824in}}%
\pgfpathlineto{\pgfqpoint{0.675111in}{1.259263in}}%
\pgfpathlineto{\pgfqpoint{0.711881in}{1.248855in}}%
\pgfpathlineto{\pgfqpoint{0.740954in}{1.242686in}}%
\pgfpathlineto{\pgfqpoint{0.769617in}{1.238254in}}%
\pgfpathlineto{\pgfqpoint{0.803773in}{1.234962in}}%
\pgfpathlineto{\pgfqpoint{0.845914in}{1.233645in}}%
\pgfpathlineto{\pgfqpoint{0.891893in}{1.235346in}}%
\pgfpathlineto{\pgfqpoint{0.933737in}{1.239452in}}%
\pgfpathlineto{\pgfqpoint{0.976310in}{1.245863in}}%
\pgfpathlineto{\pgfqpoint{1.030154in}{1.256804in}}%
\pgfpathlineto{\pgfqpoint{1.078582in}{1.268957in}}%
\pgfpathlineto{\pgfqpoint{1.147371in}{1.289231in}}%
\pgfpathlineto{\pgfqpoint{1.232854in}{1.317834in}}%
\pgfpathlineto{\pgfqpoint{1.424424in}{1.384036in}}%
\pgfpathlineto{\pgfqpoint{1.470305in}{1.397788in}}%
\pgfpathlineto{\pgfqpoint{1.529694in}{1.412876in}}%
\pgfpathlineto{\pgfqpoint{1.564867in}{1.419961in}}%
\pgfpathlineto{\pgfqpoint{1.613666in}{1.426994in}}%
\pgfpathlineto{\pgfqpoint{1.659418in}{1.430149in}}%
\pgfpathlineto{\pgfqpoint{1.679277in}{1.430359in}}%
\pgfpathlineto{\pgfqpoint{1.720263in}{1.428369in}}%
\pgfpathlineto{\pgfqpoint{1.754047in}{1.424106in}}%
\pgfpathlineto{\pgfqpoint{1.783402in}{1.418360in}}%
\pgfpathlineto{\pgfqpoint{1.816477in}{1.409507in}}%
\pgfpathlineto{\pgfqpoint{1.842277in}{1.400798in}}%
\pgfpathlineto{\pgfqpoint{1.875018in}{1.387415in}}%
\pgfpathlineto{\pgfqpoint{1.916433in}{1.366693in}}%
\pgfpathlineto{\pgfqpoint{1.939008in}{1.353603in}}%
\pgfpathlineto{\pgfqpoint{1.969603in}{1.333841in}}%
\pgfpathlineto{\pgfqpoint{2.003182in}{1.309497in}}%
\pgfpathlineto{\pgfqpoint{2.043086in}{1.277014in}}%
\pgfpathlineto{\pgfqpoint{2.088754in}{1.235229in}}%
\pgfpathlineto{\pgfqpoint{2.122225in}{1.201583in}}%
\pgfpathlineto{\pgfqpoint{2.167873in}{1.151751in}}%
\pgfpathlineto{\pgfqpoint{2.215938in}{1.094588in}}%
\pgfpathlineto{\pgfqpoint{2.259053in}{1.039443in}}%
\pgfpathlineto{\pgfqpoint{2.292597in}{0.994140in}}%
\pgfpathlineto{\pgfqpoint{2.347977in}{0.915011in}}%
\pgfpathlineto{\pgfqpoint{2.383078in}{0.862210in}}%
\pgfpathlineto{\pgfqpoint{2.448710in}{0.758340in}}%
\pgfpathlineto{\pgfqpoint{2.503269in}{0.667221in}}%
\pgfpathlineto{\pgfqpoint{2.569766in}{0.550744in}}%
\pgfpathlineto{\pgfqpoint{2.626799in}{0.446419in}}%
\pgfpathlineto{\pgfqpoint{2.693446in}{0.319681in}}%
\pgfpathlineto{\pgfqpoint{2.770473in}{0.167157in}}%
\pgfpathlineto{\pgfqpoint{2.808291in}{0.090000in}}%
\pgfpathlineto{\pgfqpoint{2.808291in}{0.090000in}}%
\pgfusepath{stroke}%
\end{pgfscope}%
\begin{pgfscope}%
\pgfsetrectcap%
\pgfsetmiterjoin%
\pgfsetlinewidth{0.803000pt}%
\definecolor{currentstroke}{rgb}{0.000000,0.000000,0.000000}%
\pgfsetstrokecolor{currentstroke}%
\pgfsetdash{}{0pt}%
\pgfpathmoveto{\pgfqpoint{1.259630in}{0.100000in}}%
\pgfpathlineto{\pgfqpoint{1.259630in}{4.589426in}}%
\pgfusepath{stroke}%
\end{pgfscope}%
\begin{pgfscope}%
\pgfsetrectcap%
\pgfsetmiterjoin%
\pgfsetlinewidth{0.803000pt}%
\definecolor{currentstroke}{rgb}{0.000000,0.000000,0.000000}%
\pgfsetstrokecolor{currentstroke}%
\pgfsetdash{}{0pt}%
\pgfpathmoveto{\pgfqpoint{0.100000in}{2.344713in}}%
\pgfpathlineto{\pgfqpoint{3.483066in}{2.344713in}}%
\pgfusepath{stroke}%
\end{pgfscope}%
\end{pgfpicture}%
\makeatother%
\endgroup%
}
      \end{figure}
      \pause
      \column{0.6\textwidth}
      \begin{figure}
        \scalebox{0.5}{%% Creator: Matplotlib, PGF backend
%%
%% To include the figure in your LaTeX document, write
%%   \input{<filename>.pgf}
%%
%% Make sure the required packages are loaded in your preamble
%%   \usepackage{pgf}
%%
%% Also ensure that all the required font packages are loaded; for instance,
%% the lmodern package is sometimes necessary when using math font.
%%   \usepackage{lmodern}
%%
%% Figures using additional raster images can only be included by \input if
%% they are in the same directory as the main LaTeX file. For loading figures
%% from other directories you can use the `import` package
%%   \usepackage{import}
%%
%% and then include the figures with
%%   \import{<path to file>}{<filename>.pgf}
%%
%% Matplotlib used the following preamble
%%   \usepackage{fontspec}
%%   \setmainfont{DejaVuSerif.ttf}[Path=\detokenize{/usr/lib/python3.10/site-packages/matplotlib/mpl-data/fonts/ttf/}]
%%   \setsansfont{DejaVuSans.ttf}[Path=\detokenize{/usr/lib/python3.10/site-packages/matplotlib/mpl-data/fonts/ttf/}]
%%   \setmonofont{DejaVuSansMono.ttf}[Path=\detokenize{/usr/lib/python3.10/site-packages/matplotlib/mpl-data/fonts/ttf/}]
%%
\begingroup%
\makeatletter%
\begin{pgfpicture}%
\pgfpathrectangle{\pgfpointorigin}{\pgfqpoint{4.838403in}{4.700000in}}%
\pgfusepath{use as bounding box, clip}%
\begin{pgfscope}%
\pgfsetbuttcap%
\pgfsetmiterjoin%
\definecolor{currentfill}{rgb}{1.000000,1.000000,1.000000}%
\pgfsetfillcolor{currentfill}%
\pgfsetlinewidth{0.000000pt}%
\definecolor{currentstroke}{rgb}{1.000000,1.000000,1.000000}%
\pgfsetstrokecolor{currentstroke}%
\pgfsetdash{}{0pt}%
\pgfpathmoveto{\pgfqpoint{0.000000in}{0.000000in}}%
\pgfpathlineto{\pgfqpoint{4.838403in}{0.000000in}}%
\pgfpathlineto{\pgfqpoint{4.838403in}{4.700000in}}%
\pgfpathlineto{\pgfqpoint{0.000000in}{4.700000in}}%
\pgfpathlineto{\pgfqpoint{0.000000in}{0.000000in}}%
\pgfpathclose%
\pgfusepath{fill}%
\end{pgfscope}%
\begin{pgfscope}%
\pgfsetbuttcap%
\pgfsetmiterjoin%
\definecolor{currentfill}{rgb}{1.000000,1.000000,1.000000}%
\pgfsetfillcolor{currentfill}%
\pgfsetlinewidth{0.000000pt}%
\definecolor{currentstroke}{rgb}{0.000000,0.000000,0.000000}%
\pgfsetstrokecolor{currentstroke}%
\pgfsetstrokeopacity{0.000000}%
\pgfsetdash{}{0pt}%
\pgfpathmoveto{\pgfqpoint{0.288431in}{0.470524in}}%
\pgfpathlineto{\pgfqpoint{4.735559in}{0.470524in}}%
\pgfpathlineto{\pgfqpoint{4.735559in}{4.600000in}}%
\pgfpathlineto{\pgfqpoint{0.288431in}{4.600000in}}%
\pgfpathlineto{\pgfqpoint{0.288431in}{0.470524in}}%
\pgfpathclose%
\pgfusepath{fill}%
\end{pgfscope}%
\begin{pgfscope}%
\pgfpathrectangle{\pgfqpoint{0.288431in}{0.470524in}}{\pgfqpoint{4.447128in}{4.129476in}}%
\pgfusepath{clip}%
\pgfsetbuttcap%
\pgfsetroundjoin%
\definecolor{currentfill}{rgb}{0.000000,0.000000,1.000000}%
\pgfsetfillcolor{currentfill}%
\pgfsetlinewidth{0.000000pt}%
\definecolor{currentstroke}{rgb}{0.000000,0.000000,0.000000}%
\pgfsetstrokecolor{currentstroke}%
\pgfsetdash{}{0pt}%
\pgfpathmoveto{\pgfqpoint{0.373953in}{1.077759in}}%
\pgfpathcurveto{\pgfqpoint{0.379777in}{1.077759in}}{\pgfqpoint{0.385363in}{1.080073in}}{\pgfqpoint{0.389481in}{1.084191in}}%
\pgfpathcurveto{\pgfqpoint{0.393599in}{1.088309in}}{\pgfqpoint{0.395913in}{1.093895in}}{\pgfqpoint{0.395913in}{1.099719in}}%
\pgfpathcurveto{\pgfqpoint{0.395913in}{1.105543in}}{\pgfqpoint{0.393599in}{1.111129in}}{\pgfqpoint{0.389481in}{1.115247in}}%
\pgfpathcurveto{\pgfqpoint{0.385363in}{1.119366in}}{\pgfqpoint{0.379777in}{1.121679in}}{\pgfqpoint{0.373953in}{1.121679in}}%
\pgfpathcurveto{\pgfqpoint{0.368129in}{1.121679in}}{\pgfqpoint{0.362543in}{1.119366in}}{\pgfqpoint{0.358425in}{1.115247in}}%
\pgfpathcurveto{\pgfqpoint{0.354307in}{1.111129in}}{\pgfqpoint{0.351993in}{1.105543in}}{\pgfqpoint{0.351993in}{1.099719in}}%
\pgfpathcurveto{\pgfqpoint{0.351993in}{1.093895in}}{\pgfqpoint{0.354307in}{1.088309in}}{\pgfqpoint{0.358425in}{1.084191in}}%
\pgfpathcurveto{\pgfqpoint{0.362543in}{1.080073in}}{\pgfqpoint{0.368129in}{1.077759in}}{\pgfqpoint{0.373953in}{1.077759in}}%
\pgfpathlineto{\pgfqpoint{0.373953in}{1.077759in}}%
\pgfpathclose%
\pgfusepath{fill}%
\end{pgfscope}%
\begin{pgfscope}%
\pgfpathrectangle{\pgfqpoint{0.288431in}{0.470524in}}{\pgfqpoint{4.447128in}{4.129476in}}%
\pgfusepath{clip}%
\pgfsetbuttcap%
\pgfsetroundjoin%
\definecolor{currentfill}{rgb}{0.000000,0.000000,1.000000}%
\pgfsetfillcolor{currentfill}%
\pgfsetlinewidth{0.000000pt}%
\definecolor{currentstroke}{rgb}{0.000000,0.000000,0.000000}%
\pgfsetstrokecolor{currentstroke}%
\pgfsetdash{}{0pt}%
\pgfpathmoveto{\pgfqpoint{0.373953in}{3.948844in}}%
\pgfpathcurveto{\pgfqpoint{0.379777in}{3.948844in}}{\pgfqpoint{0.385363in}{3.951158in}}{\pgfqpoint{0.389481in}{3.955276in}}%
\pgfpathcurveto{\pgfqpoint{0.393599in}{3.959395in}}{\pgfqpoint{0.395913in}{3.964981in}}{\pgfqpoint{0.395913in}{3.970805in}}%
\pgfpathcurveto{\pgfqpoint{0.395913in}{3.976629in}}{\pgfqpoint{0.393599in}{3.982215in}}{\pgfqpoint{0.389481in}{3.986333in}}%
\pgfpathcurveto{\pgfqpoint{0.385363in}{3.990451in}}{\pgfqpoint{0.379777in}{3.992765in}}{\pgfqpoint{0.373953in}{3.992765in}}%
\pgfpathcurveto{\pgfqpoint{0.368129in}{3.992765in}}{\pgfqpoint{0.362543in}{3.990451in}}{\pgfqpoint{0.358425in}{3.986333in}}%
\pgfpathcurveto{\pgfqpoint{0.354307in}{3.982215in}}{\pgfqpoint{0.351993in}{3.976629in}}{\pgfqpoint{0.351993in}{3.970805in}}%
\pgfpathcurveto{\pgfqpoint{0.351993in}{3.964981in}}{\pgfqpoint{0.354307in}{3.959395in}}{\pgfqpoint{0.358425in}{3.955276in}}%
\pgfpathcurveto{\pgfqpoint{0.362543in}{3.951158in}}{\pgfqpoint{0.368129in}{3.948844in}}{\pgfqpoint{0.373953in}{3.948844in}}%
\pgfpathlineto{\pgfqpoint{0.373953in}{3.948844in}}%
\pgfpathclose%
\pgfusepath{fill}%
\end{pgfscope}%
\begin{pgfscope}%
\pgfpathrectangle{\pgfqpoint{0.288431in}{0.470524in}}{\pgfqpoint{4.447128in}{4.129476in}}%
\pgfusepath{clip}%
\pgfsetbuttcap%
\pgfsetroundjoin%
\definecolor{currentfill}{rgb}{0.000000,0.000000,1.000000}%
\pgfsetfillcolor{currentfill}%
\pgfsetlinewidth{0.000000pt}%
\definecolor{currentstroke}{rgb}{0.000000,0.000000,0.000000}%
\pgfsetstrokecolor{currentstroke}%
\pgfsetdash{}{0pt}%
\pgfpathmoveto{\pgfqpoint{0.435040in}{1.077759in}}%
\pgfpathcurveto{\pgfqpoint{0.440864in}{1.077759in}}{\pgfqpoint{0.446450in}{1.080073in}}{\pgfqpoint{0.450568in}{1.084191in}}%
\pgfpathcurveto{\pgfqpoint{0.454686in}{1.088309in}}{\pgfqpoint{0.457000in}{1.093895in}}{\pgfqpoint{0.457000in}{1.099719in}}%
\pgfpathcurveto{\pgfqpoint{0.457000in}{1.105543in}}{\pgfqpoint{0.454686in}{1.111129in}}{\pgfqpoint{0.450568in}{1.115247in}}%
\pgfpathcurveto{\pgfqpoint{0.446450in}{1.119366in}}{\pgfqpoint{0.440864in}{1.121679in}}{\pgfqpoint{0.435040in}{1.121679in}}%
\pgfpathcurveto{\pgfqpoint{0.429216in}{1.121679in}}{\pgfqpoint{0.423630in}{1.119366in}}{\pgfqpoint{0.419512in}{1.115247in}}%
\pgfpathcurveto{\pgfqpoint{0.415393in}{1.111129in}}{\pgfqpoint{0.413080in}{1.105543in}}{\pgfqpoint{0.413080in}{1.099719in}}%
\pgfpathcurveto{\pgfqpoint{0.413080in}{1.093895in}}{\pgfqpoint{0.415393in}{1.088309in}}{\pgfqpoint{0.419512in}{1.084191in}}%
\pgfpathcurveto{\pgfqpoint{0.423630in}{1.080073in}}{\pgfqpoint{0.429216in}{1.077759in}}{\pgfqpoint{0.435040in}{1.077759in}}%
\pgfpathlineto{\pgfqpoint{0.435040in}{1.077759in}}%
\pgfpathclose%
\pgfusepath{fill}%
\end{pgfscope}%
\begin{pgfscope}%
\pgfpathrectangle{\pgfqpoint{0.288431in}{0.470524in}}{\pgfqpoint{4.447128in}{4.129476in}}%
\pgfusepath{clip}%
\pgfsetbuttcap%
\pgfsetroundjoin%
\definecolor{currentfill}{rgb}{0.000000,0.000000,1.000000}%
\pgfsetfillcolor{currentfill}%
\pgfsetlinewidth{0.000000pt}%
\definecolor{currentstroke}{rgb}{0.000000,0.000000,0.000000}%
\pgfsetstrokecolor{currentstroke}%
\pgfsetdash{}{0pt}%
\pgfpathmoveto{\pgfqpoint{0.435040in}{3.948844in}}%
\pgfpathcurveto{\pgfqpoint{0.440864in}{3.948844in}}{\pgfqpoint{0.446450in}{3.951158in}}{\pgfqpoint{0.450568in}{3.955276in}}%
\pgfpathcurveto{\pgfqpoint{0.454686in}{3.959395in}}{\pgfqpoint{0.457000in}{3.964981in}}{\pgfqpoint{0.457000in}{3.970805in}}%
\pgfpathcurveto{\pgfqpoint{0.457000in}{3.976629in}}{\pgfqpoint{0.454686in}{3.982215in}}{\pgfqpoint{0.450568in}{3.986333in}}%
\pgfpathcurveto{\pgfqpoint{0.446450in}{3.990451in}}{\pgfqpoint{0.440864in}{3.992765in}}{\pgfqpoint{0.435040in}{3.992765in}}%
\pgfpathcurveto{\pgfqpoint{0.429216in}{3.992765in}}{\pgfqpoint{0.423630in}{3.990451in}}{\pgfqpoint{0.419512in}{3.986333in}}%
\pgfpathcurveto{\pgfqpoint{0.415393in}{3.982215in}}{\pgfqpoint{0.413080in}{3.976629in}}{\pgfqpoint{0.413080in}{3.970805in}}%
\pgfpathcurveto{\pgfqpoint{0.413080in}{3.964981in}}{\pgfqpoint{0.415393in}{3.959395in}}{\pgfqpoint{0.419512in}{3.955276in}}%
\pgfpathcurveto{\pgfqpoint{0.423630in}{3.951158in}}{\pgfqpoint{0.429216in}{3.948844in}}{\pgfqpoint{0.435040in}{3.948844in}}%
\pgfpathlineto{\pgfqpoint{0.435040in}{3.948844in}}%
\pgfpathclose%
\pgfusepath{fill}%
\end{pgfscope}%
\begin{pgfscope}%
\pgfpathrectangle{\pgfqpoint{0.288431in}{0.470524in}}{\pgfqpoint{4.447128in}{4.129476in}}%
\pgfusepath{clip}%
\pgfsetbuttcap%
\pgfsetroundjoin%
\definecolor{currentfill}{rgb}{0.000000,0.000000,1.000000}%
\pgfsetfillcolor{currentfill}%
\pgfsetlinewidth{0.000000pt}%
\definecolor{currentstroke}{rgb}{0.000000,0.000000,0.000000}%
\pgfsetstrokecolor{currentstroke}%
\pgfsetdash{}{0pt}%
\pgfpathmoveto{\pgfqpoint{0.496127in}{1.810802in}}%
\pgfpathcurveto{\pgfqpoint{0.501951in}{1.810802in}}{\pgfqpoint{0.507537in}{1.813116in}}{\pgfqpoint{0.511655in}{1.817234in}}%
\pgfpathcurveto{\pgfqpoint{0.515773in}{1.821352in}}{\pgfqpoint{0.518087in}{1.826938in}}{\pgfqpoint{0.518087in}{1.832762in}}%
\pgfpathcurveto{\pgfqpoint{0.518087in}{1.838586in}}{\pgfqpoint{0.515773in}{1.844172in}}{\pgfqpoint{0.511655in}{1.848291in}}%
\pgfpathcurveto{\pgfqpoint{0.507537in}{1.852409in}}{\pgfqpoint{0.501951in}{1.854723in}}{\pgfqpoint{0.496127in}{1.854723in}}%
\pgfpathcurveto{\pgfqpoint{0.490303in}{1.854723in}}{\pgfqpoint{0.484717in}{1.852409in}}{\pgfqpoint{0.480599in}{1.848291in}}%
\pgfpathcurveto{\pgfqpoint{0.476480in}{1.844172in}}{\pgfqpoint{0.474167in}{1.838586in}}{\pgfqpoint{0.474167in}{1.832762in}}%
\pgfpathcurveto{\pgfqpoint{0.474167in}{1.826938in}}{\pgfqpoint{0.476480in}{1.821352in}}{\pgfqpoint{0.480599in}{1.817234in}}%
\pgfpathcurveto{\pgfqpoint{0.484717in}{1.813116in}}{\pgfqpoint{0.490303in}{1.810802in}}{\pgfqpoint{0.496127in}{1.810802in}}%
\pgfpathlineto{\pgfqpoint{0.496127in}{1.810802in}}%
\pgfpathclose%
\pgfusepath{fill}%
\end{pgfscope}%
\begin{pgfscope}%
\pgfpathrectangle{\pgfqpoint{0.288431in}{0.470524in}}{\pgfqpoint{4.447128in}{4.129476in}}%
\pgfusepath{clip}%
\pgfsetbuttcap%
\pgfsetroundjoin%
\definecolor{currentfill}{rgb}{0.000000,0.000000,1.000000}%
\pgfsetfillcolor{currentfill}%
\pgfsetlinewidth{0.000000pt}%
\definecolor{currentstroke}{rgb}{0.000000,0.000000,0.000000}%
\pgfsetstrokecolor{currentstroke}%
\pgfsetdash{}{0pt}%
\pgfpathmoveto{\pgfqpoint{0.496127in}{3.215801in}}%
\pgfpathcurveto{\pgfqpoint{0.501951in}{3.215801in}}{\pgfqpoint{0.507537in}{3.218115in}}{\pgfqpoint{0.511655in}{3.222233in}}%
\pgfpathcurveto{\pgfqpoint{0.515773in}{3.226351in}}{\pgfqpoint{0.518087in}{3.231938in}}{\pgfqpoint{0.518087in}{3.237762in}}%
\pgfpathcurveto{\pgfqpoint{0.518087in}{3.243586in}}{\pgfqpoint{0.515773in}{3.249172in}}{\pgfqpoint{0.511655in}{3.253290in}}%
\pgfpathcurveto{\pgfqpoint{0.507537in}{3.257408in}}{\pgfqpoint{0.501951in}{3.259722in}}{\pgfqpoint{0.496127in}{3.259722in}}%
\pgfpathcurveto{\pgfqpoint{0.490303in}{3.259722in}}{\pgfqpoint{0.484717in}{3.257408in}}{\pgfqpoint{0.480599in}{3.253290in}}%
\pgfpathcurveto{\pgfqpoint{0.476480in}{3.249172in}}{\pgfqpoint{0.474167in}{3.243586in}}{\pgfqpoint{0.474167in}{3.237762in}}%
\pgfpathcurveto{\pgfqpoint{0.474167in}{3.231938in}}{\pgfqpoint{0.476480in}{3.226351in}}{\pgfqpoint{0.480599in}{3.222233in}}%
\pgfpathcurveto{\pgfqpoint{0.484717in}{3.218115in}}{\pgfqpoint{0.490303in}{3.215801in}}{\pgfqpoint{0.496127in}{3.215801in}}%
\pgfpathlineto{\pgfqpoint{0.496127in}{3.215801in}}%
\pgfpathclose%
\pgfusepath{fill}%
\end{pgfscope}%
\begin{pgfscope}%
\pgfpathrectangle{\pgfqpoint{0.288431in}{0.470524in}}{\pgfqpoint{4.447128in}{4.129476in}}%
\pgfusepath{clip}%
\pgfsetbuttcap%
\pgfsetroundjoin%
\definecolor{currentfill}{rgb}{0.000000,0.000000,1.000000}%
\pgfsetfillcolor{currentfill}%
\pgfsetlinewidth{0.000000pt}%
\definecolor{currentstroke}{rgb}{0.000000,0.000000,0.000000}%
\pgfsetstrokecolor{currentstroke}%
\pgfsetdash{}{0pt}%
\pgfpathmoveto{\pgfqpoint{0.801561in}{1.199933in}}%
\pgfpathcurveto{\pgfqpoint{0.807385in}{1.199933in}}{\pgfqpoint{0.812971in}{1.202247in}}{\pgfqpoint{0.817090in}{1.206365in}}%
\pgfpathcurveto{\pgfqpoint{0.821208in}{1.210483in}}{\pgfqpoint{0.823522in}{1.216069in}}{\pgfqpoint{0.823522in}{1.221893in}}%
\pgfpathcurveto{\pgfqpoint{0.823522in}{1.227717in}}{\pgfqpoint{0.821208in}{1.233303in}}{\pgfqpoint{0.817090in}{1.237421in}}%
\pgfpathcurveto{\pgfqpoint{0.812971in}{1.241539in}}{\pgfqpoint{0.807385in}{1.243853in}}{\pgfqpoint{0.801561in}{1.243853in}}%
\pgfpathcurveto{\pgfqpoint{0.795737in}{1.243853in}}{\pgfqpoint{0.790151in}{1.241539in}}{\pgfqpoint{0.786033in}{1.237421in}}%
\pgfpathcurveto{\pgfqpoint{0.781915in}{1.233303in}}{\pgfqpoint{0.779601in}{1.227717in}}{\pgfqpoint{0.779601in}{1.221893in}}%
\pgfpathcurveto{\pgfqpoint{0.779601in}{1.216069in}}{\pgfqpoint{0.781915in}{1.210483in}}{\pgfqpoint{0.786033in}{1.206365in}}%
\pgfpathcurveto{\pgfqpoint{0.790151in}{1.202247in}}{\pgfqpoint{0.795737in}{1.199933in}}{\pgfqpoint{0.801561in}{1.199933in}}%
\pgfpathlineto{\pgfqpoint{0.801561in}{1.199933in}}%
\pgfpathclose%
\pgfusepath{fill}%
\end{pgfscope}%
\begin{pgfscope}%
\pgfpathrectangle{\pgfqpoint{0.288431in}{0.470524in}}{\pgfqpoint{4.447128in}{4.129476in}}%
\pgfusepath{clip}%
\pgfsetbuttcap%
\pgfsetroundjoin%
\definecolor{currentfill}{rgb}{0.000000,0.000000,1.000000}%
\pgfsetfillcolor{currentfill}%
\pgfsetlinewidth{0.000000pt}%
\definecolor{currentstroke}{rgb}{0.000000,0.000000,0.000000}%
\pgfsetstrokecolor{currentstroke}%
\pgfsetdash{}{0pt}%
\pgfpathmoveto{\pgfqpoint{0.801561in}{3.826671in}}%
\pgfpathcurveto{\pgfqpoint{0.807385in}{3.826671in}}{\pgfqpoint{0.812971in}{3.828984in}}{\pgfqpoint{0.817090in}{3.833103in}}%
\pgfpathcurveto{\pgfqpoint{0.821208in}{3.837221in}}{\pgfqpoint{0.823522in}{3.842807in}}{\pgfqpoint{0.823522in}{3.848631in}}%
\pgfpathcurveto{\pgfqpoint{0.823522in}{3.854455in}}{\pgfqpoint{0.821208in}{3.860041in}}{\pgfqpoint{0.817090in}{3.864159in}}%
\pgfpathcurveto{\pgfqpoint{0.812971in}{3.868277in}}{\pgfqpoint{0.807385in}{3.870591in}}{\pgfqpoint{0.801561in}{3.870591in}}%
\pgfpathcurveto{\pgfqpoint{0.795737in}{3.870591in}}{\pgfqpoint{0.790151in}{3.868277in}}{\pgfqpoint{0.786033in}{3.864159in}}%
\pgfpathcurveto{\pgfqpoint{0.781915in}{3.860041in}}{\pgfqpoint{0.779601in}{3.854455in}}{\pgfqpoint{0.779601in}{3.848631in}}%
\pgfpathcurveto{\pgfqpoint{0.779601in}{3.842807in}}{\pgfqpoint{0.781915in}{3.837221in}}{\pgfqpoint{0.786033in}{3.833103in}}%
\pgfpathcurveto{\pgfqpoint{0.790151in}{3.828984in}}{\pgfqpoint{0.795737in}{3.826671in}}{\pgfqpoint{0.801561in}{3.826671in}}%
\pgfpathlineto{\pgfqpoint{0.801561in}{3.826671in}}%
\pgfpathclose%
\pgfusepath{fill}%
\end{pgfscope}%
\begin{pgfscope}%
\pgfpathrectangle{\pgfqpoint{0.288431in}{0.470524in}}{\pgfqpoint{4.447128in}{4.129476in}}%
\pgfusepath{clip}%
\pgfsetbuttcap%
\pgfsetroundjoin%
\definecolor{currentfill}{rgb}{0.000000,0.000000,1.000000}%
\pgfsetfillcolor{currentfill}%
\pgfsetlinewidth{0.000000pt}%
\definecolor{currentstroke}{rgb}{0.000000,0.000000,0.000000}%
\pgfsetstrokecolor{currentstroke}%
\pgfsetdash{}{0pt}%
\pgfpathmoveto{\pgfqpoint{0.862648in}{0.527977in}}%
\pgfpathcurveto{\pgfqpoint{0.868472in}{0.527977in}}{\pgfqpoint{0.874058in}{0.530291in}}{\pgfqpoint{0.878177in}{0.534409in}}%
\pgfpathcurveto{\pgfqpoint{0.882295in}{0.538527in}}{\pgfqpoint{0.884609in}{0.544113in}}{\pgfqpoint{0.884609in}{0.549937in}}%
\pgfpathcurveto{\pgfqpoint{0.884609in}{0.555761in}}{\pgfqpoint{0.882295in}{0.561347in}}{\pgfqpoint{0.878177in}{0.565465in}}%
\pgfpathcurveto{\pgfqpoint{0.874058in}{0.569583in}}{\pgfqpoint{0.868472in}{0.571897in}}{\pgfqpoint{0.862648in}{0.571897in}}%
\pgfpathcurveto{\pgfqpoint{0.856824in}{0.571897in}}{\pgfqpoint{0.851238in}{0.569583in}}{\pgfqpoint{0.847120in}{0.565465in}}%
\pgfpathcurveto{\pgfqpoint{0.843002in}{0.561347in}}{\pgfqpoint{0.840688in}{0.555761in}}{\pgfqpoint{0.840688in}{0.549937in}}%
\pgfpathcurveto{\pgfqpoint{0.840688in}{0.544113in}}{\pgfqpoint{0.843002in}{0.538527in}}{\pgfqpoint{0.847120in}{0.534409in}}%
\pgfpathcurveto{\pgfqpoint{0.851238in}{0.530291in}}{\pgfqpoint{0.856824in}{0.527977in}}{\pgfqpoint{0.862648in}{0.527977in}}%
\pgfpathlineto{\pgfqpoint{0.862648in}{0.527977in}}%
\pgfpathclose%
\pgfusepath{fill}%
\end{pgfscope}%
\begin{pgfscope}%
\pgfpathrectangle{\pgfqpoint{0.288431in}{0.470524in}}{\pgfqpoint{4.447128in}{4.129476in}}%
\pgfusepath{clip}%
\pgfsetbuttcap%
\pgfsetroundjoin%
\definecolor{currentfill}{rgb}{0.000000,0.000000,1.000000}%
\pgfsetfillcolor{currentfill}%
\pgfsetlinewidth{0.000000pt}%
\definecolor{currentstroke}{rgb}{0.000000,0.000000,0.000000}%
\pgfsetstrokecolor{currentstroke}%
\pgfsetdash{}{0pt}%
\pgfpathmoveto{\pgfqpoint{0.862648in}{4.498627in}}%
\pgfpathcurveto{\pgfqpoint{0.868472in}{4.498627in}}{\pgfqpoint{0.874058in}{4.500941in}}{\pgfqpoint{0.878177in}{4.505059in}}%
\pgfpathcurveto{\pgfqpoint{0.882295in}{4.509177in}}{\pgfqpoint{0.884609in}{4.514763in}}{\pgfqpoint{0.884609in}{4.520587in}}%
\pgfpathcurveto{\pgfqpoint{0.884609in}{4.526411in}}{\pgfqpoint{0.882295in}{4.531997in}}{\pgfqpoint{0.878177in}{4.536115in}}%
\pgfpathcurveto{\pgfqpoint{0.874058in}{4.540233in}}{\pgfqpoint{0.868472in}{4.542547in}}{\pgfqpoint{0.862648in}{4.542547in}}%
\pgfpathcurveto{\pgfqpoint{0.856824in}{4.542547in}}{\pgfqpoint{0.851238in}{4.540233in}}{\pgfqpoint{0.847120in}{4.536115in}}%
\pgfpathcurveto{\pgfqpoint{0.843002in}{4.531997in}}{\pgfqpoint{0.840688in}{4.526411in}}{\pgfqpoint{0.840688in}{4.520587in}}%
\pgfpathcurveto{\pgfqpoint{0.840688in}{4.514763in}}{\pgfqpoint{0.843002in}{4.509177in}}{\pgfqpoint{0.847120in}{4.505059in}}%
\pgfpathcurveto{\pgfqpoint{0.851238in}{4.500941in}}{\pgfqpoint{0.856824in}{4.498627in}}{\pgfqpoint{0.862648in}{4.498627in}}%
\pgfpathlineto{\pgfqpoint{0.862648in}{4.498627in}}%
\pgfpathclose%
\pgfusepath{fill}%
\end{pgfscope}%
\begin{pgfscope}%
\pgfpathrectangle{\pgfqpoint{0.288431in}{0.470524in}}{\pgfqpoint{4.447128in}{4.129476in}}%
\pgfusepath{clip}%
\pgfsetbuttcap%
\pgfsetroundjoin%
\definecolor{currentfill}{rgb}{0.000000,0.000000,1.000000}%
\pgfsetfillcolor{currentfill}%
\pgfsetlinewidth{0.000000pt}%
\definecolor{currentstroke}{rgb}{0.000000,0.000000,0.000000}%
\pgfsetstrokecolor{currentstroke}%
\pgfsetdash{}{0pt}%
\pgfpathmoveto{\pgfqpoint{0.923735in}{1.261020in}}%
\pgfpathcurveto{\pgfqpoint{0.929559in}{1.261020in}}{\pgfqpoint{0.935145in}{1.263334in}}{\pgfqpoint{0.939263in}{1.267452in}}%
\pgfpathcurveto{\pgfqpoint{0.943382in}{1.271570in}}{\pgfqpoint{0.945696in}{1.277156in}}{\pgfqpoint{0.945696in}{1.282980in}}%
\pgfpathcurveto{\pgfqpoint{0.945696in}{1.288804in}}{\pgfqpoint{0.943382in}{1.294390in}}{\pgfqpoint{0.939263in}{1.298508in}}%
\pgfpathcurveto{\pgfqpoint{0.935145in}{1.302626in}}{\pgfqpoint{0.929559in}{1.304940in}}{\pgfqpoint{0.923735in}{1.304940in}}%
\pgfpathcurveto{\pgfqpoint{0.917911in}{1.304940in}}{\pgfqpoint{0.912325in}{1.302626in}}{\pgfqpoint{0.908207in}{1.298508in}}%
\pgfpathcurveto{\pgfqpoint{0.904089in}{1.294390in}}{\pgfqpoint{0.901775in}{1.288804in}}{\pgfqpoint{0.901775in}{1.282980in}}%
\pgfpathcurveto{\pgfqpoint{0.901775in}{1.277156in}}{\pgfqpoint{0.904089in}{1.271570in}}{\pgfqpoint{0.908207in}{1.267452in}}%
\pgfpathcurveto{\pgfqpoint{0.912325in}{1.263334in}}{\pgfqpoint{0.917911in}{1.261020in}}{\pgfqpoint{0.923735in}{1.261020in}}%
\pgfpathlineto{\pgfqpoint{0.923735in}{1.261020in}}%
\pgfpathclose%
\pgfusepath{fill}%
\end{pgfscope}%
\begin{pgfscope}%
\pgfpathrectangle{\pgfqpoint{0.288431in}{0.470524in}}{\pgfqpoint{4.447128in}{4.129476in}}%
\pgfusepath{clip}%
\pgfsetbuttcap%
\pgfsetroundjoin%
\definecolor{currentfill}{rgb}{0.000000,0.000000,1.000000}%
\pgfsetfillcolor{currentfill}%
\pgfsetlinewidth{0.000000pt}%
\definecolor{currentstroke}{rgb}{0.000000,0.000000,0.000000}%
\pgfsetstrokecolor{currentstroke}%
\pgfsetdash{}{0pt}%
\pgfpathmoveto{\pgfqpoint{0.923735in}{3.765584in}}%
\pgfpathcurveto{\pgfqpoint{0.929559in}{3.765584in}}{\pgfqpoint{0.935145in}{3.767898in}}{\pgfqpoint{0.939263in}{3.772016in}}%
\pgfpathcurveto{\pgfqpoint{0.943382in}{3.776134in}}{\pgfqpoint{0.945696in}{3.781720in}}{\pgfqpoint{0.945696in}{3.787544in}}%
\pgfpathcurveto{\pgfqpoint{0.945696in}{3.793368in}}{\pgfqpoint{0.943382in}{3.798954in}}{\pgfqpoint{0.939263in}{3.803072in}}%
\pgfpathcurveto{\pgfqpoint{0.935145in}{3.807190in}}{\pgfqpoint{0.929559in}{3.809504in}}{\pgfqpoint{0.923735in}{3.809504in}}%
\pgfpathcurveto{\pgfqpoint{0.917911in}{3.809504in}}{\pgfqpoint{0.912325in}{3.807190in}}{\pgfqpoint{0.908207in}{3.803072in}}%
\pgfpathcurveto{\pgfqpoint{0.904089in}{3.798954in}}{\pgfqpoint{0.901775in}{3.793368in}}{\pgfqpoint{0.901775in}{3.787544in}}%
\pgfpathcurveto{\pgfqpoint{0.901775in}{3.781720in}}{\pgfqpoint{0.904089in}{3.776134in}}{\pgfqpoint{0.908207in}{3.772016in}}%
\pgfpathcurveto{\pgfqpoint{0.912325in}{3.767898in}}{\pgfqpoint{0.917911in}{3.765584in}}{\pgfqpoint{0.923735in}{3.765584in}}%
\pgfpathlineto{\pgfqpoint{0.923735in}{3.765584in}}%
\pgfpathclose%
\pgfusepath{fill}%
\end{pgfscope}%
\begin{pgfscope}%
\pgfpathrectangle{\pgfqpoint{0.288431in}{0.470524in}}{\pgfqpoint{4.447128in}{4.129476in}}%
\pgfusepath{clip}%
\pgfsetbuttcap%
\pgfsetroundjoin%
\definecolor{currentfill}{rgb}{0.000000,0.000000,1.000000}%
\pgfsetfillcolor{currentfill}%
\pgfsetlinewidth{0.000000pt}%
\definecolor{currentstroke}{rgb}{0.000000,0.000000,0.000000}%
\pgfsetstrokecolor{currentstroke}%
\pgfsetdash{}{0pt}%
\pgfpathmoveto{\pgfqpoint{1.290257in}{0.650150in}}%
\pgfpathcurveto{\pgfqpoint{1.296081in}{0.650150in}}{\pgfqpoint{1.301667in}{0.652464in}}{\pgfqpoint{1.305785in}{0.656582in}}%
\pgfpathcurveto{\pgfqpoint{1.309903in}{0.660701in}}{\pgfqpoint{1.312217in}{0.666287in}}{\pgfqpoint{1.312217in}{0.672111in}}%
\pgfpathcurveto{\pgfqpoint{1.312217in}{0.677935in}}{\pgfqpoint{1.309903in}{0.683521in}}{\pgfqpoint{1.305785in}{0.687639in}}%
\pgfpathcurveto{\pgfqpoint{1.301667in}{0.691757in}}{\pgfqpoint{1.296081in}{0.694071in}}{\pgfqpoint{1.290257in}{0.694071in}}%
\pgfpathcurveto{\pgfqpoint{1.284433in}{0.694071in}}{\pgfqpoint{1.278847in}{0.691757in}}{\pgfqpoint{1.274729in}{0.687639in}}%
\pgfpathcurveto{\pgfqpoint{1.270610in}{0.683521in}}{\pgfqpoint{1.268297in}{0.677935in}}{\pgfqpoint{1.268297in}{0.672111in}}%
\pgfpathcurveto{\pgfqpoint{1.268297in}{0.666287in}}{\pgfqpoint{1.270610in}{0.660701in}}{\pgfqpoint{1.274729in}{0.656582in}}%
\pgfpathcurveto{\pgfqpoint{1.278847in}{0.652464in}}{\pgfqpoint{1.284433in}{0.650150in}}{\pgfqpoint{1.290257in}{0.650150in}}%
\pgfpathlineto{\pgfqpoint{1.290257in}{0.650150in}}%
\pgfpathclose%
\pgfusepath{fill}%
\end{pgfscope}%
\begin{pgfscope}%
\pgfpathrectangle{\pgfqpoint{0.288431in}{0.470524in}}{\pgfqpoint{4.447128in}{4.129476in}}%
\pgfusepath{clip}%
\pgfsetbuttcap%
\pgfsetroundjoin%
\definecolor{currentfill}{rgb}{0.000000,0.000000,1.000000}%
\pgfsetfillcolor{currentfill}%
\pgfsetlinewidth{0.000000pt}%
\definecolor{currentstroke}{rgb}{0.000000,0.000000,0.000000}%
\pgfsetstrokecolor{currentstroke}%
\pgfsetdash{}{0pt}%
\pgfpathmoveto{\pgfqpoint{1.290257in}{4.376453in}}%
\pgfpathcurveto{\pgfqpoint{1.296081in}{4.376453in}}{\pgfqpoint{1.301667in}{4.378767in}}{\pgfqpoint{1.305785in}{4.382885in}}%
\pgfpathcurveto{\pgfqpoint{1.309903in}{4.387003in}}{\pgfqpoint{1.312217in}{4.392589in}}{\pgfqpoint{1.312217in}{4.398413in}}%
\pgfpathcurveto{\pgfqpoint{1.312217in}{4.404237in}}{\pgfqpoint{1.309903in}{4.409823in}}{\pgfqpoint{1.305785in}{4.413941in}}%
\pgfpathcurveto{\pgfqpoint{1.301667in}{4.418060in}}{\pgfqpoint{1.296081in}{4.420373in}}{\pgfqpoint{1.290257in}{4.420373in}}%
\pgfpathcurveto{\pgfqpoint{1.284433in}{4.420373in}}{\pgfqpoint{1.278847in}{4.418060in}}{\pgfqpoint{1.274729in}{4.413941in}}%
\pgfpathcurveto{\pgfqpoint{1.270610in}{4.409823in}}{\pgfqpoint{1.268297in}{4.404237in}}{\pgfqpoint{1.268297in}{4.398413in}}%
\pgfpathcurveto{\pgfqpoint{1.268297in}{4.392589in}}{\pgfqpoint{1.270610in}{4.387003in}}{\pgfqpoint{1.274729in}{4.382885in}}%
\pgfpathcurveto{\pgfqpoint{1.278847in}{4.378767in}}{\pgfqpoint{1.284433in}{4.376453in}}{\pgfqpoint{1.290257in}{4.376453in}}%
\pgfpathlineto{\pgfqpoint{1.290257in}{4.376453in}}%
\pgfpathclose%
\pgfusepath{fill}%
\end{pgfscope}%
\begin{pgfscope}%
\pgfpathrectangle{\pgfqpoint{0.288431in}{0.470524in}}{\pgfqpoint{4.447128in}{4.129476in}}%
\pgfusepath{clip}%
\pgfsetbuttcap%
\pgfsetroundjoin%
\definecolor{currentfill}{rgb}{0.000000,0.000000,1.000000}%
\pgfsetfillcolor{currentfill}%
\pgfsetlinewidth{0.000000pt}%
\definecolor{currentstroke}{rgb}{0.000000,0.000000,0.000000}%
\pgfsetstrokecolor{currentstroke}%
\pgfsetdash{}{0pt}%
\pgfpathmoveto{\pgfqpoint{1.595691in}{2.299497in}}%
\pgfpathcurveto{\pgfqpoint{1.601515in}{2.299497in}}{\pgfqpoint{1.607102in}{2.301811in}}{\pgfqpoint{1.611220in}{2.305929in}}%
\pgfpathcurveto{\pgfqpoint{1.615338in}{2.310048in}}{\pgfqpoint{1.617652in}{2.315634in}}{\pgfqpoint{1.617652in}{2.321458in}}%
\pgfpathcurveto{\pgfqpoint{1.617652in}{2.327282in}}{\pgfqpoint{1.615338in}{2.332868in}}{\pgfqpoint{1.611220in}{2.336986in}}%
\pgfpathcurveto{\pgfqpoint{1.607102in}{2.341104in}}{\pgfqpoint{1.601515in}{2.343418in}}{\pgfqpoint{1.595691in}{2.343418in}}%
\pgfpathcurveto{\pgfqpoint{1.589867in}{2.343418in}}{\pgfqpoint{1.584281in}{2.341104in}}{\pgfqpoint{1.580163in}{2.336986in}}%
\pgfpathcurveto{\pgfqpoint{1.576045in}{2.332868in}}{\pgfqpoint{1.573731in}{2.327282in}}{\pgfqpoint{1.573731in}{2.321458in}}%
\pgfpathcurveto{\pgfqpoint{1.573731in}{2.315634in}}{\pgfqpoint{1.576045in}{2.310048in}}{\pgfqpoint{1.580163in}{2.305929in}}%
\pgfpathcurveto{\pgfqpoint{1.584281in}{2.301811in}}{\pgfqpoint{1.589867in}{2.299497in}}{\pgfqpoint{1.595691in}{2.299497in}}%
\pgfpathlineto{\pgfqpoint{1.595691in}{2.299497in}}%
\pgfpathclose%
\pgfusepath{fill}%
\end{pgfscope}%
\begin{pgfscope}%
\pgfpathrectangle{\pgfqpoint{0.288431in}{0.470524in}}{\pgfqpoint{4.447128in}{4.129476in}}%
\pgfusepath{clip}%
\pgfsetbuttcap%
\pgfsetroundjoin%
\definecolor{currentfill}{rgb}{0.000000,0.000000,1.000000}%
\pgfsetfillcolor{currentfill}%
\pgfsetlinewidth{0.000000pt}%
\definecolor{currentstroke}{rgb}{0.000000,0.000000,0.000000}%
\pgfsetstrokecolor{currentstroke}%
\pgfsetdash{}{0pt}%
\pgfpathmoveto{\pgfqpoint{1.595691in}{2.727106in}}%
\pgfpathcurveto{\pgfqpoint{1.601515in}{2.727106in}}{\pgfqpoint{1.607102in}{2.729420in}}{\pgfqpoint{1.611220in}{2.733538in}}%
\pgfpathcurveto{\pgfqpoint{1.615338in}{2.737656in}}{\pgfqpoint{1.617652in}{2.743242in}}{\pgfqpoint{1.617652in}{2.749066in}}%
\pgfpathcurveto{\pgfqpoint{1.617652in}{2.754890in}}{\pgfqpoint{1.615338in}{2.760476in}}{\pgfqpoint{1.611220in}{2.764594in}}%
\pgfpathcurveto{\pgfqpoint{1.607102in}{2.768713in}}{\pgfqpoint{1.601515in}{2.771026in}}{\pgfqpoint{1.595691in}{2.771026in}}%
\pgfpathcurveto{\pgfqpoint{1.589867in}{2.771026in}}{\pgfqpoint{1.584281in}{2.768713in}}{\pgfqpoint{1.580163in}{2.764594in}}%
\pgfpathcurveto{\pgfqpoint{1.576045in}{2.760476in}}{\pgfqpoint{1.573731in}{2.754890in}}{\pgfqpoint{1.573731in}{2.749066in}}%
\pgfpathcurveto{\pgfqpoint{1.573731in}{2.743242in}}{\pgfqpoint{1.576045in}{2.737656in}}{\pgfqpoint{1.580163in}{2.733538in}}%
\pgfpathcurveto{\pgfqpoint{1.584281in}{2.729420in}}{\pgfqpoint{1.589867in}{2.727106in}}{\pgfqpoint{1.595691in}{2.727106in}}%
\pgfpathlineto{\pgfqpoint{1.595691in}{2.727106in}}%
\pgfpathclose%
\pgfusepath{fill}%
\end{pgfscope}%
\begin{pgfscope}%
\pgfpathrectangle{\pgfqpoint{0.288431in}{0.470524in}}{\pgfqpoint{4.447128in}{4.129476in}}%
\pgfusepath{clip}%
\pgfsetbuttcap%
\pgfsetroundjoin%
\definecolor{currentfill}{rgb}{0.000000,0.000000,1.000000}%
\pgfsetfillcolor{currentfill}%
\pgfsetlinewidth{0.000000pt}%
\definecolor{currentstroke}{rgb}{0.000000,0.000000,0.000000}%
\pgfsetstrokecolor{currentstroke}%
\pgfsetdash{}{0pt}%
\pgfpathmoveto{\pgfqpoint{1.656778in}{1.261020in}}%
\pgfpathcurveto{\pgfqpoint{1.662602in}{1.261020in}}{\pgfqpoint{1.668188in}{1.263334in}}{\pgfqpoint{1.672307in}{1.267452in}}%
\pgfpathcurveto{\pgfqpoint{1.676425in}{1.271570in}}{\pgfqpoint{1.678739in}{1.277156in}}{\pgfqpoint{1.678739in}{1.282980in}}%
\pgfpathcurveto{\pgfqpoint{1.678739in}{1.288804in}}{\pgfqpoint{1.676425in}{1.294390in}}{\pgfqpoint{1.672307in}{1.298508in}}%
\pgfpathcurveto{\pgfqpoint{1.668188in}{1.302626in}}{\pgfqpoint{1.662602in}{1.304940in}}{\pgfqpoint{1.656778in}{1.304940in}}%
\pgfpathcurveto{\pgfqpoint{1.650954in}{1.304940in}}{\pgfqpoint{1.645368in}{1.302626in}}{\pgfqpoint{1.641250in}{1.298508in}}%
\pgfpathcurveto{\pgfqpoint{1.637132in}{1.294390in}}{\pgfqpoint{1.634818in}{1.288804in}}{\pgfqpoint{1.634818in}{1.282980in}}%
\pgfpathcurveto{\pgfqpoint{1.634818in}{1.277156in}}{\pgfqpoint{1.637132in}{1.271570in}}{\pgfqpoint{1.641250in}{1.267452in}}%
\pgfpathcurveto{\pgfqpoint{1.645368in}{1.263334in}}{\pgfqpoint{1.650954in}{1.261020in}}{\pgfqpoint{1.656778in}{1.261020in}}%
\pgfpathlineto{\pgfqpoint{1.656778in}{1.261020in}}%
\pgfpathclose%
\pgfusepath{fill}%
\end{pgfscope}%
\begin{pgfscope}%
\pgfpathrectangle{\pgfqpoint{0.288431in}{0.470524in}}{\pgfqpoint{4.447128in}{4.129476in}}%
\pgfusepath{clip}%
\pgfsetbuttcap%
\pgfsetroundjoin%
\definecolor{currentfill}{rgb}{0.000000,0.000000,1.000000}%
\pgfsetfillcolor{currentfill}%
\pgfsetlinewidth{0.000000pt}%
\definecolor{currentstroke}{rgb}{0.000000,0.000000,0.000000}%
\pgfsetstrokecolor{currentstroke}%
\pgfsetdash{}{0pt}%
\pgfpathmoveto{\pgfqpoint{1.656778in}{3.765584in}}%
\pgfpathcurveto{\pgfqpoint{1.662602in}{3.765584in}}{\pgfqpoint{1.668188in}{3.767898in}}{\pgfqpoint{1.672307in}{3.772016in}}%
\pgfpathcurveto{\pgfqpoint{1.676425in}{3.776134in}}{\pgfqpoint{1.678739in}{3.781720in}}{\pgfqpoint{1.678739in}{3.787544in}}%
\pgfpathcurveto{\pgfqpoint{1.678739in}{3.793368in}}{\pgfqpoint{1.676425in}{3.798954in}}{\pgfqpoint{1.672307in}{3.803072in}}%
\pgfpathcurveto{\pgfqpoint{1.668188in}{3.807190in}}{\pgfqpoint{1.662602in}{3.809504in}}{\pgfqpoint{1.656778in}{3.809504in}}%
\pgfpathcurveto{\pgfqpoint{1.650954in}{3.809504in}}{\pgfqpoint{1.645368in}{3.807190in}}{\pgfqpoint{1.641250in}{3.803072in}}%
\pgfpathcurveto{\pgfqpoint{1.637132in}{3.798954in}}{\pgfqpoint{1.634818in}{3.793368in}}{\pgfqpoint{1.634818in}{3.787544in}}%
\pgfpathcurveto{\pgfqpoint{1.634818in}{3.781720in}}{\pgfqpoint{1.637132in}{3.776134in}}{\pgfqpoint{1.641250in}{3.772016in}}%
\pgfpathcurveto{\pgfqpoint{1.645368in}{3.767898in}}{\pgfqpoint{1.650954in}{3.765584in}}{\pgfqpoint{1.656778in}{3.765584in}}%
\pgfpathlineto{\pgfqpoint{1.656778in}{3.765584in}}%
\pgfpathclose%
\pgfusepath{fill}%
\end{pgfscope}%
\begin{pgfscope}%
\pgfpathrectangle{\pgfqpoint{0.288431in}{0.470524in}}{\pgfqpoint{4.447128in}{4.129476in}}%
\pgfusepath{clip}%
\pgfsetbuttcap%
\pgfsetroundjoin%
\definecolor{currentfill}{rgb}{0.000000,0.000000,1.000000}%
\pgfsetfillcolor{currentfill}%
\pgfsetlinewidth{0.000000pt}%
\definecolor{currentstroke}{rgb}{0.000000,0.000000,0.000000}%
\pgfsetstrokecolor{currentstroke}%
\pgfsetdash{}{0pt}%
\pgfpathmoveto{\pgfqpoint{1.717865in}{0.772324in}}%
\pgfpathcurveto{\pgfqpoint{1.723689in}{0.772324in}}{\pgfqpoint{1.729275in}{0.774638in}}{\pgfqpoint{1.733394in}{0.778756in}}%
\pgfpathcurveto{\pgfqpoint{1.737512in}{0.782874in}}{\pgfqpoint{1.739826in}{0.788461in}}{\pgfqpoint{1.739826in}{0.794285in}}%
\pgfpathcurveto{\pgfqpoint{1.739826in}{0.800109in}}{\pgfqpoint{1.737512in}{0.805695in}}{\pgfqpoint{1.733394in}{0.809813in}}%
\pgfpathcurveto{\pgfqpoint{1.729275in}{0.813931in}}{\pgfqpoint{1.723689in}{0.816245in}}{\pgfqpoint{1.717865in}{0.816245in}}%
\pgfpathcurveto{\pgfqpoint{1.712041in}{0.816245in}}{\pgfqpoint{1.706455in}{0.813931in}}{\pgfqpoint{1.702337in}{0.809813in}}%
\pgfpathcurveto{\pgfqpoint{1.698219in}{0.805695in}}{\pgfqpoint{1.695905in}{0.800109in}}{\pgfqpoint{1.695905in}{0.794285in}}%
\pgfpathcurveto{\pgfqpoint{1.695905in}{0.788461in}}{\pgfqpoint{1.698219in}{0.782874in}}{\pgfqpoint{1.702337in}{0.778756in}}%
\pgfpathcurveto{\pgfqpoint{1.706455in}{0.774638in}}{\pgfqpoint{1.712041in}{0.772324in}}{\pgfqpoint{1.717865in}{0.772324in}}%
\pgfpathlineto{\pgfqpoint{1.717865in}{0.772324in}}%
\pgfpathclose%
\pgfusepath{fill}%
\end{pgfscope}%
\begin{pgfscope}%
\pgfpathrectangle{\pgfqpoint{0.288431in}{0.470524in}}{\pgfqpoint{4.447128in}{4.129476in}}%
\pgfusepath{clip}%
\pgfsetbuttcap%
\pgfsetroundjoin%
\definecolor{currentfill}{rgb}{0.000000,0.000000,1.000000}%
\pgfsetfillcolor{currentfill}%
\pgfsetlinewidth{0.000000pt}%
\definecolor{currentstroke}{rgb}{0.000000,0.000000,0.000000}%
\pgfsetstrokecolor{currentstroke}%
\pgfsetdash{}{0pt}%
\pgfpathmoveto{\pgfqpoint{1.717865in}{4.254279in}}%
\pgfpathcurveto{\pgfqpoint{1.723689in}{4.254279in}}{\pgfqpoint{1.729275in}{4.256593in}}{\pgfqpoint{1.733394in}{4.260711in}}%
\pgfpathcurveto{\pgfqpoint{1.737512in}{4.264829in}}{\pgfqpoint{1.739826in}{4.270415in}}{\pgfqpoint{1.739826in}{4.276239in}}%
\pgfpathcurveto{\pgfqpoint{1.739826in}{4.282063in}}{\pgfqpoint{1.737512in}{4.287649in}}{\pgfqpoint{1.733394in}{4.291768in}}%
\pgfpathcurveto{\pgfqpoint{1.729275in}{4.295886in}}{\pgfqpoint{1.723689in}{4.298200in}}{\pgfqpoint{1.717865in}{4.298200in}}%
\pgfpathcurveto{\pgfqpoint{1.712041in}{4.298200in}}{\pgfqpoint{1.706455in}{4.295886in}}{\pgfqpoint{1.702337in}{4.291768in}}%
\pgfpathcurveto{\pgfqpoint{1.698219in}{4.287649in}}{\pgfqpoint{1.695905in}{4.282063in}}{\pgfqpoint{1.695905in}{4.276239in}}%
\pgfpathcurveto{\pgfqpoint{1.695905in}{4.270415in}}{\pgfqpoint{1.698219in}{4.264829in}}{\pgfqpoint{1.702337in}{4.260711in}}%
\pgfpathcurveto{\pgfqpoint{1.706455in}{4.256593in}}{\pgfqpoint{1.712041in}{4.254279in}}{\pgfqpoint{1.717865in}{4.254279in}}%
\pgfpathlineto{\pgfqpoint{1.717865in}{4.254279in}}%
\pgfpathclose%
\pgfusepath{fill}%
\end{pgfscope}%
\begin{pgfscope}%
\pgfpathrectangle{\pgfqpoint{0.288431in}{0.470524in}}{\pgfqpoint{4.447128in}{4.129476in}}%
\pgfusepath{clip}%
\pgfsetbuttcap%
\pgfsetroundjoin%
\definecolor{currentfill}{rgb}{0.000000,0.000000,1.000000}%
\pgfsetfillcolor{currentfill}%
\pgfsetlinewidth{0.000000pt}%
\definecolor{currentstroke}{rgb}{0.000000,0.000000,0.000000}%
\pgfsetstrokecolor{currentstroke}%
\pgfsetdash{}{0pt}%
\pgfpathmoveto{\pgfqpoint{1.778952in}{1.383194in}}%
\pgfpathcurveto{\pgfqpoint{1.784776in}{1.383194in}}{\pgfqpoint{1.790362in}{1.385507in}}{\pgfqpoint{1.794480in}{1.389626in}}%
\pgfpathcurveto{\pgfqpoint{1.798599in}{1.393744in}}{\pgfqpoint{1.800912in}{1.399330in}}{\pgfqpoint{1.800912in}{1.405154in}}%
\pgfpathcurveto{\pgfqpoint{1.800912in}{1.410978in}}{\pgfqpoint{1.798599in}{1.416564in}}{\pgfqpoint{1.794480in}{1.420682in}}%
\pgfpathcurveto{\pgfqpoint{1.790362in}{1.424800in}}{\pgfqpoint{1.784776in}{1.427114in}}{\pgfqpoint{1.778952in}{1.427114in}}%
\pgfpathcurveto{\pgfqpoint{1.773128in}{1.427114in}}{\pgfqpoint{1.767542in}{1.424800in}}{\pgfqpoint{1.763424in}{1.420682in}}%
\pgfpathcurveto{\pgfqpoint{1.759306in}{1.416564in}}{\pgfqpoint{1.756992in}{1.410978in}}{\pgfqpoint{1.756992in}{1.405154in}}%
\pgfpathcurveto{\pgfqpoint{1.756992in}{1.399330in}}{\pgfqpoint{1.759306in}{1.393744in}}{\pgfqpoint{1.763424in}{1.389626in}}%
\pgfpathcurveto{\pgfqpoint{1.767542in}{1.385507in}}{\pgfqpoint{1.773128in}{1.383194in}}{\pgfqpoint{1.778952in}{1.383194in}}%
\pgfpathlineto{\pgfqpoint{1.778952in}{1.383194in}}%
\pgfpathclose%
\pgfusepath{fill}%
\end{pgfscope}%
\begin{pgfscope}%
\pgfpathrectangle{\pgfqpoint{0.288431in}{0.470524in}}{\pgfqpoint{4.447128in}{4.129476in}}%
\pgfusepath{clip}%
\pgfsetbuttcap%
\pgfsetroundjoin%
\definecolor{currentfill}{rgb}{0.000000,0.000000,1.000000}%
\pgfsetfillcolor{currentfill}%
\pgfsetlinewidth{0.000000pt}%
\definecolor{currentstroke}{rgb}{0.000000,0.000000,0.000000}%
\pgfsetstrokecolor{currentstroke}%
\pgfsetdash{}{0pt}%
\pgfpathmoveto{\pgfqpoint{1.778952in}{3.643410in}}%
\pgfpathcurveto{\pgfqpoint{1.784776in}{3.643410in}}{\pgfqpoint{1.790362in}{3.645724in}}{\pgfqpoint{1.794480in}{3.649842in}}%
\pgfpathcurveto{\pgfqpoint{1.798599in}{3.653960in}}{\pgfqpoint{1.800912in}{3.659546in}}{\pgfqpoint{1.800912in}{3.665370in}}%
\pgfpathcurveto{\pgfqpoint{1.800912in}{3.671194in}}{\pgfqpoint{1.798599in}{3.676780in}}{\pgfqpoint{1.794480in}{3.680898in}}%
\pgfpathcurveto{\pgfqpoint{1.790362in}{3.685016in}}{\pgfqpoint{1.784776in}{3.687330in}}{\pgfqpoint{1.778952in}{3.687330in}}%
\pgfpathcurveto{\pgfqpoint{1.773128in}{3.687330in}}{\pgfqpoint{1.767542in}{3.685016in}}{\pgfqpoint{1.763424in}{3.680898in}}%
\pgfpathcurveto{\pgfqpoint{1.759306in}{3.676780in}}{\pgfqpoint{1.756992in}{3.671194in}}{\pgfqpoint{1.756992in}{3.665370in}}%
\pgfpathcurveto{\pgfqpoint{1.756992in}{3.659546in}}{\pgfqpoint{1.759306in}{3.653960in}}{\pgfqpoint{1.763424in}{3.649842in}}%
\pgfpathcurveto{\pgfqpoint{1.767542in}{3.645724in}}{\pgfqpoint{1.773128in}{3.643410in}}{\pgfqpoint{1.778952in}{3.643410in}}%
\pgfpathlineto{\pgfqpoint{1.778952in}{3.643410in}}%
\pgfpathclose%
\pgfusepath{fill}%
\end{pgfscope}%
\begin{pgfscope}%
\pgfpathrectangle{\pgfqpoint{0.288431in}{0.470524in}}{\pgfqpoint{4.447128in}{4.129476in}}%
\pgfusepath{clip}%
\pgfsetbuttcap%
\pgfsetroundjoin%
\definecolor{currentfill}{rgb}{0.000000,0.000000,1.000000}%
\pgfsetfillcolor{currentfill}%
\pgfsetlinewidth{0.000000pt}%
\definecolor{currentstroke}{rgb}{0.000000,0.000000,0.000000}%
\pgfsetstrokecolor{currentstroke}%
\pgfsetdash{}{0pt}%
\pgfpathmoveto{\pgfqpoint{1.962213in}{1.627541in}}%
\pgfpathcurveto{\pgfqpoint{1.968037in}{1.627541in}}{\pgfqpoint{1.973623in}{1.629855in}}{\pgfqpoint{1.977741in}{1.633973in}}%
\pgfpathcurveto{\pgfqpoint{1.981859in}{1.638091in}}{\pgfqpoint{1.984173in}{1.643678in}}{\pgfqpoint{1.984173in}{1.649502in}}%
\pgfpathcurveto{\pgfqpoint{1.984173in}{1.655325in}}{\pgfqpoint{1.981859in}{1.660912in}}{\pgfqpoint{1.977741in}{1.665030in}}%
\pgfpathcurveto{\pgfqpoint{1.973623in}{1.669148in}}{\pgfqpoint{1.968037in}{1.671462in}}{\pgfqpoint{1.962213in}{1.671462in}}%
\pgfpathcurveto{\pgfqpoint{1.956389in}{1.671462in}}{\pgfqpoint{1.950803in}{1.669148in}}{\pgfqpoint{1.946685in}{1.665030in}}%
\pgfpathcurveto{\pgfqpoint{1.942567in}{1.660912in}}{\pgfqpoint{1.940253in}{1.655325in}}{\pgfqpoint{1.940253in}{1.649502in}}%
\pgfpathcurveto{\pgfqpoint{1.940253in}{1.643678in}}{\pgfqpoint{1.942567in}{1.638091in}}{\pgfqpoint{1.946685in}{1.633973in}}%
\pgfpathcurveto{\pgfqpoint{1.950803in}{1.629855in}}{\pgfqpoint{1.956389in}{1.627541in}}{\pgfqpoint{1.962213in}{1.627541in}}%
\pgfpathlineto{\pgfqpoint{1.962213in}{1.627541in}}%
\pgfpathclose%
\pgfusepath{fill}%
\end{pgfscope}%
\begin{pgfscope}%
\pgfpathrectangle{\pgfqpoint{0.288431in}{0.470524in}}{\pgfqpoint{4.447128in}{4.129476in}}%
\pgfusepath{clip}%
\pgfsetbuttcap%
\pgfsetroundjoin%
\definecolor{currentfill}{rgb}{0.000000,0.000000,1.000000}%
\pgfsetfillcolor{currentfill}%
\pgfsetlinewidth{0.000000pt}%
\definecolor{currentstroke}{rgb}{0.000000,0.000000,0.000000}%
\pgfsetstrokecolor{currentstroke}%
\pgfsetdash{}{0pt}%
\pgfpathmoveto{\pgfqpoint{1.962213in}{3.399062in}}%
\pgfpathcurveto{\pgfqpoint{1.968037in}{3.399062in}}{\pgfqpoint{1.973623in}{3.401376in}}{\pgfqpoint{1.977741in}{3.405494in}}%
\pgfpathcurveto{\pgfqpoint{1.981859in}{3.409612in}}{\pgfqpoint{1.984173in}{3.415198in}}{\pgfqpoint{1.984173in}{3.421022in}}%
\pgfpathcurveto{\pgfqpoint{1.984173in}{3.426846in}}{\pgfqpoint{1.981859in}{3.432432in}}{\pgfqpoint{1.977741in}{3.436551in}}%
\pgfpathcurveto{\pgfqpoint{1.973623in}{3.440669in}}{\pgfqpoint{1.968037in}{3.442983in}}{\pgfqpoint{1.962213in}{3.442983in}}%
\pgfpathcurveto{\pgfqpoint{1.956389in}{3.442983in}}{\pgfqpoint{1.950803in}{3.440669in}}{\pgfqpoint{1.946685in}{3.436551in}}%
\pgfpathcurveto{\pgfqpoint{1.942567in}{3.432432in}}{\pgfqpoint{1.940253in}{3.426846in}}{\pgfqpoint{1.940253in}{3.421022in}}%
\pgfpathcurveto{\pgfqpoint{1.940253in}{3.415198in}}{\pgfqpoint{1.942567in}{3.409612in}}{\pgfqpoint{1.946685in}{3.405494in}}%
\pgfpathcurveto{\pgfqpoint{1.950803in}{3.401376in}}{\pgfqpoint{1.956389in}{3.399062in}}{\pgfqpoint{1.962213in}{3.399062in}}%
\pgfpathlineto{\pgfqpoint{1.962213in}{3.399062in}}%
\pgfpathclose%
\pgfusepath{fill}%
\end{pgfscope}%
\begin{pgfscope}%
\pgfpathrectangle{\pgfqpoint{0.288431in}{0.470524in}}{\pgfqpoint{4.447128in}{4.129476in}}%
\pgfusepath{clip}%
\pgfsetbuttcap%
\pgfsetroundjoin%
\definecolor{currentfill}{rgb}{0.000000,0.000000,1.000000}%
\pgfsetfillcolor{currentfill}%
\pgfsetlinewidth{0.000000pt}%
\definecolor{currentstroke}{rgb}{0.000000,0.000000,0.000000}%
\pgfsetstrokecolor{currentstroke}%
\pgfsetdash{}{0pt}%
\pgfpathmoveto{\pgfqpoint{2.084387in}{1.688628in}}%
\pgfpathcurveto{\pgfqpoint{2.090211in}{1.688628in}}{\pgfqpoint{2.095797in}{1.690942in}}{\pgfqpoint{2.099915in}{1.695060in}}%
\pgfpathcurveto{\pgfqpoint{2.104033in}{1.699178in}}{\pgfqpoint{2.106347in}{1.704765in}}{\pgfqpoint{2.106347in}{1.710588in}}%
\pgfpathcurveto{\pgfqpoint{2.106347in}{1.716412in}}{\pgfqpoint{2.104033in}{1.721999in}}{\pgfqpoint{2.099915in}{1.726117in}}%
\pgfpathcurveto{\pgfqpoint{2.095797in}{1.730235in}}{\pgfqpoint{2.090211in}{1.732549in}}{\pgfqpoint{2.084387in}{1.732549in}}%
\pgfpathcurveto{\pgfqpoint{2.078563in}{1.732549in}}{\pgfqpoint{2.072977in}{1.730235in}}{\pgfqpoint{2.068859in}{1.726117in}}%
\pgfpathcurveto{\pgfqpoint{2.064740in}{1.721999in}}{\pgfqpoint{2.062427in}{1.716412in}}{\pgfqpoint{2.062427in}{1.710588in}}%
\pgfpathcurveto{\pgfqpoint{2.062427in}{1.704765in}}{\pgfqpoint{2.064740in}{1.699178in}}{\pgfqpoint{2.068859in}{1.695060in}}%
\pgfpathcurveto{\pgfqpoint{2.072977in}{1.690942in}}{\pgfqpoint{2.078563in}{1.688628in}}{\pgfqpoint{2.084387in}{1.688628in}}%
\pgfpathlineto{\pgfqpoint{2.084387in}{1.688628in}}%
\pgfpathclose%
\pgfusepath{fill}%
\end{pgfscope}%
\begin{pgfscope}%
\pgfpathrectangle{\pgfqpoint{0.288431in}{0.470524in}}{\pgfqpoint{4.447128in}{4.129476in}}%
\pgfusepath{clip}%
\pgfsetbuttcap%
\pgfsetroundjoin%
\definecolor{currentfill}{rgb}{0.000000,0.000000,1.000000}%
\pgfsetfillcolor{currentfill}%
\pgfsetlinewidth{0.000000pt}%
\definecolor{currentstroke}{rgb}{0.000000,0.000000,0.000000}%
\pgfsetstrokecolor{currentstroke}%
\pgfsetdash{}{0pt}%
\pgfpathmoveto{\pgfqpoint{2.084387in}{3.337975in}}%
\pgfpathcurveto{\pgfqpoint{2.090211in}{3.337975in}}{\pgfqpoint{2.095797in}{3.340289in}}{\pgfqpoint{2.099915in}{3.344407in}}%
\pgfpathcurveto{\pgfqpoint{2.104033in}{3.348525in}}{\pgfqpoint{2.106347in}{3.354111in}}{\pgfqpoint{2.106347in}{3.359935in}}%
\pgfpathcurveto{\pgfqpoint{2.106347in}{3.365759in}}{\pgfqpoint{2.104033in}{3.371346in}}{\pgfqpoint{2.099915in}{3.375464in}}%
\pgfpathcurveto{\pgfqpoint{2.095797in}{3.379582in}}{\pgfqpoint{2.090211in}{3.381896in}}{\pgfqpoint{2.084387in}{3.381896in}}%
\pgfpathcurveto{\pgfqpoint{2.078563in}{3.381896in}}{\pgfqpoint{2.072977in}{3.379582in}}{\pgfqpoint{2.068859in}{3.375464in}}%
\pgfpathcurveto{\pgfqpoint{2.064740in}{3.371346in}}{\pgfqpoint{2.062427in}{3.365759in}}{\pgfqpoint{2.062427in}{3.359935in}}%
\pgfpathcurveto{\pgfqpoint{2.062427in}{3.354111in}}{\pgfqpoint{2.064740in}{3.348525in}}{\pgfqpoint{2.068859in}{3.344407in}}%
\pgfpathcurveto{\pgfqpoint{2.072977in}{3.340289in}}{\pgfqpoint{2.078563in}{3.337975in}}{\pgfqpoint{2.084387in}{3.337975in}}%
\pgfpathlineto{\pgfqpoint{2.084387in}{3.337975in}}%
\pgfpathclose%
\pgfusepath{fill}%
\end{pgfscope}%
\begin{pgfscope}%
\pgfpathrectangle{\pgfqpoint{0.288431in}{0.470524in}}{\pgfqpoint{4.447128in}{4.129476in}}%
\pgfusepath{clip}%
\pgfsetbuttcap%
\pgfsetroundjoin%
\definecolor{currentfill}{rgb}{0.000000,0.000000,1.000000}%
\pgfsetfillcolor{currentfill}%
\pgfsetlinewidth{0.000000pt}%
\definecolor{currentstroke}{rgb}{0.000000,0.000000,0.000000}%
\pgfsetstrokecolor{currentstroke}%
\pgfsetdash{}{0pt}%
\pgfpathmoveto{\pgfqpoint{2.145474in}{0.527977in}}%
\pgfpathcurveto{\pgfqpoint{2.151298in}{0.527977in}}{\pgfqpoint{2.156884in}{0.530291in}}{\pgfqpoint{2.161002in}{0.534409in}}%
\pgfpathcurveto{\pgfqpoint{2.165120in}{0.538527in}}{\pgfqpoint{2.167434in}{0.544113in}}{\pgfqpoint{2.167434in}{0.549937in}}%
\pgfpathcurveto{\pgfqpoint{2.167434in}{0.555761in}}{\pgfqpoint{2.165120in}{0.561347in}}{\pgfqpoint{2.161002in}{0.565465in}}%
\pgfpathcurveto{\pgfqpoint{2.156884in}{0.569583in}}{\pgfqpoint{2.151298in}{0.571897in}}{\pgfqpoint{2.145474in}{0.571897in}}%
\pgfpathcurveto{\pgfqpoint{2.139650in}{0.571897in}}{\pgfqpoint{2.134064in}{0.569583in}}{\pgfqpoint{2.129945in}{0.565465in}}%
\pgfpathcurveto{\pgfqpoint{2.125827in}{0.561347in}}{\pgfqpoint{2.123513in}{0.555761in}}{\pgfqpoint{2.123513in}{0.549937in}}%
\pgfpathcurveto{\pgfqpoint{2.123513in}{0.544113in}}{\pgfqpoint{2.125827in}{0.538527in}}{\pgfqpoint{2.129945in}{0.534409in}}%
\pgfpathcurveto{\pgfqpoint{2.134064in}{0.530291in}}{\pgfqpoint{2.139650in}{0.527977in}}{\pgfqpoint{2.145474in}{0.527977in}}%
\pgfpathlineto{\pgfqpoint{2.145474in}{0.527977in}}%
\pgfpathclose%
\pgfusepath{fill}%
\end{pgfscope}%
\begin{pgfscope}%
\pgfpathrectangle{\pgfqpoint{0.288431in}{0.470524in}}{\pgfqpoint{4.447128in}{4.129476in}}%
\pgfusepath{clip}%
\pgfsetbuttcap%
\pgfsetroundjoin%
\definecolor{currentfill}{rgb}{0.000000,0.000000,1.000000}%
\pgfsetfillcolor{currentfill}%
\pgfsetlinewidth{0.000000pt}%
\definecolor{currentstroke}{rgb}{0.000000,0.000000,0.000000}%
\pgfsetstrokecolor{currentstroke}%
\pgfsetdash{}{0pt}%
\pgfpathmoveto{\pgfqpoint{2.145474in}{4.498627in}}%
\pgfpathcurveto{\pgfqpoint{2.151298in}{4.498627in}}{\pgfqpoint{2.156884in}{4.500941in}}{\pgfqpoint{2.161002in}{4.505059in}}%
\pgfpathcurveto{\pgfqpoint{2.165120in}{4.509177in}}{\pgfqpoint{2.167434in}{4.514763in}}{\pgfqpoint{2.167434in}{4.520587in}}%
\pgfpathcurveto{\pgfqpoint{2.167434in}{4.526411in}}{\pgfqpoint{2.165120in}{4.531997in}}{\pgfqpoint{2.161002in}{4.536115in}}%
\pgfpathcurveto{\pgfqpoint{2.156884in}{4.540233in}}{\pgfqpoint{2.151298in}{4.542547in}}{\pgfqpoint{2.145474in}{4.542547in}}%
\pgfpathcurveto{\pgfqpoint{2.139650in}{4.542547in}}{\pgfqpoint{2.134064in}{4.540233in}}{\pgfqpoint{2.129945in}{4.536115in}}%
\pgfpathcurveto{\pgfqpoint{2.125827in}{4.531997in}}{\pgfqpoint{2.123513in}{4.526411in}}{\pgfqpoint{2.123513in}{4.520587in}}%
\pgfpathcurveto{\pgfqpoint{2.123513in}{4.514763in}}{\pgfqpoint{2.125827in}{4.509177in}}{\pgfqpoint{2.129945in}{4.505059in}}%
\pgfpathcurveto{\pgfqpoint{2.134064in}{4.500941in}}{\pgfqpoint{2.139650in}{4.498627in}}{\pgfqpoint{2.145474in}{4.498627in}}%
\pgfpathlineto{\pgfqpoint{2.145474in}{4.498627in}}%
\pgfpathclose%
\pgfusepath{fill}%
\end{pgfscope}%
\begin{pgfscope}%
\pgfpathrectangle{\pgfqpoint{0.288431in}{0.470524in}}{\pgfqpoint{4.447128in}{4.129476in}}%
\pgfusepath{clip}%
\pgfsetbuttcap%
\pgfsetroundjoin%
\definecolor{currentfill}{rgb}{0.000000,0.000000,1.000000}%
\pgfsetfillcolor{currentfill}%
\pgfsetlinewidth{0.000000pt}%
\definecolor{currentstroke}{rgb}{0.000000,0.000000,0.000000}%
\pgfsetstrokecolor{currentstroke}%
\pgfsetdash{}{0pt}%
\pgfpathmoveto{\pgfqpoint{2.450908in}{0.527977in}}%
\pgfpathcurveto{\pgfqpoint{2.456732in}{0.527977in}}{\pgfqpoint{2.462318in}{0.530291in}}{\pgfqpoint{2.466437in}{0.534409in}}%
\pgfpathcurveto{\pgfqpoint{2.470555in}{0.538527in}}{\pgfqpoint{2.472869in}{0.544113in}}{\pgfqpoint{2.472869in}{0.549937in}}%
\pgfpathcurveto{\pgfqpoint{2.472869in}{0.555761in}}{\pgfqpoint{2.470555in}{0.561347in}}{\pgfqpoint{2.466437in}{0.565465in}}%
\pgfpathcurveto{\pgfqpoint{2.462318in}{0.569583in}}{\pgfqpoint{2.456732in}{0.571897in}}{\pgfqpoint{2.450908in}{0.571897in}}%
\pgfpathcurveto{\pgfqpoint{2.445084in}{0.571897in}}{\pgfqpoint{2.439498in}{0.569583in}}{\pgfqpoint{2.435380in}{0.565465in}}%
\pgfpathcurveto{\pgfqpoint{2.431262in}{0.561347in}}{\pgfqpoint{2.428948in}{0.555761in}}{\pgfqpoint{2.428948in}{0.549937in}}%
\pgfpathcurveto{\pgfqpoint{2.428948in}{0.544113in}}{\pgfqpoint{2.431262in}{0.538527in}}{\pgfqpoint{2.435380in}{0.534409in}}%
\pgfpathcurveto{\pgfqpoint{2.439498in}{0.530291in}}{\pgfqpoint{2.445084in}{0.527977in}}{\pgfqpoint{2.450908in}{0.527977in}}%
\pgfpathlineto{\pgfqpoint{2.450908in}{0.527977in}}%
\pgfpathclose%
\pgfusepath{fill}%
\end{pgfscope}%
\begin{pgfscope}%
\pgfpathrectangle{\pgfqpoint{0.288431in}{0.470524in}}{\pgfqpoint{4.447128in}{4.129476in}}%
\pgfusepath{clip}%
\pgfsetbuttcap%
\pgfsetroundjoin%
\definecolor{currentfill}{rgb}{0.000000,0.000000,1.000000}%
\pgfsetfillcolor{currentfill}%
\pgfsetlinewidth{0.000000pt}%
\definecolor{currentstroke}{rgb}{0.000000,0.000000,0.000000}%
\pgfsetstrokecolor{currentstroke}%
\pgfsetdash{}{0pt}%
\pgfpathmoveto{\pgfqpoint{2.450908in}{4.498627in}}%
\pgfpathcurveto{\pgfqpoint{2.456732in}{4.498627in}}{\pgfqpoint{2.462318in}{4.500941in}}{\pgfqpoint{2.466437in}{4.505059in}}%
\pgfpathcurveto{\pgfqpoint{2.470555in}{4.509177in}}{\pgfqpoint{2.472869in}{4.514763in}}{\pgfqpoint{2.472869in}{4.520587in}}%
\pgfpathcurveto{\pgfqpoint{2.472869in}{4.526411in}}{\pgfqpoint{2.470555in}{4.531997in}}{\pgfqpoint{2.466437in}{4.536115in}}%
\pgfpathcurveto{\pgfqpoint{2.462318in}{4.540233in}}{\pgfqpoint{2.456732in}{4.542547in}}{\pgfqpoint{2.450908in}{4.542547in}}%
\pgfpathcurveto{\pgfqpoint{2.445084in}{4.542547in}}{\pgfqpoint{2.439498in}{4.540233in}}{\pgfqpoint{2.435380in}{4.536115in}}%
\pgfpathcurveto{\pgfqpoint{2.431262in}{4.531997in}}{\pgfqpoint{2.428948in}{4.526411in}}{\pgfqpoint{2.428948in}{4.520587in}}%
\pgfpathcurveto{\pgfqpoint{2.428948in}{4.514763in}}{\pgfqpoint{2.431262in}{4.509177in}}{\pgfqpoint{2.435380in}{4.505059in}}%
\pgfpathcurveto{\pgfqpoint{2.439498in}{4.500941in}}{\pgfqpoint{2.445084in}{4.498627in}}{\pgfqpoint{2.450908in}{4.498627in}}%
\pgfpathlineto{\pgfqpoint{2.450908in}{4.498627in}}%
\pgfpathclose%
\pgfusepath{fill}%
\end{pgfscope}%
\begin{pgfscope}%
\pgfpathrectangle{\pgfqpoint{0.288431in}{0.470524in}}{\pgfqpoint{4.447128in}{4.129476in}}%
\pgfusepath{clip}%
\pgfsetbuttcap%
\pgfsetroundjoin%
\definecolor{currentfill}{rgb}{0.000000,0.000000,1.000000}%
\pgfsetfillcolor{currentfill}%
\pgfsetlinewidth{0.000000pt}%
\definecolor{currentstroke}{rgb}{0.000000,0.000000,0.000000}%
\pgfsetstrokecolor{currentstroke}%
\pgfsetdash{}{0pt}%
\pgfpathmoveto{\pgfqpoint{2.511995in}{0.955585in}}%
\pgfpathcurveto{\pgfqpoint{2.517819in}{0.955585in}}{\pgfqpoint{2.523405in}{0.957899in}}{\pgfqpoint{2.527524in}{0.962017in}}%
\pgfpathcurveto{\pgfqpoint{2.531642in}{0.966135in}}{\pgfqpoint{2.533956in}{0.971721in}}{\pgfqpoint{2.533956in}{0.977545in}}%
\pgfpathcurveto{\pgfqpoint{2.533956in}{0.983369in}}{\pgfqpoint{2.531642in}{0.988955in}}{\pgfqpoint{2.527524in}{0.993074in}}%
\pgfpathcurveto{\pgfqpoint{2.523405in}{0.997192in}}{\pgfqpoint{2.517819in}{0.999506in}}{\pgfqpoint{2.511995in}{0.999506in}}%
\pgfpathcurveto{\pgfqpoint{2.506171in}{0.999506in}}{\pgfqpoint{2.500585in}{0.997192in}}{\pgfqpoint{2.496467in}{0.993074in}}%
\pgfpathcurveto{\pgfqpoint{2.492349in}{0.988955in}}{\pgfqpoint{2.490035in}{0.983369in}}{\pgfqpoint{2.490035in}{0.977545in}}%
\pgfpathcurveto{\pgfqpoint{2.490035in}{0.971721in}}{\pgfqpoint{2.492349in}{0.966135in}}{\pgfqpoint{2.496467in}{0.962017in}}%
\pgfpathcurveto{\pgfqpoint{2.500585in}{0.957899in}}{\pgfqpoint{2.506171in}{0.955585in}}{\pgfqpoint{2.511995in}{0.955585in}}%
\pgfpathlineto{\pgfqpoint{2.511995in}{0.955585in}}%
\pgfpathclose%
\pgfusepath{fill}%
\end{pgfscope}%
\begin{pgfscope}%
\pgfpathrectangle{\pgfqpoint{0.288431in}{0.470524in}}{\pgfqpoint{4.447128in}{4.129476in}}%
\pgfusepath{clip}%
\pgfsetbuttcap%
\pgfsetroundjoin%
\definecolor{currentfill}{rgb}{0.000000,0.000000,1.000000}%
\pgfsetfillcolor{currentfill}%
\pgfsetlinewidth{0.000000pt}%
\definecolor{currentstroke}{rgb}{0.000000,0.000000,0.000000}%
\pgfsetstrokecolor{currentstroke}%
\pgfsetdash{}{0pt}%
\pgfpathmoveto{\pgfqpoint{2.511995in}{4.071018in}}%
\pgfpathcurveto{\pgfqpoint{2.517819in}{4.071018in}}{\pgfqpoint{2.523405in}{4.073332in}}{\pgfqpoint{2.527524in}{4.077450in}}%
\pgfpathcurveto{\pgfqpoint{2.531642in}{4.081568in}}{\pgfqpoint{2.533956in}{4.087155in}}{\pgfqpoint{2.533956in}{4.092979in}}%
\pgfpathcurveto{\pgfqpoint{2.533956in}{4.098802in}}{\pgfqpoint{2.531642in}{4.104389in}}{\pgfqpoint{2.527524in}{4.108507in}}%
\pgfpathcurveto{\pgfqpoint{2.523405in}{4.112625in}}{\pgfqpoint{2.517819in}{4.114939in}}{\pgfqpoint{2.511995in}{4.114939in}}%
\pgfpathcurveto{\pgfqpoint{2.506171in}{4.114939in}}{\pgfqpoint{2.500585in}{4.112625in}}{\pgfqpoint{2.496467in}{4.108507in}}%
\pgfpathcurveto{\pgfqpoint{2.492349in}{4.104389in}}{\pgfqpoint{2.490035in}{4.098802in}}{\pgfqpoint{2.490035in}{4.092979in}}%
\pgfpathcurveto{\pgfqpoint{2.490035in}{4.087155in}}{\pgfqpoint{2.492349in}{4.081568in}}{\pgfqpoint{2.496467in}{4.077450in}}%
\pgfpathcurveto{\pgfqpoint{2.500585in}{4.073332in}}{\pgfqpoint{2.506171in}{4.071018in}}{\pgfqpoint{2.511995in}{4.071018in}}%
\pgfpathlineto{\pgfqpoint{2.511995in}{4.071018in}}%
\pgfpathclose%
\pgfusepath{fill}%
\end{pgfscope}%
\begin{pgfscope}%
\pgfpathrectangle{\pgfqpoint{0.288431in}{0.470524in}}{\pgfqpoint{4.447128in}{4.129476in}}%
\pgfusepath{clip}%
\pgfsetbuttcap%
\pgfsetroundjoin%
\definecolor{currentfill}{rgb}{0.000000,0.000000,1.000000}%
\pgfsetfillcolor{currentfill}%
\pgfsetlinewidth{0.000000pt}%
\definecolor{currentstroke}{rgb}{0.000000,0.000000,0.000000}%
\pgfsetstrokecolor{currentstroke}%
\pgfsetdash{}{0pt}%
\pgfpathmoveto{\pgfqpoint{2.695256in}{2.360584in}}%
\pgfpathcurveto{\pgfqpoint{2.701080in}{2.360584in}}{\pgfqpoint{2.706666in}{2.362898in}}{\pgfqpoint{2.710784in}{2.367016in}}%
\pgfpathcurveto{\pgfqpoint{2.714902in}{2.371135in}}{\pgfqpoint{2.717216in}{2.376721in}}{\pgfqpoint{2.717216in}{2.382545in}}%
\pgfpathcurveto{\pgfqpoint{2.717216in}{2.388369in}}{\pgfqpoint{2.714902in}{2.393955in}}{\pgfqpoint{2.710784in}{2.398073in}}%
\pgfpathcurveto{\pgfqpoint{2.706666in}{2.402191in}}{\pgfqpoint{2.701080in}{2.404505in}}{\pgfqpoint{2.695256in}{2.404505in}}%
\pgfpathcurveto{\pgfqpoint{2.689432in}{2.404505in}}{\pgfqpoint{2.683846in}{2.402191in}}{\pgfqpoint{2.679728in}{2.398073in}}%
\pgfpathcurveto{\pgfqpoint{2.675610in}{2.393955in}}{\pgfqpoint{2.673296in}{2.388369in}}{\pgfqpoint{2.673296in}{2.382545in}}%
\pgfpathcurveto{\pgfqpoint{2.673296in}{2.376721in}}{\pgfqpoint{2.675610in}{2.371135in}}{\pgfqpoint{2.679728in}{2.367016in}}%
\pgfpathcurveto{\pgfqpoint{2.683846in}{2.362898in}}{\pgfqpoint{2.689432in}{2.360584in}}{\pgfqpoint{2.695256in}{2.360584in}}%
\pgfpathlineto{\pgfqpoint{2.695256in}{2.360584in}}%
\pgfpathclose%
\pgfusepath{fill}%
\end{pgfscope}%
\begin{pgfscope}%
\pgfpathrectangle{\pgfqpoint{0.288431in}{0.470524in}}{\pgfqpoint{4.447128in}{4.129476in}}%
\pgfusepath{clip}%
\pgfsetbuttcap%
\pgfsetroundjoin%
\definecolor{currentfill}{rgb}{0.000000,0.000000,1.000000}%
\pgfsetfillcolor{currentfill}%
\pgfsetlinewidth{0.000000pt}%
\definecolor{currentstroke}{rgb}{0.000000,0.000000,0.000000}%
\pgfsetstrokecolor{currentstroke}%
\pgfsetdash{}{0pt}%
\pgfpathmoveto{\pgfqpoint{2.695256in}{2.666019in}}%
\pgfpathcurveto{\pgfqpoint{2.701080in}{2.666019in}}{\pgfqpoint{2.706666in}{2.668333in}}{\pgfqpoint{2.710784in}{2.672451in}}%
\pgfpathcurveto{\pgfqpoint{2.714902in}{2.676569in}}{\pgfqpoint{2.717216in}{2.682155in}}{\pgfqpoint{2.717216in}{2.687979in}}%
\pgfpathcurveto{\pgfqpoint{2.717216in}{2.693803in}}{\pgfqpoint{2.714902in}{2.699389in}}{\pgfqpoint{2.710784in}{2.703508in}}%
\pgfpathcurveto{\pgfqpoint{2.706666in}{2.707626in}}{\pgfqpoint{2.701080in}{2.709940in}}{\pgfqpoint{2.695256in}{2.709940in}}%
\pgfpathcurveto{\pgfqpoint{2.689432in}{2.709940in}}{\pgfqpoint{2.683846in}{2.707626in}}{\pgfqpoint{2.679728in}{2.703508in}}%
\pgfpathcurveto{\pgfqpoint{2.675610in}{2.699389in}}{\pgfqpoint{2.673296in}{2.693803in}}{\pgfqpoint{2.673296in}{2.687979in}}%
\pgfpathcurveto{\pgfqpoint{2.673296in}{2.682155in}}{\pgfqpoint{2.675610in}{2.676569in}}{\pgfqpoint{2.679728in}{2.672451in}}%
\pgfpathcurveto{\pgfqpoint{2.683846in}{2.668333in}}{\pgfqpoint{2.689432in}{2.666019in}}{\pgfqpoint{2.695256in}{2.666019in}}%
\pgfpathlineto{\pgfqpoint{2.695256in}{2.666019in}}%
\pgfpathclose%
\pgfusepath{fill}%
\end{pgfscope}%
\begin{pgfscope}%
\pgfpathrectangle{\pgfqpoint{0.288431in}{0.470524in}}{\pgfqpoint{4.447128in}{4.129476in}}%
\pgfusepath{clip}%
\pgfsetbuttcap%
\pgfsetroundjoin%
\definecolor{currentfill}{rgb}{0.000000,0.000000,1.000000}%
\pgfsetfillcolor{currentfill}%
\pgfsetlinewidth{0.000000pt}%
\definecolor{currentstroke}{rgb}{0.000000,0.000000,0.000000}%
\pgfsetstrokecolor{currentstroke}%
\pgfsetdash{}{0pt}%
\pgfpathmoveto{\pgfqpoint{2.878517in}{1.261020in}}%
\pgfpathcurveto{\pgfqpoint{2.884341in}{1.261020in}}{\pgfqpoint{2.889927in}{1.263334in}}{\pgfqpoint{2.894045in}{1.267452in}}%
\pgfpathcurveto{\pgfqpoint{2.898163in}{1.271570in}}{\pgfqpoint{2.900477in}{1.277156in}}{\pgfqpoint{2.900477in}{1.282980in}}%
\pgfpathcurveto{\pgfqpoint{2.900477in}{1.288804in}}{\pgfqpoint{2.898163in}{1.294390in}}{\pgfqpoint{2.894045in}{1.298508in}}%
\pgfpathcurveto{\pgfqpoint{2.889927in}{1.302626in}}{\pgfqpoint{2.884341in}{1.304940in}}{\pgfqpoint{2.878517in}{1.304940in}}%
\pgfpathcurveto{\pgfqpoint{2.872693in}{1.304940in}}{\pgfqpoint{2.867107in}{1.302626in}}{\pgfqpoint{2.862989in}{1.298508in}}%
\pgfpathcurveto{\pgfqpoint{2.858870in}{1.294390in}}{\pgfqpoint{2.856557in}{1.288804in}}{\pgfqpoint{2.856557in}{1.282980in}}%
\pgfpathcurveto{\pgfqpoint{2.856557in}{1.277156in}}{\pgfqpoint{2.858870in}{1.271570in}}{\pgfqpoint{2.862989in}{1.267452in}}%
\pgfpathcurveto{\pgfqpoint{2.867107in}{1.263334in}}{\pgfqpoint{2.872693in}{1.261020in}}{\pgfqpoint{2.878517in}{1.261020in}}%
\pgfpathlineto{\pgfqpoint{2.878517in}{1.261020in}}%
\pgfpathclose%
\pgfusepath{fill}%
\end{pgfscope}%
\begin{pgfscope}%
\pgfpathrectangle{\pgfqpoint{0.288431in}{0.470524in}}{\pgfqpoint{4.447128in}{4.129476in}}%
\pgfusepath{clip}%
\pgfsetbuttcap%
\pgfsetroundjoin%
\definecolor{currentfill}{rgb}{0.000000,0.000000,1.000000}%
\pgfsetfillcolor{currentfill}%
\pgfsetlinewidth{0.000000pt}%
\definecolor{currentstroke}{rgb}{0.000000,0.000000,0.000000}%
\pgfsetstrokecolor{currentstroke}%
\pgfsetdash{}{0pt}%
\pgfpathmoveto{\pgfqpoint{2.878517in}{3.765584in}}%
\pgfpathcurveto{\pgfqpoint{2.884341in}{3.765584in}}{\pgfqpoint{2.889927in}{3.767898in}}{\pgfqpoint{2.894045in}{3.772016in}}%
\pgfpathcurveto{\pgfqpoint{2.898163in}{3.776134in}}{\pgfqpoint{2.900477in}{3.781720in}}{\pgfqpoint{2.900477in}{3.787544in}}%
\pgfpathcurveto{\pgfqpoint{2.900477in}{3.793368in}}{\pgfqpoint{2.898163in}{3.798954in}}{\pgfqpoint{2.894045in}{3.803072in}}%
\pgfpathcurveto{\pgfqpoint{2.889927in}{3.807190in}}{\pgfqpoint{2.884341in}{3.809504in}}{\pgfqpoint{2.878517in}{3.809504in}}%
\pgfpathcurveto{\pgfqpoint{2.872693in}{3.809504in}}{\pgfqpoint{2.867107in}{3.807190in}}{\pgfqpoint{2.862989in}{3.803072in}}%
\pgfpathcurveto{\pgfqpoint{2.858870in}{3.798954in}}{\pgfqpoint{2.856557in}{3.793368in}}{\pgfqpoint{2.856557in}{3.787544in}}%
\pgfpathcurveto{\pgfqpoint{2.856557in}{3.781720in}}{\pgfqpoint{2.858870in}{3.776134in}}{\pgfqpoint{2.862989in}{3.772016in}}%
\pgfpathcurveto{\pgfqpoint{2.867107in}{3.767898in}}{\pgfqpoint{2.872693in}{3.765584in}}{\pgfqpoint{2.878517in}{3.765584in}}%
\pgfpathlineto{\pgfqpoint{2.878517in}{3.765584in}}%
\pgfpathclose%
\pgfusepath{fill}%
\end{pgfscope}%
\begin{pgfscope}%
\pgfpathrectangle{\pgfqpoint{0.288431in}{0.470524in}}{\pgfqpoint{4.447128in}{4.129476in}}%
\pgfusepath{clip}%
\pgfsetbuttcap%
\pgfsetroundjoin%
\definecolor{currentfill}{rgb}{0.000000,0.000000,1.000000}%
\pgfsetfillcolor{currentfill}%
\pgfsetlinewidth{0.000000pt}%
\definecolor{currentstroke}{rgb}{0.000000,0.000000,0.000000}%
\pgfsetstrokecolor{currentstroke}%
\pgfsetdash{}{0pt}%
\pgfpathmoveto{\pgfqpoint{3.122865in}{2.116237in}}%
\pgfpathcurveto{\pgfqpoint{3.128688in}{2.116237in}}{\pgfqpoint{3.134275in}{2.118551in}}{\pgfqpoint{3.138393in}{2.122669in}}%
\pgfpathcurveto{\pgfqpoint{3.142511in}{2.126787in}}{\pgfqpoint{3.144825in}{2.132373in}}{\pgfqpoint{3.144825in}{2.138197in}}%
\pgfpathcurveto{\pgfqpoint{3.144825in}{2.144021in}}{\pgfqpoint{3.142511in}{2.149607in}}{\pgfqpoint{3.138393in}{2.153725in}}%
\pgfpathcurveto{\pgfqpoint{3.134275in}{2.157843in}}{\pgfqpoint{3.128688in}{2.160157in}}{\pgfqpoint{3.122865in}{2.160157in}}%
\pgfpathcurveto{\pgfqpoint{3.117041in}{2.160157in}}{\pgfqpoint{3.111454in}{2.157843in}}{\pgfqpoint{3.107336in}{2.153725in}}%
\pgfpathcurveto{\pgfqpoint{3.103218in}{2.149607in}}{\pgfqpoint{3.100904in}{2.144021in}}{\pgfqpoint{3.100904in}{2.138197in}}%
\pgfpathcurveto{\pgfqpoint{3.100904in}{2.132373in}}{\pgfqpoint{3.103218in}{2.126787in}}{\pgfqpoint{3.107336in}{2.122669in}}%
\pgfpathcurveto{\pgfqpoint{3.111454in}{2.118551in}}{\pgfqpoint{3.117041in}{2.116237in}}{\pgfqpoint{3.122865in}{2.116237in}}%
\pgfpathlineto{\pgfqpoint{3.122865in}{2.116237in}}%
\pgfpathclose%
\pgfusepath{fill}%
\end{pgfscope}%
\begin{pgfscope}%
\pgfpathrectangle{\pgfqpoint{0.288431in}{0.470524in}}{\pgfqpoint{4.447128in}{4.129476in}}%
\pgfusepath{clip}%
\pgfsetbuttcap%
\pgfsetroundjoin%
\definecolor{currentfill}{rgb}{0.000000,0.000000,1.000000}%
\pgfsetfillcolor{currentfill}%
\pgfsetlinewidth{0.000000pt}%
\definecolor{currentstroke}{rgb}{0.000000,0.000000,0.000000}%
\pgfsetstrokecolor{currentstroke}%
\pgfsetdash{}{0pt}%
\pgfpathmoveto{\pgfqpoint{3.122865in}{2.910367in}}%
\pgfpathcurveto{\pgfqpoint{3.128688in}{2.910367in}}{\pgfqpoint{3.134275in}{2.912681in}}{\pgfqpoint{3.138393in}{2.916799in}}%
\pgfpathcurveto{\pgfqpoint{3.142511in}{2.920917in}}{\pgfqpoint{3.144825in}{2.926503in}}{\pgfqpoint{3.144825in}{2.932327in}}%
\pgfpathcurveto{\pgfqpoint{3.144825in}{2.938151in}}{\pgfqpoint{3.142511in}{2.943737in}}{\pgfqpoint{3.138393in}{2.947855in}}%
\pgfpathcurveto{\pgfqpoint{3.134275in}{2.951973in}}{\pgfqpoint{3.128688in}{2.954287in}}{\pgfqpoint{3.122865in}{2.954287in}}%
\pgfpathcurveto{\pgfqpoint{3.117041in}{2.954287in}}{\pgfqpoint{3.111454in}{2.951973in}}{\pgfqpoint{3.107336in}{2.947855in}}%
\pgfpathcurveto{\pgfqpoint{3.103218in}{2.943737in}}{\pgfqpoint{3.100904in}{2.938151in}}{\pgfqpoint{3.100904in}{2.932327in}}%
\pgfpathcurveto{\pgfqpoint{3.100904in}{2.926503in}}{\pgfqpoint{3.103218in}{2.920917in}}{\pgfqpoint{3.107336in}{2.916799in}}%
\pgfpathcurveto{\pgfqpoint{3.111454in}{2.912681in}}{\pgfqpoint{3.117041in}{2.910367in}}{\pgfqpoint{3.122865in}{2.910367in}}%
\pgfpathlineto{\pgfqpoint{3.122865in}{2.910367in}}%
\pgfpathclose%
\pgfusepath{fill}%
\end{pgfscope}%
\begin{pgfscope}%
\pgfpathrectangle{\pgfqpoint{0.288431in}{0.470524in}}{\pgfqpoint{4.447128in}{4.129476in}}%
\pgfusepath{clip}%
\pgfsetbuttcap%
\pgfsetroundjoin%
\definecolor{currentfill}{rgb}{0.000000,0.000000,1.000000}%
\pgfsetfillcolor{currentfill}%
\pgfsetlinewidth{0.000000pt}%
\definecolor{currentstroke}{rgb}{0.000000,0.000000,0.000000}%
\pgfsetstrokecolor{currentstroke}%
\pgfsetdash{}{0pt}%
\pgfpathmoveto{\pgfqpoint{3.489386in}{1.566454in}}%
\pgfpathcurveto{\pgfqpoint{3.495210in}{1.566454in}}{\pgfqpoint{3.500796in}{1.568768in}}{\pgfqpoint{3.504914in}{1.572886in}}%
\pgfpathcurveto{\pgfqpoint{3.509032in}{1.577005in}}{\pgfqpoint{3.511346in}{1.582591in}}{\pgfqpoint{3.511346in}{1.588415in}}%
\pgfpathcurveto{\pgfqpoint{3.511346in}{1.594239in}}{\pgfqpoint{3.509032in}{1.599825in}}{\pgfqpoint{3.504914in}{1.603943in}}%
\pgfpathcurveto{\pgfqpoint{3.500796in}{1.608061in}}{\pgfqpoint{3.495210in}{1.610375in}}{\pgfqpoint{3.489386in}{1.610375in}}%
\pgfpathcurveto{\pgfqpoint{3.483562in}{1.610375in}}{\pgfqpoint{3.477976in}{1.608061in}}{\pgfqpoint{3.473858in}{1.603943in}}%
\pgfpathcurveto{\pgfqpoint{3.469740in}{1.599825in}}{\pgfqpoint{3.467426in}{1.594239in}}{\pgfqpoint{3.467426in}{1.588415in}}%
\pgfpathcurveto{\pgfqpoint{3.467426in}{1.582591in}}{\pgfqpoint{3.469740in}{1.577005in}}{\pgfqpoint{3.473858in}{1.572886in}}%
\pgfpathcurveto{\pgfqpoint{3.477976in}{1.568768in}}{\pgfqpoint{3.483562in}{1.566454in}}{\pgfqpoint{3.489386in}{1.566454in}}%
\pgfpathlineto{\pgfqpoint{3.489386in}{1.566454in}}%
\pgfpathclose%
\pgfusepath{fill}%
\end{pgfscope}%
\begin{pgfscope}%
\pgfpathrectangle{\pgfqpoint{0.288431in}{0.470524in}}{\pgfqpoint{4.447128in}{4.129476in}}%
\pgfusepath{clip}%
\pgfsetbuttcap%
\pgfsetroundjoin%
\definecolor{currentfill}{rgb}{0.000000,0.000000,1.000000}%
\pgfsetfillcolor{currentfill}%
\pgfsetlinewidth{0.000000pt}%
\definecolor{currentstroke}{rgb}{0.000000,0.000000,0.000000}%
\pgfsetstrokecolor{currentstroke}%
\pgfsetdash{}{0pt}%
\pgfpathmoveto{\pgfqpoint{3.489386in}{3.460149in}}%
\pgfpathcurveto{\pgfqpoint{3.495210in}{3.460149in}}{\pgfqpoint{3.500796in}{3.462463in}}{\pgfqpoint{3.504914in}{3.466581in}}%
\pgfpathcurveto{\pgfqpoint{3.509032in}{3.470699in}}{\pgfqpoint{3.511346in}{3.476285in}}{\pgfqpoint{3.511346in}{3.482109in}}%
\pgfpathcurveto{\pgfqpoint{3.511346in}{3.487933in}}{\pgfqpoint{3.509032in}{3.493519in}}{\pgfqpoint{3.504914in}{3.497638in}}%
\pgfpathcurveto{\pgfqpoint{3.500796in}{3.501756in}}{\pgfqpoint{3.495210in}{3.504070in}}{\pgfqpoint{3.489386in}{3.504070in}}%
\pgfpathcurveto{\pgfqpoint{3.483562in}{3.504070in}}{\pgfqpoint{3.477976in}{3.501756in}}{\pgfqpoint{3.473858in}{3.497638in}}%
\pgfpathcurveto{\pgfqpoint{3.469740in}{3.493519in}}{\pgfqpoint{3.467426in}{3.487933in}}{\pgfqpoint{3.467426in}{3.482109in}}%
\pgfpathcurveto{\pgfqpoint{3.467426in}{3.476285in}}{\pgfqpoint{3.469740in}{3.470699in}}{\pgfqpoint{3.473858in}{3.466581in}}%
\pgfpathcurveto{\pgfqpoint{3.477976in}{3.462463in}}{\pgfqpoint{3.483562in}{3.460149in}}{\pgfqpoint{3.489386in}{3.460149in}}%
\pgfpathlineto{\pgfqpoint{3.489386in}{3.460149in}}%
\pgfpathclose%
\pgfusepath{fill}%
\end{pgfscope}%
\begin{pgfscope}%
\pgfpathrectangle{\pgfqpoint{0.288431in}{0.470524in}}{\pgfqpoint{4.447128in}{4.129476in}}%
\pgfusepath{clip}%
\pgfsetbuttcap%
\pgfsetroundjoin%
\definecolor{currentfill}{rgb}{0.000000,0.000000,1.000000}%
\pgfsetfillcolor{currentfill}%
\pgfsetlinewidth{0.000000pt}%
\definecolor{currentstroke}{rgb}{0.000000,0.000000,0.000000}%
\pgfsetstrokecolor{currentstroke}%
\pgfsetdash{}{0pt}%
\pgfpathmoveto{\pgfqpoint{3.550473in}{0.772324in}}%
\pgfpathcurveto{\pgfqpoint{3.556297in}{0.772324in}}{\pgfqpoint{3.561883in}{0.774638in}}{\pgfqpoint{3.566001in}{0.778756in}}%
\pgfpathcurveto{\pgfqpoint{3.570119in}{0.782874in}}{\pgfqpoint{3.572433in}{0.788461in}}{\pgfqpoint{3.572433in}{0.794285in}}%
\pgfpathcurveto{\pgfqpoint{3.572433in}{0.800109in}}{\pgfqpoint{3.570119in}{0.805695in}}{\pgfqpoint{3.566001in}{0.809813in}}%
\pgfpathcurveto{\pgfqpoint{3.561883in}{0.813931in}}{\pgfqpoint{3.556297in}{0.816245in}}{\pgfqpoint{3.550473in}{0.816245in}}%
\pgfpathcurveto{\pgfqpoint{3.544649in}{0.816245in}}{\pgfqpoint{3.539063in}{0.813931in}}{\pgfqpoint{3.534945in}{0.809813in}}%
\pgfpathcurveto{\pgfqpoint{3.530827in}{0.805695in}}{\pgfqpoint{3.528513in}{0.800109in}}{\pgfqpoint{3.528513in}{0.794285in}}%
\pgfpathcurveto{\pgfqpoint{3.528513in}{0.788461in}}{\pgfqpoint{3.530827in}{0.782874in}}{\pgfqpoint{3.534945in}{0.778756in}}%
\pgfpathcurveto{\pgfqpoint{3.539063in}{0.774638in}}{\pgfqpoint{3.544649in}{0.772324in}}{\pgfqpoint{3.550473in}{0.772324in}}%
\pgfpathlineto{\pgfqpoint{3.550473in}{0.772324in}}%
\pgfpathclose%
\pgfusepath{fill}%
\end{pgfscope}%
\begin{pgfscope}%
\pgfpathrectangle{\pgfqpoint{0.288431in}{0.470524in}}{\pgfqpoint{4.447128in}{4.129476in}}%
\pgfusepath{clip}%
\pgfsetbuttcap%
\pgfsetroundjoin%
\definecolor{currentfill}{rgb}{0.000000,0.000000,1.000000}%
\pgfsetfillcolor{currentfill}%
\pgfsetlinewidth{0.000000pt}%
\definecolor{currentstroke}{rgb}{0.000000,0.000000,0.000000}%
\pgfsetstrokecolor{currentstroke}%
\pgfsetdash{}{0pt}%
\pgfpathmoveto{\pgfqpoint{3.550473in}{4.254279in}}%
\pgfpathcurveto{\pgfqpoint{3.556297in}{4.254279in}}{\pgfqpoint{3.561883in}{4.256593in}}{\pgfqpoint{3.566001in}{4.260711in}}%
\pgfpathcurveto{\pgfqpoint{3.570119in}{4.264829in}}{\pgfqpoint{3.572433in}{4.270415in}}{\pgfqpoint{3.572433in}{4.276239in}}%
\pgfpathcurveto{\pgfqpoint{3.572433in}{4.282063in}}{\pgfqpoint{3.570119in}{4.287649in}}{\pgfqpoint{3.566001in}{4.291768in}}%
\pgfpathcurveto{\pgfqpoint{3.561883in}{4.295886in}}{\pgfqpoint{3.556297in}{4.298200in}}{\pgfqpoint{3.550473in}{4.298200in}}%
\pgfpathcurveto{\pgfqpoint{3.544649in}{4.298200in}}{\pgfqpoint{3.539063in}{4.295886in}}{\pgfqpoint{3.534945in}{4.291768in}}%
\pgfpathcurveto{\pgfqpoint{3.530827in}{4.287649in}}{\pgfqpoint{3.528513in}{4.282063in}}{\pgfqpoint{3.528513in}{4.276239in}}%
\pgfpathcurveto{\pgfqpoint{3.528513in}{4.270415in}}{\pgfqpoint{3.530827in}{4.264829in}}{\pgfqpoint{3.534945in}{4.260711in}}%
\pgfpathcurveto{\pgfqpoint{3.539063in}{4.256593in}}{\pgfqpoint{3.544649in}{4.254279in}}{\pgfqpoint{3.550473in}{4.254279in}}%
\pgfpathlineto{\pgfqpoint{3.550473in}{4.254279in}}%
\pgfpathclose%
\pgfusepath{fill}%
\end{pgfscope}%
\begin{pgfscope}%
\pgfpathrectangle{\pgfqpoint{0.288431in}{0.470524in}}{\pgfqpoint{4.447128in}{4.129476in}}%
\pgfusepath{clip}%
\pgfsetbuttcap%
\pgfsetroundjoin%
\definecolor{currentfill}{rgb}{0.000000,0.000000,1.000000}%
\pgfsetfillcolor{currentfill}%
\pgfsetlinewidth{0.000000pt}%
\definecolor{currentstroke}{rgb}{0.000000,0.000000,0.000000}%
\pgfsetstrokecolor{currentstroke}%
\pgfsetdash{}{0pt}%
\pgfpathmoveto{\pgfqpoint{3.611560in}{0.894498in}}%
\pgfpathcurveto{\pgfqpoint{3.617384in}{0.894498in}}{\pgfqpoint{3.622970in}{0.896812in}}{\pgfqpoint{3.627088in}{0.900930in}}%
\pgfpathcurveto{\pgfqpoint{3.631206in}{0.905048in}}{\pgfqpoint{3.633520in}{0.910635in}}{\pgfqpoint{3.633520in}{0.916458in}}%
\pgfpathcurveto{\pgfqpoint{3.633520in}{0.922282in}}{\pgfqpoint{3.631206in}{0.927869in}}{\pgfqpoint{3.627088in}{0.931987in}}%
\pgfpathcurveto{\pgfqpoint{3.622970in}{0.936105in}}{\pgfqpoint{3.617384in}{0.938419in}}{\pgfqpoint{3.611560in}{0.938419in}}%
\pgfpathcurveto{\pgfqpoint{3.605736in}{0.938419in}}{\pgfqpoint{3.600150in}{0.936105in}}{\pgfqpoint{3.596032in}{0.931987in}}%
\pgfpathcurveto{\pgfqpoint{3.591914in}{0.927869in}}{\pgfqpoint{3.589600in}{0.922282in}}{\pgfqpoint{3.589600in}{0.916458in}}%
\pgfpathcurveto{\pgfqpoint{3.589600in}{0.910635in}}{\pgfqpoint{3.591914in}{0.905048in}}{\pgfqpoint{3.596032in}{0.900930in}}%
\pgfpathcurveto{\pgfqpoint{3.600150in}{0.896812in}}{\pgfqpoint{3.605736in}{0.894498in}}{\pgfqpoint{3.611560in}{0.894498in}}%
\pgfpathlineto{\pgfqpoint{3.611560in}{0.894498in}}%
\pgfpathclose%
\pgfusepath{fill}%
\end{pgfscope}%
\begin{pgfscope}%
\pgfpathrectangle{\pgfqpoint{0.288431in}{0.470524in}}{\pgfqpoint{4.447128in}{4.129476in}}%
\pgfusepath{clip}%
\pgfsetbuttcap%
\pgfsetroundjoin%
\definecolor{currentfill}{rgb}{0.000000,0.000000,1.000000}%
\pgfsetfillcolor{currentfill}%
\pgfsetlinewidth{0.000000pt}%
\definecolor{currentstroke}{rgb}{0.000000,0.000000,0.000000}%
\pgfsetstrokecolor{currentstroke}%
\pgfsetdash{}{0pt}%
\pgfpathmoveto{\pgfqpoint{3.611560in}{4.132105in}}%
\pgfpathcurveto{\pgfqpoint{3.617384in}{4.132105in}}{\pgfqpoint{3.622970in}{4.134419in}}{\pgfqpoint{3.627088in}{4.138537in}}%
\pgfpathcurveto{\pgfqpoint{3.631206in}{4.142655in}}{\pgfqpoint{3.633520in}{4.148242in}}{\pgfqpoint{3.633520in}{4.154065in}}%
\pgfpathcurveto{\pgfqpoint{3.633520in}{4.159889in}}{\pgfqpoint{3.631206in}{4.165476in}}{\pgfqpoint{3.627088in}{4.169594in}}%
\pgfpathcurveto{\pgfqpoint{3.622970in}{4.173712in}}{\pgfqpoint{3.617384in}{4.176026in}}{\pgfqpoint{3.611560in}{4.176026in}}%
\pgfpathcurveto{\pgfqpoint{3.605736in}{4.176026in}}{\pgfqpoint{3.600150in}{4.173712in}}{\pgfqpoint{3.596032in}{4.169594in}}%
\pgfpathcurveto{\pgfqpoint{3.591914in}{4.165476in}}{\pgfqpoint{3.589600in}{4.159889in}}{\pgfqpoint{3.589600in}{4.154065in}}%
\pgfpathcurveto{\pgfqpoint{3.589600in}{4.148242in}}{\pgfqpoint{3.591914in}{4.142655in}}{\pgfqpoint{3.596032in}{4.138537in}}%
\pgfpathcurveto{\pgfqpoint{3.600150in}{4.134419in}}{\pgfqpoint{3.605736in}{4.132105in}}{\pgfqpoint{3.611560in}{4.132105in}}%
\pgfpathlineto{\pgfqpoint{3.611560in}{4.132105in}}%
\pgfpathclose%
\pgfusepath{fill}%
\end{pgfscope}%
\begin{pgfscope}%
\pgfpathrectangle{\pgfqpoint{0.288431in}{0.470524in}}{\pgfqpoint{4.447128in}{4.129476in}}%
\pgfusepath{clip}%
\pgfsetbuttcap%
\pgfsetroundjoin%
\definecolor{currentfill}{rgb}{0.000000,0.000000,1.000000}%
\pgfsetfillcolor{currentfill}%
\pgfsetlinewidth{0.000000pt}%
\definecolor{currentstroke}{rgb}{0.000000,0.000000,0.000000}%
\pgfsetstrokecolor{currentstroke}%
\pgfsetdash{}{0pt}%
\pgfpathmoveto{\pgfqpoint{3.672647in}{1.383194in}}%
\pgfpathcurveto{\pgfqpoint{3.678471in}{1.383194in}}{\pgfqpoint{3.684057in}{1.385507in}}{\pgfqpoint{3.688175in}{1.389626in}}%
\pgfpathcurveto{\pgfqpoint{3.692293in}{1.393744in}}{\pgfqpoint{3.694607in}{1.399330in}}{\pgfqpoint{3.694607in}{1.405154in}}%
\pgfpathcurveto{\pgfqpoint{3.694607in}{1.410978in}}{\pgfqpoint{3.692293in}{1.416564in}}{\pgfqpoint{3.688175in}{1.420682in}}%
\pgfpathcurveto{\pgfqpoint{3.684057in}{1.424800in}}{\pgfqpoint{3.678471in}{1.427114in}}{\pgfqpoint{3.672647in}{1.427114in}}%
\pgfpathcurveto{\pgfqpoint{3.666823in}{1.427114in}}{\pgfqpoint{3.661237in}{1.424800in}}{\pgfqpoint{3.657119in}{1.420682in}}%
\pgfpathcurveto{\pgfqpoint{3.653000in}{1.416564in}}{\pgfqpoint{3.650687in}{1.410978in}}{\pgfqpoint{3.650687in}{1.405154in}}%
\pgfpathcurveto{\pgfqpoint{3.650687in}{1.399330in}}{\pgfqpoint{3.653000in}{1.393744in}}{\pgfqpoint{3.657119in}{1.389626in}}%
\pgfpathcurveto{\pgfqpoint{3.661237in}{1.385507in}}{\pgfqpoint{3.666823in}{1.383194in}}{\pgfqpoint{3.672647in}{1.383194in}}%
\pgfpathlineto{\pgfqpoint{3.672647in}{1.383194in}}%
\pgfpathclose%
\pgfusepath{fill}%
\end{pgfscope}%
\begin{pgfscope}%
\pgfpathrectangle{\pgfqpoint{0.288431in}{0.470524in}}{\pgfqpoint{4.447128in}{4.129476in}}%
\pgfusepath{clip}%
\pgfsetbuttcap%
\pgfsetroundjoin%
\definecolor{currentfill}{rgb}{0.000000,0.000000,1.000000}%
\pgfsetfillcolor{currentfill}%
\pgfsetlinewidth{0.000000pt}%
\definecolor{currentstroke}{rgb}{0.000000,0.000000,0.000000}%
\pgfsetstrokecolor{currentstroke}%
\pgfsetdash{}{0pt}%
\pgfpathmoveto{\pgfqpoint{3.672647in}{3.643410in}}%
\pgfpathcurveto{\pgfqpoint{3.678471in}{3.643410in}}{\pgfqpoint{3.684057in}{3.645724in}}{\pgfqpoint{3.688175in}{3.649842in}}%
\pgfpathcurveto{\pgfqpoint{3.692293in}{3.653960in}}{\pgfqpoint{3.694607in}{3.659546in}}{\pgfqpoint{3.694607in}{3.665370in}}%
\pgfpathcurveto{\pgfqpoint{3.694607in}{3.671194in}}{\pgfqpoint{3.692293in}{3.676780in}}{\pgfqpoint{3.688175in}{3.680898in}}%
\pgfpathcurveto{\pgfqpoint{3.684057in}{3.685016in}}{\pgfqpoint{3.678471in}{3.687330in}}{\pgfqpoint{3.672647in}{3.687330in}}%
\pgfpathcurveto{\pgfqpoint{3.666823in}{3.687330in}}{\pgfqpoint{3.661237in}{3.685016in}}{\pgfqpoint{3.657119in}{3.680898in}}%
\pgfpathcurveto{\pgfqpoint{3.653000in}{3.676780in}}{\pgfqpoint{3.650687in}{3.671194in}}{\pgfqpoint{3.650687in}{3.665370in}}%
\pgfpathcurveto{\pgfqpoint{3.650687in}{3.659546in}}{\pgfqpoint{3.653000in}{3.653960in}}{\pgfqpoint{3.657119in}{3.649842in}}%
\pgfpathcurveto{\pgfqpoint{3.661237in}{3.645724in}}{\pgfqpoint{3.666823in}{3.643410in}}{\pgfqpoint{3.672647in}{3.643410in}}%
\pgfpathlineto{\pgfqpoint{3.672647in}{3.643410in}}%
\pgfpathclose%
\pgfusepath{fill}%
\end{pgfscope}%
\begin{pgfscope}%
\pgfpathrectangle{\pgfqpoint{0.288431in}{0.470524in}}{\pgfqpoint{4.447128in}{4.129476in}}%
\pgfusepath{clip}%
\pgfsetbuttcap%
\pgfsetroundjoin%
\definecolor{currentfill}{rgb}{0.000000,0.000000,1.000000}%
\pgfsetfillcolor{currentfill}%
\pgfsetlinewidth{0.000000pt}%
\definecolor{currentstroke}{rgb}{0.000000,0.000000,0.000000}%
\pgfsetstrokecolor{currentstroke}%
\pgfsetdash{}{0pt}%
\pgfpathmoveto{\pgfqpoint{3.733734in}{1.444281in}}%
\pgfpathcurveto{\pgfqpoint{3.739558in}{1.444281in}}{\pgfqpoint{3.745144in}{1.446594in}}{\pgfqpoint{3.749262in}{1.450713in}}%
\pgfpathcurveto{\pgfqpoint{3.753380in}{1.454831in}}{\pgfqpoint{3.755694in}{1.460417in}}{\pgfqpoint{3.755694in}{1.466241in}}%
\pgfpathcurveto{\pgfqpoint{3.755694in}{1.472065in}}{\pgfqpoint{3.753380in}{1.477651in}}{\pgfqpoint{3.749262in}{1.481769in}}%
\pgfpathcurveto{\pgfqpoint{3.745144in}{1.485887in}}{\pgfqpoint{3.739558in}{1.488201in}}{\pgfqpoint{3.733734in}{1.488201in}}%
\pgfpathcurveto{\pgfqpoint{3.727910in}{1.488201in}}{\pgfqpoint{3.722324in}{1.485887in}}{\pgfqpoint{3.718206in}{1.481769in}}%
\pgfpathcurveto{\pgfqpoint{3.714087in}{1.477651in}}{\pgfqpoint{3.711774in}{1.472065in}}{\pgfqpoint{3.711774in}{1.466241in}}%
\pgfpathcurveto{\pgfqpoint{3.711774in}{1.460417in}}{\pgfqpoint{3.714087in}{1.454831in}}{\pgfqpoint{3.718206in}{1.450713in}}%
\pgfpathcurveto{\pgfqpoint{3.722324in}{1.446594in}}{\pgfqpoint{3.727910in}{1.444281in}}{\pgfqpoint{3.733734in}{1.444281in}}%
\pgfpathlineto{\pgfqpoint{3.733734in}{1.444281in}}%
\pgfpathclose%
\pgfusepath{fill}%
\end{pgfscope}%
\begin{pgfscope}%
\pgfpathrectangle{\pgfqpoint{0.288431in}{0.470524in}}{\pgfqpoint{4.447128in}{4.129476in}}%
\pgfusepath{clip}%
\pgfsetbuttcap%
\pgfsetroundjoin%
\definecolor{currentfill}{rgb}{0.000000,0.000000,1.000000}%
\pgfsetfillcolor{currentfill}%
\pgfsetlinewidth{0.000000pt}%
\definecolor{currentstroke}{rgb}{0.000000,0.000000,0.000000}%
\pgfsetstrokecolor{currentstroke}%
\pgfsetdash{}{0pt}%
\pgfpathmoveto{\pgfqpoint{3.733734in}{3.582323in}}%
\pgfpathcurveto{\pgfqpoint{3.739558in}{3.582323in}}{\pgfqpoint{3.745144in}{3.584637in}}{\pgfqpoint{3.749262in}{3.588755in}}%
\pgfpathcurveto{\pgfqpoint{3.753380in}{3.592873in}}{\pgfqpoint{3.755694in}{3.598459in}}{\pgfqpoint{3.755694in}{3.604283in}}%
\pgfpathcurveto{\pgfqpoint{3.755694in}{3.610107in}}{\pgfqpoint{3.753380in}{3.615693in}}{\pgfqpoint{3.749262in}{3.619811in}}%
\pgfpathcurveto{\pgfqpoint{3.745144in}{3.623930in}}{\pgfqpoint{3.739558in}{3.626243in}}{\pgfqpoint{3.733734in}{3.626243in}}%
\pgfpathcurveto{\pgfqpoint{3.727910in}{3.626243in}}{\pgfqpoint{3.722324in}{3.623930in}}{\pgfqpoint{3.718206in}{3.619811in}}%
\pgfpathcurveto{\pgfqpoint{3.714087in}{3.615693in}}{\pgfqpoint{3.711774in}{3.610107in}}{\pgfqpoint{3.711774in}{3.604283in}}%
\pgfpathcurveto{\pgfqpoint{3.711774in}{3.598459in}}{\pgfqpoint{3.714087in}{3.592873in}}{\pgfqpoint{3.718206in}{3.588755in}}%
\pgfpathcurveto{\pgfqpoint{3.722324in}{3.584637in}}{\pgfqpoint{3.727910in}{3.582323in}}{\pgfqpoint{3.733734in}{3.582323in}}%
\pgfpathlineto{\pgfqpoint{3.733734in}{3.582323in}}%
\pgfpathclose%
\pgfusepath{fill}%
\end{pgfscope}%
\begin{pgfscope}%
\pgfpathrectangle{\pgfqpoint{0.288431in}{0.470524in}}{\pgfqpoint{4.447128in}{4.129476in}}%
\pgfusepath{clip}%
\pgfsetbuttcap%
\pgfsetroundjoin%
\definecolor{currentfill}{rgb}{0.000000,0.000000,1.000000}%
\pgfsetfillcolor{currentfill}%
\pgfsetlinewidth{0.000000pt}%
\definecolor{currentstroke}{rgb}{0.000000,0.000000,0.000000}%
\pgfsetstrokecolor{currentstroke}%
\pgfsetdash{}{0pt}%
\pgfpathmoveto{\pgfqpoint{3.794821in}{1.016672in}}%
\pgfpathcurveto{\pgfqpoint{3.800645in}{1.016672in}}{\pgfqpoint{3.806231in}{1.018986in}}{\pgfqpoint{3.810349in}{1.023104in}}%
\pgfpathcurveto{\pgfqpoint{3.814467in}{1.027222in}}{\pgfqpoint{3.816781in}{1.032808in}}{\pgfqpoint{3.816781in}{1.038632in}}%
\pgfpathcurveto{\pgfqpoint{3.816781in}{1.044456in}}{\pgfqpoint{3.814467in}{1.050042in}}{\pgfqpoint{3.810349in}{1.054161in}}%
\pgfpathcurveto{\pgfqpoint{3.806231in}{1.058279in}}{\pgfqpoint{3.800645in}{1.060593in}}{\pgfqpoint{3.794821in}{1.060593in}}%
\pgfpathcurveto{\pgfqpoint{3.788997in}{1.060593in}}{\pgfqpoint{3.783411in}{1.058279in}}{\pgfqpoint{3.779292in}{1.054161in}}%
\pgfpathcurveto{\pgfqpoint{3.775174in}{1.050042in}}{\pgfqpoint{3.772860in}{1.044456in}}{\pgfqpoint{3.772860in}{1.038632in}}%
\pgfpathcurveto{\pgfqpoint{3.772860in}{1.032808in}}{\pgfqpoint{3.775174in}{1.027222in}}{\pgfqpoint{3.779292in}{1.023104in}}%
\pgfpathcurveto{\pgfqpoint{3.783411in}{1.018986in}}{\pgfqpoint{3.788997in}{1.016672in}}{\pgfqpoint{3.794821in}{1.016672in}}%
\pgfpathlineto{\pgfqpoint{3.794821in}{1.016672in}}%
\pgfpathclose%
\pgfusepath{fill}%
\end{pgfscope}%
\begin{pgfscope}%
\pgfpathrectangle{\pgfqpoint{0.288431in}{0.470524in}}{\pgfqpoint{4.447128in}{4.129476in}}%
\pgfusepath{clip}%
\pgfsetbuttcap%
\pgfsetroundjoin%
\definecolor{currentfill}{rgb}{0.000000,0.000000,1.000000}%
\pgfsetfillcolor{currentfill}%
\pgfsetlinewidth{0.000000pt}%
\definecolor{currentstroke}{rgb}{0.000000,0.000000,0.000000}%
\pgfsetstrokecolor{currentstroke}%
\pgfsetdash{}{0pt}%
\pgfpathmoveto{\pgfqpoint{3.794821in}{4.009931in}}%
\pgfpathcurveto{\pgfqpoint{3.800645in}{4.009931in}}{\pgfqpoint{3.806231in}{4.012245in}}{\pgfqpoint{3.810349in}{4.016363in}}%
\pgfpathcurveto{\pgfqpoint{3.814467in}{4.020481in}}{\pgfqpoint{3.816781in}{4.026068in}}{\pgfqpoint{3.816781in}{4.031892in}}%
\pgfpathcurveto{\pgfqpoint{3.816781in}{4.037716in}}{\pgfqpoint{3.814467in}{4.043302in}}{\pgfqpoint{3.810349in}{4.047420in}}%
\pgfpathcurveto{\pgfqpoint{3.806231in}{4.051538in}}{\pgfqpoint{3.800645in}{4.053852in}}{\pgfqpoint{3.794821in}{4.053852in}}%
\pgfpathcurveto{\pgfqpoint{3.788997in}{4.053852in}}{\pgfqpoint{3.783411in}{4.051538in}}{\pgfqpoint{3.779292in}{4.047420in}}%
\pgfpathcurveto{\pgfqpoint{3.775174in}{4.043302in}}{\pgfqpoint{3.772860in}{4.037716in}}{\pgfqpoint{3.772860in}{4.031892in}}%
\pgfpathcurveto{\pgfqpoint{3.772860in}{4.026068in}}{\pgfqpoint{3.775174in}{4.020481in}}{\pgfqpoint{3.779292in}{4.016363in}}%
\pgfpathcurveto{\pgfqpoint{3.783411in}{4.012245in}}{\pgfqpoint{3.788997in}{4.009931in}}{\pgfqpoint{3.794821in}{4.009931in}}%
\pgfpathlineto{\pgfqpoint{3.794821in}{4.009931in}}%
\pgfpathclose%
\pgfusepath{fill}%
\end{pgfscope}%
\begin{pgfscope}%
\pgfpathrectangle{\pgfqpoint{0.288431in}{0.470524in}}{\pgfqpoint{4.447128in}{4.129476in}}%
\pgfusepath{clip}%
\pgfsetbuttcap%
\pgfsetroundjoin%
\definecolor{currentfill}{rgb}{0.000000,0.000000,1.000000}%
\pgfsetfillcolor{currentfill}%
\pgfsetlinewidth{0.000000pt}%
\definecolor{currentstroke}{rgb}{0.000000,0.000000,0.000000}%
\pgfsetstrokecolor{currentstroke}%
\pgfsetdash{}{0pt}%
\pgfpathmoveto{\pgfqpoint{3.916995in}{1.994063in}}%
\pgfpathcurveto{\pgfqpoint{3.922818in}{1.994063in}}{\pgfqpoint{3.928405in}{1.996377in}}{\pgfqpoint{3.932523in}{2.000495in}}%
\pgfpathcurveto{\pgfqpoint{3.936641in}{2.004613in}}{\pgfqpoint{3.938955in}{2.010199in}}{\pgfqpoint{3.938955in}{2.016023in}}%
\pgfpathcurveto{\pgfqpoint{3.938955in}{2.021847in}}{\pgfqpoint{3.936641in}{2.027433in}}{\pgfqpoint{3.932523in}{2.031551in}}%
\pgfpathcurveto{\pgfqpoint{3.928405in}{2.035669in}}{\pgfqpoint{3.922818in}{2.037983in}}{\pgfqpoint{3.916995in}{2.037983in}}%
\pgfpathcurveto{\pgfqpoint{3.911171in}{2.037983in}}{\pgfqpoint{3.905584in}{2.035669in}}{\pgfqpoint{3.901466in}{2.031551in}}%
\pgfpathcurveto{\pgfqpoint{3.897348in}{2.027433in}}{\pgfqpoint{3.895034in}{2.021847in}}{\pgfqpoint{3.895034in}{2.016023in}}%
\pgfpathcurveto{\pgfqpoint{3.895034in}{2.010199in}}{\pgfqpoint{3.897348in}{2.004613in}}{\pgfqpoint{3.901466in}{2.000495in}}%
\pgfpathcurveto{\pgfqpoint{3.905584in}{1.996377in}}{\pgfqpoint{3.911171in}{1.994063in}}{\pgfqpoint{3.916995in}{1.994063in}}%
\pgfpathlineto{\pgfqpoint{3.916995in}{1.994063in}}%
\pgfpathclose%
\pgfusepath{fill}%
\end{pgfscope}%
\begin{pgfscope}%
\pgfpathrectangle{\pgfqpoint{0.288431in}{0.470524in}}{\pgfqpoint{4.447128in}{4.129476in}}%
\pgfusepath{clip}%
\pgfsetbuttcap%
\pgfsetroundjoin%
\definecolor{currentfill}{rgb}{0.000000,0.000000,1.000000}%
\pgfsetfillcolor{currentfill}%
\pgfsetlinewidth{0.000000pt}%
\definecolor{currentstroke}{rgb}{0.000000,0.000000,0.000000}%
\pgfsetstrokecolor{currentstroke}%
\pgfsetdash{}{0pt}%
\pgfpathmoveto{\pgfqpoint{3.916995in}{3.032541in}}%
\pgfpathcurveto{\pgfqpoint{3.922818in}{3.032541in}}{\pgfqpoint{3.928405in}{3.034854in}}{\pgfqpoint{3.932523in}{3.038973in}}%
\pgfpathcurveto{\pgfqpoint{3.936641in}{3.043091in}}{\pgfqpoint{3.938955in}{3.048677in}}{\pgfqpoint{3.938955in}{3.054501in}}%
\pgfpathcurveto{\pgfqpoint{3.938955in}{3.060325in}}{\pgfqpoint{3.936641in}{3.065911in}}{\pgfqpoint{3.932523in}{3.070029in}}%
\pgfpathcurveto{\pgfqpoint{3.928405in}{3.074147in}}{\pgfqpoint{3.922818in}{3.076461in}}{\pgfqpoint{3.916995in}{3.076461in}}%
\pgfpathcurveto{\pgfqpoint{3.911171in}{3.076461in}}{\pgfqpoint{3.905584in}{3.074147in}}{\pgfqpoint{3.901466in}{3.070029in}}%
\pgfpathcurveto{\pgfqpoint{3.897348in}{3.065911in}}{\pgfqpoint{3.895034in}{3.060325in}}{\pgfqpoint{3.895034in}{3.054501in}}%
\pgfpathcurveto{\pgfqpoint{3.895034in}{3.048677in}}{\pgfqpoint{3.897348in}{3.043091in}}{\pgfqpoint{3.901466in}{3.038973in}}%
\pgfpathcurveto{\pgfqpoint{3.905584in}{3.034854in}}{\pgfqpoint{3.911171in}{3.032541in}}{\pgfqpoint{3.916995in}{3.032541in}}%
\pgfpathlineto{\pgfqpoint{3.916995in}{3.032541in}}%
\pgfpathclose%
\pgfusepath{fill}%
\end{pgfscope}%
\begin{pgfscope}%
\pgfpathrectangle{\pgfqpoint{0.288431in}{0.470524in}}{\pgfqpoint{4.447128in}{4.129476in}}%
\pgfusepath{clip}%
\pgfsetbuttcap%
\pgfsetroundjoin%
\definecolor{currentfill}{rgb}{0.000000,0.000000,1.000000}%
\pgfsetfillcolor{currentfill}%
\pgfsetlinewidth{0.000000pt}%
\definecolor{currentstroke}{rgb}{0.000000,0.000000,0.000000}%
\pgfsetstrokecolor{currentstroke}%
\pgfsetdash{}{0pt}%
\pgfpathmoveto{\pgfqpoint{4.100255in}{1.505367in}}%
\pgfpathcurveto{\pgfqpoint{4.106079in}{1.505367in}}{\pgfqpoint{4.111665in}{1.507681in}}{\pgfqpoint{4.115784in}{1.511799in}}%
\pgfpathcurveto{\pgfqpoint{4.119902in}{1.515918in}}{\pgfqpoint{4.122216in}{1.521504in}}{\pgfqpoint{4.122216in}{1.527328in}}%
\pgfpathcurveto{\pgfqpoint{4.122216in}{1.533152in}}{\pgfqpoint{4.119902in}{1.538738in}}{\pgfqpoint{4.115784in}{1.542856in}}%
\pgfpathcurveto{\pgfqpoint{4.111665in}{1.546974in}}{\pgfqpoint{4.106079in}{1.549288in}}{\pgfqpoint{4.100255in}{1.549288in}}%
\pgfpathcurveto{\pgfqpoint{4.094431in}{1.549288in}}{\pgfqpoint{4.088845in}{1.546974in}}{\pgfqpoint{4.084727in}{1.542856in}}%
\pgfpathcurveto{\pgfqpoint{4.080609in}{1.538738in}}{\pgfqpoint{4.078295in}{1.533152in}}{\pgfqpoint{4.078295in}{1.527328in}}%
\pgfpathcurveto{\pgfqpoint{4.078295in}{1.521504in}}{\pgfqpoint{4.080609in}{1.515918in}}{\pgfqpoint{4.084727in}{1.511799in}}%
\pgfpathcurveto{\pgfqpoint{4.088845in}{1.507681in}}{\pgfqpoint{4.094431in}{1.505367in}}{\pgfqpoint{4.100255in}{1.505367in}}%
\pgfpathlineto{\pgfqpoint{4.100255in}{1.505367in}}%
\pgfpathclose%
\pgfusepath{fill}%
\end{pgfscope}%
\begin{pgfscope}%
\pgfpathrectangle{\pgfqpoint{0.288431in}{0.470524in}}{\pgfqpoint{4.447128in}{4.129476in}}%
\pgfusepath{clip}%
\pgfsetbuttcap%
\pgfsetroundjoin%
\definecolor{currentfill}{rgb}{0.000000,0.000000,1.000000}%
\pgfsetfillcolor{currentfill}%
\pgfsetlinewidth{0.000000pt}%
\definecolor{currentstroke}{rgb}{0.000000,0.000000,0.000000}%
\pgfsetstrokecolor{currentstroke}%
\pgfsetdash{}{0pt}%
\pgfpathmoveto{\pgfqpoint{4.100255in}{3.521236in}}%
\pgfpathcurveto{\pgfqpoint{4.106079in}{3.521236in}}{\pgfqpoint{4.111665in}{3.523550in}}{\pgfqpoint{4.115784in}{3.527668in}}%
\pgfpathcurveto{\pgfqpoint{4.119902in}{3.531786in}}{\pgfqpoint{4.122216in}{3.537372in}}{\pgfqpoint{4.122216in}{3.543196in}}%
\pgfpathcurveto{\pgfqpoint{4.122216in}{3.549020in}}{\pgfqpoint{4.119902in}{3.554606in}}{\pgfqpoint{4.115784in}{3.558724in}}%
\pgfpathcurveto{\pgfqpoint{4.111665in}{3.562843in}}{\pgfqpoint{4.106079in}{3.565156in}}{\pgfqpoint{4.100255in}{3.565156in}}%
\pgfpathcurveto{\pgfqpoint{4.094431in}{3.565156in}}{\pgfqpoint{4.088845in}{3.562843in}}{\pgfqpoint{4.084727in}{3.558724in}}%
\pgfpathcurveto{\pgfqpoint{4.080609in}{3.554606in}}{\pgfqpoint{4.078295in}{3.549020in}}{\pgfqpoint{4.078295in}{3.543196in}}%
\pgfpathcurveto{\pgfqpoint{4.078295in}{3.537372in}}{\pgfqpoint{4.080609in}{3.531786in}}{\pgfqpoint{4.084727in}{3.527668in}}%
\pgfpathcurveto{\pgfqpoint{4.088845in}{3.523550in}}{\pgfqpoint{4.094431in}{3.521236in}}{\pgfqpoint{4.100255in}{3.521236in}}%
\pgfpathlineto{\pgfqpoint{4.100255in}{3.521236in}}%
\pgfpathclose%
\pgfusepath{fill}%
\end{pgfscope}%
\begin{pgfscope}%
\pgfpathrectangle{\pgfqpoint{0.288431in}{0.470524in}}{\pgfqpoint{4.447128in}{4.129476in}}%
\pgfusepath{clip}%
\pgfsetbuttcap%
\pgfsetroundjoin%
\definecolor{currentfill}{rgb}{0.000000,0.000000,1.000000}%
\pgfsetfillcolor{currentfill}%
\pgfsetlinewidth{0.000000pt}%
\definecolor{currentstroke}{rgb}{0.000000,0.000000,0.000000}%
\pgfsetstrokecolor{currentstroke}%
\pgfsetdash{}{0pt}%
\pgfpathmoveto{\pgfqpoint{4.344603in}{1.383194in}}%
\pgfpathcurveto{\pgfqpoint{4.350427in}{1.383194in}}{\pgfqpoint{4.356013in}{1.385507in}}{\pgfqpoint{4.360131in}{1.389626in}}%
\pgfpathcurveto{\pgfqpoint{4.364249in}{1.393744in}}{\pgfqpoint{4.366563in}{1.399330in}}{\pgfqpoint{4.366563in}{1.405154in}}%
\pgfpathcurveto{\pgfqpoint{4.366563in}{1.410978in}}{\pgfqpoint{4.364249in}{1.416564in}}{\pgfqpoint{4.360131in}{1.420682in}}%
\pgfpathcurveto{\pgfqpoint{4.356013in}{1.424800in}}{\pgfqpoint{4.350427in}{1.427114in}}{\pgfqpoint{4.344603in}{1.427114in}}%
\pgfpathcurveto{\pgfqpoint{4.338779in}{1.427114in}}{\pgfqpoint{4.333193in}{1.424800in}}{\pgfqpoint{4.329075in}{1.420682in}}%
\pgfpathcurveto{\pgfqpoint{4.324957in}{1.416564in}}{\pgfqpoint{4.322643in}{1.410978in}}{\pgfqpoint{4.322643in}{1.405154in}}%
\pgfpathcurveto{\pgfqpoint{4.322643in}{1.399330in}}{\pgfqpoint{4.324957in}{1.393744in}}{\pgfqpoint{4.329075in}{1.389626in}}%
\pgfpathcurveto{\pgfqpoint{4.333193in}{1.385507in}}{\pgfqpoint{4.338779in}{1.383194in}}{\pgfqpoint{4.344603in}{1.383194in}}%
\pgfpathlineto{\pgfqpoint{4.344603in}{1.383194in}}%
\pgfpathclose%
\pgfusepath{fill}%
\end{pgfscope}%
\begin{pgfscope}%
\pgfpathrectangle{\pgfqpoint{0.288431in}{0.470524in}}{\pgfqpoint{4.447128in}{4.129476in}}%
\pgfusepath{clip}%
\pgfsetbuttcap%
\pgfsetroundjoin%
\definecolor{currentfill}{rgb}{0.000000,0.000000,1.000000}%
\pgfsetfillcolor{currentfill}%
\pgfsetlinewidth{0.000000pt}%
\definecolor{currentstroke}{rgb}{0.000000,0.000000,0.000000}%
\pgfsetstrokecolor{currentstroke}%
\pgfsetdash{}{0pt}%
\pgfpathmoveto{\pgfqpoint{4.344603in}{3.643410in}}%
\pgfpathcurveto{\pgfqpoint{4.350427in}{3.643410in}}{\pgfqpoint{4.356013in}{3.645724in}}{\pgfqpoint{4.360131in}{3.649842in}}%
\pgfpathcurveto{\pgfqpoint{4.364249in}{3.653960in}}{\pgfqpoint{4.366563in}{3.659546in}}{\pgfqpoint{4.366563in}{3.665370in}}%
\pgfpathcurveto{\pgfqpoint{4.366563in}{3.671194in}}{\pgfqpoint{4.364249in}{3.676780in}}{\pgfqpoint{4.360131in}{3.680898in}}%
\pgfpathcurveto{\pgfqpoint{4.356013in}{3.685016in}}{\pgfqpoint{4.350427in}{3.687330in}}{\pgfqpoint{4.344603in}{3.687330in}}%
\pgfpathcurveto{\pgfqpoint{4.338779in}{3.687330in}}{\pgfqpoint{4.333193in}{3.685016in}}{\pgfqpoint{4.329075in}{3.680898in}}%
\pgfpathcurveto{\pgfqpoint{4.324957in}{3.676780in}}{\pgfqpoint{4.322643in}{3.671194in}}{\pgfqpoint{4.322643in}{3.665370in}}%
\pgfpathcurveto{\pgfqpoint{4.322643in}{3.659546in}}{\pgfqpoint{4.324957in}{3.653960in}}{\pgfqpoint{4.329075in}{3.649842in}}%
\pgfpathcurveto{\pgfqpoint{4.333193in}{3.645724in}}{\pgfqpoint{4.338779in}{3.643410in}}{\pgfqpoint{4.344603in}{3.643410in}}%
\pgfpathlineto{\pgfqpoint{4.344603in}{3.643410in}}%
\pgfpathclose%
\pgfusepath{fill}%
\end{pgfscope}%
\begin{pgfscope}%
\pgfpathrectangle{\pgfqpoint{0.288431in}{0.470524in}}{\pgfqpoint{4.447128in}{4.129476in}}%
\pgfusepath{clip}%
\pgfsetbuttcap%
\pgfsetroundjoin%
\definecolor{currentfill}{rgb}{0.000000,0.000000,1.000000}%
\pgfsetfillcolor{currentfill}%
\pgfsetlinewidth{0.000000pt}%
\definecolor{currentstroke}{rgb}{0.000000,0.000000,0.000000}%
\pgfsetstrokecolor{currentstroke}%
\pgfsetdash{}{0pt}%
\pgfpathmoveto{\pgfqpoint{4.405690in}{2.360584in}}%
\pgfpathcurveto{\pgfqpoint{4.411514in}{2.360584in}}{\pgfqpoint{4.417100in}{2.362898in}}{\pgfqpoint{4.421218in}{2.367016in}}%
\pgfpathcurveto{\pgfqpoint{4.425336in}{2.371135in}}{\pgfqpoint{4.427650in}{2.376721in}}{\pgfqpoint{4.427650in}{2.382545in}}%
\pgfpathcurveto{\pgfqpoint{4.427650in}{2.388369in}}{\pgfqpoint{4.425336in}{2.393955in}}{\pgfqpoint{4.421218in}{2.398073in}}%
\pgfpathcurveto{\pgfqpoint{4.417100in}{2.402191in}}{\pgfqpoint{4.411514in}{2.404505in}}{\pgfqpoint{4.405690in}{2.404505in}}%
\pgfpathcurveto{\pgfqpoint{4.399866in}{2.404505in}}{\pgfqpoint{4.394280in}{2.402191in}}{\pgfqpoint{4.390162in}{2.398073in}}%
\pgfpathcurveto{\pgfqpoint{4.386044in}{2.393955in}}{\pgfqpoint{4.383730in}{2.388369in}}{\pgfqpoint{4.383730in}{2.382545in}}%
\pgfpathcurveto{\pgfqpoint{4.383730in}{2.376721in}}{\pgfqpoint{4.386044in}{2.371135in}}{\pgfqpoint{4.390162in}{2.367016in}}%
\pgfpathcurveto{\pgfqpoint{4.394280in}{2.362898in}}{\pgfqpoint{4.399866in}{2.360584in}}{\pgfqpoint{4.405690in}{2.360584in}}%
\pgfpathlineto{\pgfqpoint{4.405690in}{2.360584in}}%
\pgfpathclose%
\pgfusepath{fill}%
\end{pgfscope}%
\begin{pgfscope}%
\pgfpathrectangle{\pgfqpoint{0.288431in}{0.470524in}}{\pgfqpoint{4.447128in}{4.129476in}}%
\pgfusepath{clip}%
\pgfsetbuttcap%
\pgfsetroundjoin%
\definecolor{currentfill}{rgb}{0.000000,0.000000,1.000000}%
\pgfsetfillcolor{currentfill}%
\pgfsetlinewidth{0.000000pt}%
\definecolor{currentstroke}{rgb}{0.000000,0.000000,0.000000}%
\pgfsetstrokecolor{currentstroke}%
\pgfsetdash{}{0pt}%
\pgfpathmoveto{\pgfqpoint{4.405690in}{2.666019in}}%
\pgfpathcurveto{\pgfqpoint{4.411514in}{2.666019in}}{\pgfqpoint{4.417100in}{2.668333in}}{\pgfqpoint{4.421218in}{2.672451in}}%
\pgfpathcurveto{\pgfqpoint{4.425336in}{2.676569in}}{\pgfqpoint{4.427650in}{2.682155in}}{\pgfqpoint{4.427650in}{2.687979in}}%
\pgfpathcurveto{\pgfqpoint{4.427650in}{2.693803in}}{\pgfqpoint{4.425336in}{2.699389in}}{\pgfqpoint{4.421218in}{2.703508in}}%
\pgfpathcurveto{\pgfqpoint{4.417100in}{2.707626in}}{\pgfqpoint{4.411514in}{2.709940in}}{\pgfqpoint{4.405690in}{2.709940in}}%
\pgfpathcurveto{\pgfqpoint{4.399866in}{2.709940in}}{\pgfqpoint{4.394280in}{2.707626in}}{\pgfqpoint{4.390162in}{2.703508in}}%
\pgfpathcurveto{\pgfqpoint{4.386044in}{2.699389in}}{\pgfqpoint{4.383730in}{2.693803in}}{\pgfqpoint{4.383730in}{2.687979in}}%
\pgfpathcurveto{\pgfqpoint{4.383730in}{2.682155in}}{\pgfqpoint{4.386044in}{2.676569in}}{\pgfqpoint{4.390162in}{2.672451in}}%
\pgfpathcurveto{\pgfqpoint{4.394280in}{2.668333in}}{\pgfqpoint{4.399866in}{2.666019in}}{\pgfqpoint{4.405690in}{2.666019in}}%
\pgfpathlineto{\pgfqpoint{4.405690in}{2.666019in}}%
\pgfpathclose%
\pgfusepath{fill}%
\end{pgfscope}%
\begin{pgfscope}%
\pgfpathrectangle{\pgfqpoint{0.288431in}{0.470524in}}{\pgfqpoint{4.447128in}{4.129476in}}%
\pgfusepath{clip}%
\pgfsetbuttcap%
\pgfsetroundjoin%
\definecolor{currentfill}{rgb}{0.000000,0.000000,1.000000}%
\pgfsetfillcolor{currentfill}%
\pgfsetlinewidth{0.000000pt}%
\definecolor{currentstroke}{rgb}{0.000000,0.000000,0.000000}%
\pgfsetstrokecolor{currentstroke}%
\pgfsetdash{}{0pt}%
\pgfpathmoveto{\pgfqpoint{4.527864in}{0.772324in}}%
\pgfpathcurveto{\pgfqpoint{4.533688in}{0.772324in}}{\pgfqpoint{4.539274in}{0.774638in}}{\pgfqpoint{4.543392in}{0.778756in}}%
\pgfpathcurveto{\pgfqpoint{4.547510in}{0.782874in}}{\pgfqpoint{4.549824in}{0.788461in}}{\pgfqpoint{4.549824in}{0.794285in}}%
\pgfpathcurveto{\pgfqpoint{4.549824in}{0.800109in}}{\pgfqpoint{4.547510in}{0.805695in}}{\pgfqpoint{4.543392in}{0.809813in}}%
\pgfpathcurveto{\pgfqpoint{4.539274in}{0.813931in}}{\pgfqpoint{4.533688in}{0.816245in}}{\pgfqpoint{4.527864in}{0.816245in}}%
\pgfpathcurveto{\pgfqpoint{4.522040in}{0.816245in}}{\pgfqpoint{4.516454in}{0.813931in}}{\pgfqpoint{4.512336in}{0.809813in}}%
\pgfpathcurveto{\pgfqpoint{4.508217in}{0.805695in}}{\pgfqpoint{4.505904in}{0.800109in}}{\pgfqpoint{4.505904in}{0.794285in}}%
\pgfpathcurveto{\pgfqpoint{4.505904in}{0.788461in}}{\pgfqpoint{4.508217in}{0.782874in}}{\pgfqpoint{4.512336in}{0.778756in}}%
\pgfpathcurveto{\pgfqpoint{4.516454in}{0.774638in}}{\pgfqpoint{4.522040in}{0.772324in}}{\pgfqpoint{4.527864in}{0.772324in}}%
\pgfpathlineto{\pgfqpoint{4.527864in}{0.772324in}}%
\pgfpathclose%
\pgfusepath{fill}%
\end{pgfscope}%
\begin{pgfscope}%
\pgfpathrectangle{\pgfqpoint{0.288431in}{0.470524in}}{\pgfqpoint{4.447128in}{4.129476in}}%
\pgfusepath{clip}%
\pgfsetbuttcap%
\pgfsetroundjoin%
\definecolor{currentfill}{rgb}{0.000000,0.000000,1.000000}%
\pgfsetfillcolor{currentfill}%
\pgfsetlinewidth{0.000000pt}%
\definecolor{currentstroke}{rgb}{0.000000,0.000000,0.000000}%
\pgfsetstrokecolor{currentstroke}%
\pgfsetdash{}{0pt}%
\pgfpathmoveto{\pgfqpoint{4.527864in}{4.254279in}}%
\pgfpathcurveto{\pgfqpoint{4.533688in}{4.254279in}}{\pgfqpoint{4.539274in}{4.256593in}}{\pgfqpoint{4.543392in}{4.260711in}}%
\pgfpathcurveto{\pgfqpoint{4.547510in}{4.264829in}}{\pgfqpoint{4.549824in}{4.270415in}}{\pgfqpoint{4.549824in}{4.276239in}}%
\pgfpathcurveto{\pgfqpoint{4.549824in}{4.282063in}}{\pgfqpoint{4.547510in}{4.287649in}}{\pgfqpoint{4.543392in}{4.291768in}}%
\pgfpathcurveto{\pgfqpoint{4.539274in}{4.295886in}}{\pgfqpoint{4.533688in}{4.298200in}}{\pgfqpoint{4.527864in}{4.298200in}}%
\pgfpathcurveto{\pgfqpoint{4.522040in}{4.298200in}}{\pgfqpoint{4.516454in}{4.295886in}}{\pgfqpoint{4.512336in}{4.291768in}}%
\pgfpathcurveto{\pgfqpoint{4.508217in}{4.287649in}}{\pgfqpoint{4.505904in}{4.282063in}}{\pgfqpoint{4.505904in}{4.276239in}}%
\pgfpathcurveto{\pgfqpoint{4.505904in}{4.270415in}}{\pgfqpoint{4.508217in}{4.264829in}}{\pgfqpoint{4.512336in}{4.260711in}}%
\pgfpathcurveto{\pgfqpoint{4.516454in}{4.256593in}}{\pgfqpoint{4.522040in}{4.254279in}}{\pgfqpoint{4.527864in}{4.254279in}}%
\pgfpathlineto{\pgfqpoint{4.527864in}{4.254279in}}%
\pgfpathclose%
\pgfusepath{fill}%
\end{pgfscope}%
\begin{pgfscope}%
\pgfpathrectangle{\pgfqpoint{0.288431in}{0.470524in}}{\pgfqpoint{4.447128in}{4.129476in}}%
\pgfusepath{clip}%
\pgfsetbuttcap%
\pgfsetroundjoin%
\definecolor{currentfill}{rgb}{0.000000,0.000000,1.000000}%
\pgfsetfillcolor{currentfill}%
\pgfsetlinewidth{0.000000pt}%
\definecolor{currentstroke}{rgb}{0.000000,0.000000,0.000000}%
\pgfsetstrokecolor{currentstroke}%
\pgfsetdash{}{0pt}%
\pgfpathmoveto{\pgfqpoint{4.650038in}{1.077759in}}%
\pgfpathcurveto{\pgfqpoint{4.655862in}{1.077759in}}{\pgfqpoint{4.661448in}{1.080073in}}{\pgfqpoint{4.665566in}{1.084191in}}%
\pgfpathcurveto{\pgfqpoint{4.669684in}{1.088309in}}{\pgfqpoint{4.671998in}{1.093895in}}{\pgfqpoint{4.671998in}{1.099719in}}%
\pgfpathcurveto{\pgfqpoint{4.671998in}{1.105543in}}{\pgfqpoint{4.669684in}{1.111129in}}{\pgfqpoint{4.665566in}{1.115247in}}%
\pgfpathcurveto{\pgfqpoint{4.661448in}{1.119366in}}{\pgfqpoint{4.655862in}{1.121679in}}{\pgfqpoint{4.650038in}{1.121679in}}%
\pgfpathcurveto{\pgfqpoint{4.644214in}{1.121679in}}{\pgfqpoint{4.638628in}{1.119366in}}{\pgfqpoint{4.634509in}{1.115247in}}%
\pgfpathcurveto{\pgfqpoint{4.630391in}{1.111129in}}{\pgfqpoint{4.628077in}{1.105543in}}{\pgfqpoint{4.628077in}{1.099719in}}%
\pgfpathcurveto{\pgfqpoint{4.628077in}{1.093895in}}{\pgfqpoint{4.630391in}{1.088309in}}{\pgfqpoint{4.634509in}{1.084191in}}%
\pgfpathcurveto{\pgfqpoint{4.638628in}{1.080073in}}{\pgfqpoint{4.644214in}{1.077759in}}{\pgfqpoint{4.650038in}{1.077759in}}%
\pgfpathlineto{\pgfqpoint{4.650038in}{1.077759in}}%
\pgfpathclose%
\pgfusepath{fill}%
\end{pgfscope}%
\begin{pgfscope}%
\pgfpathrectangle{\pgfqpoint{0.288431in}{0.470524in}}{\pgfqpoint{4.447128in}{4.129476in}}%
\pgfusepath{clip}%
\pgfsetbuttcap%
\pgfsetroundjoin%
\definecolor{currentfill}{rgb}{0.000000,0.000000,1.000000}%
\pgfsetfillcolor{currentfill}%
\pgfsetlinewidth{0.000000pt}%
\definecolor{currentstroke}{rgb}{0.000000,0.000000,0.000000}%
\pgfsetstrokecolor{currentstroke}%
\pgfsetdash{}{0pt}%
\pgfpathmoveto{\pgfqpoint{4.650038in}{3.948844in}}%
\pgfpathcurveto{\pgfqpoint{4.655862in}{3.948844in}}{\pgfqpoint{4.661448in}{3.951158in}}{\pgfqpoint{4.665566in}{3.955276in}}%
\pgfpathcurveto{\pgfqpoint{4.669684in}{3.959395in}}{\pgfqpoint{4.671998in}{3.964981in}}{\pgfqpoint{4.671998in}{3.970805in}}%
\pgfpathcurveto{\pgfqpoint{4.671998in}{3.976629in}}{\pgfqpoint{4.669684in}{3.982215in}}{\pgfqpoint{4.665566in}{3.986333in}}%
\pgfpathcurveto{\pgfqpoint{4.661448in}{3.990451in}}{\pgfqpoint{4.655862in}{3.992765in}}{\pgfqpoint{4.650038in}{3.992765in}}%
\pgfpathcurveto{\pgfqpoint{4.644214in}{3.992765in}}{\pgfqpoint{4.638628in}{3.990451in}}{\pgfqpoint{4.634509in}{3.986333in}}%
\pgfpathcurveto{\pgfqpoint{4.630391in}{3.982215in}}{\pgfqpoint{4.628077in}{3.976629in}}{\pgfqpoint{4.628077in}{3.970805in}}%
\pgfpathcurveto{\pgfqpoint{4.628077in}{3.964981in}}{\pgfqpoint{4.630391in}{3.959395in}}{\pgfqpoint{4.634509in}{3.955276in}}%
\pgfpathcurveto{\pgfqpoint{4.638628in}{3.951158in}}{\pgfqpoint{4.644214in}{3.948844in}}{\pgfqpoint{4.650038in}{3.948844in}}%
\pgfpathlineto{\pgfqpoint{4.650038in}{3.948844in}}%
\pgfpathclose%
\pgfusepath{fill}%
\end{pgfscope}%
\begin{pgfscope}%
\pgfsetbuttcap%
\pgfsetroundjoin%
\definecolor{currentfill}{rgb}{0.000000,0.000000,0.000000}%
\pgfsetfillcolor{currentfill}%
\pgfsetlinewidth{0.803000pt}%
\definecolor{currentstroke}{rgb}{0.000000,0.000000,0.000000}%
\pgfsetstrokecolor{currentstroke}%
\pgfsetdash{}{0pt}%
\pgfsys@defobject{currentmarker}{\pgfqpoint{0.000000in}{-0.048611in}}{\pgfqpoint{0.000000in}{0.000000in}}{%
\pgfpathmoveto{\pgfqpoint{0.000000in}{0.000000in}}%
\pgfpathlineto{\pgfqpoint{0.000000in}{-0.048611in}}%
\pgfusepath{stroke,fill}%
}%
\begin{pgfscope}%
\pgfsys@transformshift{0.373953in}{0.331635in}%
\pgfsys@useobject{currentmarker}{}%
\end{pgfscope}%
\end{pgfscope}%
\begin{pgfscope}%
\definecolor{textcolor}{rgb}{0.000000,0.000000,0.000000}%
\pgfsetstrokecolor{textcolor}%
\pgfsetfillcolor{textcolor}%
\pgftext[x=0.373953in,y=0.234413in,,top]{\color{textcolor}\sffamily\fontsize{10.000000}{12.000000}\selectfont 0}%
\end{pgfscope}%
\begin{pgfscope}%
\pgfsetbuttcap%
\pgfsetroundjoin%
\definecolor{currentfill}{rgb}{0.000000,0.000000,0.000000}%
\pgfsetfillcolor{currentfill}%
\pgfsetlinewidth{0.803000pt}%
\definecolor{currentstroke}{rgb}{0.000000,0.000000,0.000000}%
\pgfsetstrokecolor{currentstroke}%
\pgfsetdash{}{0pt}%
\pgfsys@defobject{currentmarker}{\pgfqpoint{0.000000in}{-0.048611in}}{\pgfqpoint{0.000000in}{0.000000in}}{%
\pgfpathmoveto{\pgfqpoint{0.000000in}{0.000000in}}%
\pgfpathlineto{\pgfqpoint{0.000000in}{-0.048611in}}%
\pgfusepath{stroke,fill}%
}%
\begin{pgfscope}%
\pgfsys@transformshift{0.984822in}{0.331635in}%
\pgfsys@useobject{currentmarker}{}%
\end{pgfscope}%
\end{pgfscope}%
\begin{pgfscope}%
\definecolor{textcolor}{rgb}{0.000000,0.000000,0.000000}%
\pgfsetstrokecolor{textcolor}%
\pgfsetfillcolor{textcolor}%
\pgftext[x=0.984822in,y=0.234413in,,top]{\color{textcolor}\sffamily\fontsize{10.000000}{12.000000}\selectfont 10}%
\end{pgfscope}%
\begin{pgfscope}%
\pgfsetbuttcap%
\pgfsetroundjoin%
\definecolor{currentfill}{rgb}{0.000000,0.000000,0.000000}%
\pgfsetfillcolor{currentfill}%
\pgfsetlinewidth{0.803000pt}%
\definecolor{currentstroke}{rgb}{0.000000,0.000000,0.000000}%
\pgfsetstrokecolor{currentstroke}%
\pgfsetdash{}{0pt}%
\pgfsys@defobject{currentmarker}{\pgfqpoint{0.000000in}{-0.048611in}}{\pgfqpoint{0.000000in}{0.000000in}}{%
\pgfpathmoveto{\pgfqpoint{0.000000in}{0.000000in}}%
\pgfpathlineto{\pgfqpoint{0.000000in}{-0.048611in}}%
\pgfusepath{stroke,fill}%
}%
\begin{pgfscope}%
\pgfsys@transformshift{1.595691in}{0.331635in}%
\pgfsys@useobject{currentmarker}{}%
\end{pgfscope}%
\end{pgfscope}%
\begin{pgfscope}%
\definecolor{textcolor}{rgb}{0.000000,0.000000,0.000000}%
\pgfsetstrokecolor{textcolor}%
\pgfsetfillcolor{textcolor}%
\pgftext[x=1.595691in,y=0.234413in,,top]{\color{textcolor}\sffamily\fontsize{10.000000}{12.000000}\selectfont 20}%
\end{pgfscope}%
\begin{pgfscope}%
\pgfsetbuttcap%
\pgfsetroundjoin%
\definecolor{currentfill}{rgb}{0.000000,0.000000,0.000000}%
\pgfsetfillcolor{currentfill}%
\pgfsetlinewidth{0.803000pt}%
\definecolor{currentstroke}{rgb}{0.000000,0.000000,0.000000}%
\pgfsetstrokecolor{currentstroke}%
\pgfsetdash{}{0pt}%
\pgfsys@defobject{currentmarker}{\pgfqpoint{0.000000in}{-0.048611in}}{\pgfqpoint{0.000000in}{0.000000in}}{%
\pgfpathmoveto{\pgfqpoint{0.000000in}{0.000000in}}%
\pgfpathlineto{\pgfqpoint{0.000000in}{-0.048611in}}%
\pgfusepath{stroke,fill}%
}%
\begin{pgfscope}%
\pgfsys@transformshift{2.206561in}{0.331635in}%
\pgfsys@useobject{currentmarker}{}%
\end{pgfscope}%
\end{pgfscope}%
\begin{pgfscope}%
\definecolor{textcolor}{rgb}{0.000000,0.000000,0.000000}%
\pgfsetstrokecolor{textcolor}%
\pgfsetfillcolor{textcolor}%
\pgftext[x=2.206561in,y=0.234413in,,top]{\color{textcolor}\sffamily\fontsize{10.000000}{12.000000}\selectfont 30}%
\end{pgfscope}%
\begin{pgfscope}%
\pgfsetbuttcap%
\pgfsetroundjoin%
\definecolor{currentfill}{rgb}{0.000000,0.000000,0.000000}%
\pgfsetfillcolor{currentfill}%
\pgfsetlinewidth{0.803000pt}%
\definecolor{currentstroke}{rgb}{0.000000,0.000000,0.000000}%
\pgfsetstrokecolor{currentstroke}%
\pgfsetdash{}{0pt}%
\pgfsys@defobject{currentmarker}{\pgfqpoint{0.000000in}{-0.048611in}}{\pgfqpoint{0.000000in}{0.000000in}}{%
\pgfpathmoveto{\pgfqpoint{0.000000in}{0.000000in}}%
\pgfpathlineto{\pgfqpoint{0.000000in}{-0.048611in}}%
\pgfusepath{stroke,fill}%
}%
\begin{pgfscope}%
\pgfsys@transformshift{2.817430in}{0.331635in}%
\pgfsys@useobject{currentmarker}{}%
\end{pgfscope}%
\end{pgfscope}%
\begin{pgfscope}%
\definecolor{textcolor}{rgb}{0.000000,0.000000,0.000000}%
\pgfsetstrokecolor{textcolor}%
\pgfsetfillcolor{textcolor}%
\pgftext[x=2.817430in,y=0.234413in,,top]{\color{textcolor}\sffamily\fontsize{10.000000}{12.000000}\selectfont 40}%
\end{pgfscope}%
\begin{pgfscope}%
\pgfsetbuttcap%
\pgfsetroundjoin%
\definecolor{currentfill}{rgb}{0.000000,0.000000,0.000000}%
\pgfsetfillcolor{currentfill}%
\pgfsetlinewidth{0.803000pt}%
\definecolor{currentstroke}{rgb}{0.000000,0.000000,0.000000}%
\pgfsetstrokecolor{currentstroke}%
\pgfsetdash{}{0pt}%
\pgfsys@defobject{currentmarker}{\pgfqpoint{0.000000in}{-0.048611in}}{\pgfqpoint{0.000000in}{0.000000in}}{%
\pgfpathmoveto{\pgfqpoint{0.000000in}{0.000000in}}%
\pgfpathlineto{\pgfqpoint{0.000000in}{-0.048611in}}%
\pgfusepath{stroke,fill}%
}%
\begin{pgfscope}%
\pgfsys@transformshift{3.428299in}{0.331635in}%
\pgfsys@useobject{currentmarker}{}%
\end{pgfscope}%
\end{pgfscope}%
\begin{pgfscope}%
\definecolor{textcolor}{rgb}{0.000000,0.000000,0.000000}%
\pgfsetstrokecolor{textcolor}%
\pgfsetfillcolor{textcolor}%
\pgftext[x=3.428299in,y=0.234413in,,top]{\color{textcolor}\sffamily\fontsize{10.000000}{12.000000}\selectfont 50}%
\end{pgfscope}%
\begin{pgfscope}%
\pgfsetbuttcap%
\pgfsetroundjoin%
\definecolor{currentfill}{rgb}{0.000000,0.000000,0.000000}%
\pgfsetfillcolor{currentfill}%
\pgfsetlinewidth{0.803000pt}%
\definecolor{currentstroke}{rgb}{0.000000,0.000000,0.000000}%
\pgfsetstrokecolor{currentstroke}%
\pgfsetdash{}{0pt}%
\pgfsys@defobject{currentmarker}{\pgfqpoint{0.000000in}{-0.048611in}}{\pgfqpoint{0.000000in}{0.000000in}}{%
\pgfpathmoveto{\pgfqpoint{0.000000in}{0.000000in}}%
\pgfpathlineto{\pgfqpoint{0.000000in}{-0.048611in}}%
\pgfusepath{stroke,fill}%
}%
\begin{pgfscope}%
\pgfsys@transformshift{4.039168in}{0.331635in}%
\pgfsys@useobject{currentmarker}{}%
\end{pgfscope}%
\end{pgfscope}%
\begin{pgfscope}%
\definecolor{textcolor}{rgb}{0.000000,0.000000,0.000000}%
\pgfsetstrokecolor{textcolor}%
\pgfsetfillcolor{textcolor}%
\pgftext[x=4.039168in,y=0.234413in,,top]{\color{textcolor}\sffamily\fontsize{10.000000}{12.000000}\selectfont 60}%
\end{pgfscope}%
\begin{pgfscope}%
\pgfsetbuttcap%
\pgfsetroundjoin%
\definecolor{currentfill}{rgb}{0.000000,0.000000,0.000000}%
\pgfsetfillcolor{currentfill}%
\pgfsetlinewidth{0.803000pt}%
\definecolor{currentstroke}{rgb}{0.000000,0.000000,0.000000}%
\pgfsetstrokecolor{currentstroke}%
\pgfsetdash{}{0pt}%
\pgfsys@defobject{currentmarker}{\pgfqpoint{0.000000in}{-0.048611in}}{\pgfqpoint{0.000000in}{0.000000in}}{%
\pgfpathmoveto{\pgfqpoint{0.000000in}{0.000000in}}%
\pgfpathlineto{\pgfqpoint{0.000000in}{-0.048611in}}%
\pgfusepath{stroke,fill}%
}%
\begin{pgfscope}%
\pgfsys@transformshift{4.650038in}{0.331635in}%
\pgfsys@useobject{currentmarker}{}%
\end{pgfscope}%
\end{pgfscope}%
\begin{pgfscope}%
\definecolor{textcolor}{rgb}{0.000000,0.000000,0.000000}%
\pgfsetstrokecolor{textcolor}%
\pgfsetfillcolor{textcolor}%
\pgftext[x=4.650038in,y=0.234413in,,top]{\color{textcolor}\sffamily\fontsize{10.000000}{12.000000}\selectfont 70}%
\end{pgfscope}%
\begin{pgfscope}%
\pgfsetbuttcap%
\pgfsetroundjoin%
\definecolor{currentfill}{rgb}{0.000000,0.000000,0.000000}%
\pgfsetfillcolor{currentfill}%
\pgfsetlinewidth{0.602250pt}%
\definecolor{currentstroke}{rgb}{0.000000,0.000000,0.000000}%
\pgfsetstrokecolor{currentstroke}%
\pgfsetdash{}{0pt}%
\pgfsys@defobject{currentmarker}{\pgfqpoint{0.000000in}{-0.027778in}}{\pgfqpoint{0.000000in}{0.000000in}}{%
\pgfpathmoveto{\pgfqpoint{0.000000in}{0.000000in}}%
\pgfpathlineto{\pgfqpoint{0.000000in}{-0.027778in}}%
\pgfusepath{stroke,fill}%
}%
\begin{pgfscope}%
\pgfsys@transformshift{0.496127in}{0.331635in}%
\pgfsys@useobject{currentmarker}{}%
\end{pgfscope}%
\end{pgfscope}%
\begin{pgfscope}%
\pgfsetbuttcap%
\pgfsetroundjoin%
\definecolor{currentfill}{rgb}{0.000000,0.000000,0.000000}%
\pgfsetfillcolor{currentfill}%
\pgfsetlinewidth{0.602250pt}%
\definecolor{currentstroke}{rgb}{0.000000,0.000000,0.000000}%
\pgfsetstrokecolor{currentstroke}%
\pgfsetdash{}{0pt}%
\pgfsys@defobject{currentmarker}{\pgfqpoint{0.000000in}{-0.027778in}}{\pgfqpoint{0.000000in}{0.000000in}}{%
\pgfpathmoveto{\pgfqpoint{0.000000in}{0.000000in}}%
\pgfpathlineto{\pgfqpoint{0.000000in}{-0.027778in}}%
\pgfusepath{stroke,fill}%
}%
\begin{pgfscope}%
\pgfsys@transformshift{0.618301in}{0.331635in}%
\pgfsys@useobject{currentmarker}{}%
\end{pgfscope}%
\end{pgfscope}%
\begin{pgfscope}%
\pgfsetbuttcap%
\pgfsetroundjoin%
\definecolor{currentfill}{rgb}{0.000000,0.000000,0.000000}%
\pgfsetfillcolor{currentfill}%
\pgfsetlinewidth{0.602250pt}%
\definecolor{currentstroke}{rgb}{0.000000,0.000000,0.000000}%
\pgfsetstrokecolor{currentstroke}%
\pgfsetdash{}{0pt}%
\pgfsys@defobject{currentmarker}{\pgfqpoint{0.000000in}{-0.027778in}}{\pgfqpoint{0.000000in}{0.000000in}}{%
\pgfpathmoveto{\pgfqpoint{0.000000in}{0.000000in}}%
\pgfpathlineto{\pgfqpoint{0.000000in}{-0.027778in}}%
\pgfusepath{stroke,fill}%
}%
\begin{pgfscope}%
\pgfsys@transformshift{0.740474in}{0.331635in}%
\pgfsys@useobject{currentmarker}{}%
\end{pgfscope}%
\end{pgfscope}%
\begin{pgfscope}%
\pgfsetbuttcap%
\pgfsetroundjoin%
\definecolor{currentfill}{rgb}{0.000000,0.000000,0.000000}%
\pgfsetfillcolor{currentfill}%
\pgfsetlinewidth{0.602250pt}%
\definecolor{currentstroke}{rgb}{0.000000,0.000000,0.000000}%
\pgfsetstrokecolor{currentstroke}%
\pgfsetdash{}{0pt}%
\pgfsys@defobject{currentmarker}{\pgfqpoint{0.000000in}{-0.027778in}}{\pgfqpoint{0.000000in}{0.000000in}}{%
\pgfpathmoveto{\pgfqpoint{0.000000in}{0.000000in}}%
\pgfpathlineto{\pgfqpoint{0.000000in}{-0.027778in}}%
\pgfusepath{stroke,fill}%
}%
\begin{pgfscope}%
\pgfsys@transformshift{0.862648in}{0.331635in}%
\pgfsys@useobject{currentmarker}{}%
\end{pgfscope}%
\end{pgfscope}%
\begin{pgfscope}%
\pgfsetbuttcap%
\pgfsetroundjoin%
\definecolor{currentfill}{rgb}{0.000000,0.000000,0.000000}%
\pgfsetfillcolor{currentfill}%
\pgfsetlinewidth{0.602250pt}%
\definecolor{currentstroke}{rgb}{0.000000,0.000000,0.000000}%
\pgfsetstrokecolor{currentstroke}%
\pgfsetdash{}{0pt}%
\pgfsys@defobject{currentmarker}{\pgfqpoint{0.000000in}{-0.027778in}}{\pgfqpoint{0.000000in}{0.000000in}}{%
\pgfpathmoveto{\pgfqpoint{0.000000in}{0.000000in}}%
\pgfpathlineto{\pgfqpoint{0.000000in}{-0.027778in}}%
\pgfusepath{stroke,fill}%
}%
\begin{pgfscope}%
\pgfsys@transformshift{1.106996in}{0.331635in}%
\pgfsys@useobject{currentmarker}{}%
\end{pgfscope}%
\end{pgfscope}%
\begin{pgfscope}%
\pgfsetbuttcap%
\pgfsetroundjoin%
\definecolor{currentfill}{rgb}{0.000000,0.000000,0.000000}%
\pgfsetfillcolor{currentfill}%
\pgfsetlinewidth{0.602250pt}%
\definecolor{currentstroke}{rgb}{0.000000,0.000000,0.000000}%
\pgfsetstrokecolor{currentstroke}%
\pgfsetdash{}{0pt}%
\pgfsys@defobject{currentmarker}{\pgfqpoint{0.000000in}{-0.027778in}}{\pgfqpoint{0.000000in}{0.000000in}}{%
\pgfpathmoveto{\pgfqpoint{0.000000in}{0.000000in}}%
\pgfpathlineto{\pgfqpoint{0.000000in}{-0.027778in}}%
\pgfusepath{stroke,fill}%
}%
\begin{pgfscope}%
\pgfsys@transformshift{1.229170in}{0.331635in}%
\pgfsys@useobject{currentmarker}{}%
\end{pgfscope}%
\end{pgfscope}%
\begin{pgfscope}%
\pgfsetbuttcap%
\pgfsetroundjoin%
\definecolor{currentfill}{rgb}{0.000000,0.000000,0.000000}%
\pgfsetfillcolor{currentfill}%
\pgfsetlinewidth{0.602250pt}%
\definecolor{currentstroke}{rgb}{0.000000,0.000000,0.000000}%
\pgfsetstrokecolor{currentstroke}%
\pgfsetdash{}{0pt}%
\pgfsys@defobject{currentmarker}{\pgfqpoint{0.000000in}{-0.027778in}}{\pgfqpoint{0.000000in}{0.000000in}}{%
\pgfpathmoveto{\pgfqpoint{0.000000in}{0.000000in}}%
\pgfpathlineto{\pgfqpoint{0.000000in}{-0.027778in}}%
\pgfusepath{stroke,fill}%
}%
\begin{pgfscope}%
\pgfsys@transformshift{1.351344in}{0.331635in}%
\pgfsys@useobject{currentmarker}{}%
\end{pgfscope}%
\end{pgfscope}%
\begin{pgfscope}%
\pgfsetbuttcap%
\pgfsetroundjoin%
\definecolor{currentfill}{rgb}{0.000000,0.000000,0.000000}%
\pgfsetfillcolor{currentfill}%
\pgfsetlinewidth{0.602250pt}%
\definecolor{currentstroke}{rgb}{0.000000,0.000000,0.000000}%
\pgfsetstrokecolor{currentstroke}%
\pgfsetdash{}{0pt}%
\pgfsys@defobject{currentmarker}{\pgfqpoint{0.000000in}{-0.027778in}}{\pgfqpoint{0.000000in}{0.000000in}}{%
\pgfpathmoveto{\pgfqpoint{0.000000in}{0.000000in}}%
\pgfpathlineto{\pgfqpoint{0.000000in}{-0.027778in}}%
\pgfusepath{stroke,fill}%
}%
\begin{pgfscope}%
\pgfsys@transformshift{1.473518in}{0.331635in}%
\pgfsys@useobject{currentmarker}{}%
\end{pgfscope}%
\end{pgfscope}%
\begin{pgfscope}%
\pgfsetbuttcap%
\pgfsetroundjoin%
\definecolor{currentfill}{rgb}{0.000000,0.000000,0.000000}%
\pgfsetfillcolor{currentfill}%
\pgfsetlinewidth{0.602250pt}%
\definecolor{currentstroke}{rgb}{0.000000,0.000000,0.000000}%
\pgfsetstrokecolor{currentstroke}%
\pgfsetdash{}{0pt}%
\pgfsys@defobject{currentmarker}{\pgfqpoint{0.000000in}{-0.027778in}}{\pgfqpoint{0.000000in}{0.000000in}}{%
\pgfpathmoveto{\pgfqpoint{0.000000in}{0.000000in}}%
\pgfpathlineto{\pgfqpoint{0.000000in}{-0.027778in}}%
\pgfusepath{stroke,fill}%
}%
\begin{pgfscope}%
\pgfsys@transformshift{1.717865in}{0.331635in}%
\pgfsys@useobject{currentmarker}{}%
\end{pgfscope}%
\end{pgfscope}%
\begin{pgfscope}%
\pgfsetbuttcap%
\pgfsetroundjoin%
\definecolor{currentfill}{rgb}{0.000000,0.000000,0.000000}%
\pgfsetfillcolor{currentfill}%
\pgfsetlinewidth{0.602250pt}%
\definecolor{currentstroke}{rgb}{0.000000,0.000000,0.000000}%
\pgfsetstrokecolor{currentstroke}%
\pgfsetdash{}{0pt}%
\pgfsys@defobject{currentmarker}{\pgfqpoint{0.000000in}{-0.027778in}}{\pgfqpoint{0.000000in}{0.000000in}}{%
\pgfpathmoveto{\pgfqpoint{0.000000in}{0.000000in}}%
\pgfpathlineto{\pgfqpoint{0.000000in}{-0.027778in}}%
\pgfusepath{stroke,fill}%
}%
\begin{pgfscope}%
\pgfsys@transformshift{1.840039in}{0.331635in}%
\pgfsys@useobject{currentmarker}{}%
\end{pgfscope}%
\end{pgfscope}%
\begin{pgfscope}%
\pgfsetbuttcap%
\pgfsetroundjoin%
\definecolor{currentfill}{rgb}{0.000000,0.000000,0.000000}%
\pgfsetfillcolor{currentfill}%
\pgfsetlinewidth{0.602250pt}%
\definecolor{currentstroke}{rgb}{0.000000,0.000000,0.000000}%
\pgfsetstrokecolor{currentstroke}%
\pgfsetdash{}{0pt}%
\pgfsys@defobject{currentmarker}{\pgfqpoint{0.000000in}{-0.027778in}}{\pgfqpoint{0.000000in}{0.000000in}}{%
\pgfpathmoveto{\pgfqpoint{0.000000in}{0.000000in}}%
\pgfpathlineto{\pgfqpoint{0.000000in}{-0.027778in}}%
\pgfusepath{stroke,fill}%
}%
\begin{pgfscope}%
\pgfsys@transformshift{1.962213in}{0.331635in}%
\pgfsys@useobject{currentmarker}{}%
\end{pgfscope}%
\end{pgfscope}%
\begin{pgfscope}%
\pgfsetbuttcap%
\pgfsetroundjoin%
\definecolor{currentfill}{rgb}{0.000000,0.000000,0.000000}%
\pgfsetfillcolor{currentfill}%
\pgfsetlinewidth{0.602250pt}%
\definecolor{currentstroke}{rgb}{0.000000,0.000000,0.000000}%
\pgfsetstrokecolor{currentstroke}%
\pgfsetdash{}{0pt}%
\pgfsys@defobject{currentmarker}{\pgfqpoint{0.000000in}{-0.027778in}}{\pgfqpoint{0.000000in}{0.000000in}}{%
\pgfpathmoveto{\pgfqpoint{0.000000in}{0.000000in}}%
\pgfpathlineto{\pgfqpoint{0.000000in}{-0.027778in}}%
\pgfusepath{stroke,fill}%
}%
\begin{pgfscope}%
\pgfsys@transformshift{2.084387in}{0.331635in}%
\pgfsys@useobject{currentmarker}{}%
\end{pgfscope}%
\end{pgfscope}%
\begin{pgfscope}%
\pgfsetbuttcap%
\pgfsetroundjoin%
\definecolor{currentfill}{rgb}{0.000000,0.000000,0.000000}%
\pgfsetfillcolor{currentfill}%
\pgfsetlinewidth{0.602250pt}%
\definecolor{currentstroke}{rgb}{0.000000,0.000000,0.000000}%
\pgfsetstrokecolor{currentstroke}%
\pgfsetdash{}{0pt}%
\pgfsys@defobject{currentmarker}{\pgfqpoint{0.000000in}{-0.027778in}}{\pgfqpoint{0.000000in}{0.000000in}}{%
\pgfpathmoveto{\pgfqpoint{0.000000in}{0.000000in}}%
\pgfpathlineto{\pgfqpoint{0.000000in}{-0.027778in}}%
\pgfusepath{stroke,fill}%
}%
\begin{pgfscope}%
\pgfsys@transformshift{2.328735in}{0.331635in}%
\pgfsys@useobject{currentmarker}{}%
\end{pgfscope}%
\end{pgfscope}%
\begin{pgfscope}%
\pgfsetbuttcap%
\pgfsetroundjoin%
\definecolor{currentfill}{rgb}{0.000000,0.000000,0.000000}%
\pgfsetfillcolor{currentfill}%
\pgfsetlinewidth{0.602250pt}%
\definecolor{currentstroke}{rgb}{0.000000,0.000000,0.000000}%
\pgfsetstrokecolor{currentstroke}%
\pgfsetdash{}{0pt}%
\pgfsys@defobject{currentmarker}{\pgfqpoint{0.000000in}{-0.027778in}}{\pgfqpoint{0.000000in}{0.000000in}}{%
\pgfpathmoveto{\pgfqpoint{0.000000in}{0.000000in}}%
\pgfpathlineto{\pgfqpoint{0.000000in}{-0.027778in}}%
\pgfusepath{stroke,fill}%
}%
\begin{pgfscope}%
\pgfsys@transformshift{2.450908in}{0.331635in}%
\pgfsys@useobject{currentmarker}{}%
\end{pgfscope}%
\end{pgfscope}%
\begin{pgfscope}%
\pgfsetbuttcap%
\pgfsetroundjoin%
\definecolor{currentfill}{rgb}{0.000000,0.000000,0.000000}%
\pgfsetfillcolor{currentfill}%
\pgfsetlinewidth{0.602250pt}%
\definecolor{currentstroke}{rgb}{0.000000,0.000000,0.000000}%
\pgfsetstrokecolor{currentstroke}%
\pgfsetdash{}{0pt}%
\pgfsys@defobject{currentmarker}{\pgfqpoint{0.000000in}{-0.027778in}}{\pgfqpoint{0.000000in}{0.000000in}}{%
\pgfpathmoveto{\pgfqpoint{0.000000in}{0.000000in}}%
\pgfpathlineto{\pgfqpoint{0.000000in}{-0.027778in}}%
\pgfusepath{stroke,fill}%
}%
\begin{pgfscope}%
\pgfsys@transformshift{2.573082in}{0.331635in}%
\pgfsys@useobject{currentmarker}{}%
\end{pgfscope}%
\end{pgfscope}%
\begin{pgfscope}%
\pgfsetbuttcap%
\pgfsetroundjoin%
\definecolor{currentfill}{rgb}{0.000000,0.000000,0.000000}%
\pgfsetfillcolor{currentfill}%
\pgfsetlinewidth{0.602250pt}%
\definecolor{currentstroke}{rgb}{0.000000,0.000000,0.000000}%
\pgfsetstrokecolor{currentstroke}%
\pgfsetdash{}{0pt}%
\pgfsys@defobject{currentmarker}{\pgfqpoint{0.000000in}{-0.027778in}}{\pgfqpoint{0.000000in}{0.000000in}}{%
\pgfpathmoveto{\pgfqpoint{0.000000in}{0.000000in}}%
\pgfpathlineto{\pgfqpoint{0.000000in}{-0.027778in}}%
\pgfusepath{stroke,fill}%
}%
\begin{pgfscope}%
\pgfsys@transformshift{2.695256in}{0.331635in}%
\pgfsys@useobject{currentmarker}{}%
\end{pgfscope}%
\end{pgfscope}%
\begin{pgfscope}%
\pgfsetbuttcap%
\pgfsetroundjoin%
\definecolor{currentfill}{rgb}{0.000000,0.000000,0.000000}%
\pgfsetfillcolor{currentfill}%
\pgfsetlinewidth{0.602250pt}%
\definecolor{currentstroke}{rgb}{0.000000,0.000000,0.000000}%
\pgfsetstrokecolor{currentstroke}%
\pgfsetdash{}{0pt}%
\pgfsys@defobject{currentmarker}{\pgfqpoint{0.000000in}{-0.027778in}}{\pgfqpoint{0.000000in}{0.000000in}}{%
\pgfpathmoveto{\pgfqpoint{0.000000in}{0.000000in}}%
\pgfpathlineto{\pgfqpoint{0.000000in}{-0.027778in}}%
\pgfusepath{stroke,fill}%
}%
\begin{pgfscope}%
\pgfsys@transformshift{2.939604in}{0.331635in}%
\pgfsys@useobject{currentmarker}{}%
\end{pgfscope}%
\end{pgfscope}%
\begin{pgfscope}%
\pgfsetbuttcap%
\pgfsetroundjoin%
\definecolor{currentfill}{rgb}{0.000000,0.000000,0.000000}%
\pgfsetfillcolor{currentfill}%
\pgfsetlinewidth{0.602250pt}%
\definecolor{currentstroke}{rgb}{0.000000,0.000000,0.000000}%
\pgfsetstrokecolor{currentstroke}%
\pgfsetdash{}{0pt}%
\pgfsys@defobject{currentmarker}{\pgfqpoint{0.000000in}{-0.027778in}}{\pgfqpoint{0.000000in}{0.000000in}}{%
\pgfpathmoveto{\pgfqpoint{0.000000in}{0.000000in}}%
\pgfpathlineto{\pgfqpoint{0.000000in}{-0.027778in}}%
\pgfusepath{stroke,fill}%
}%
\begin{pgfscope}%
\pgfsys@transformshift{3.061778in}{0.331635in}%
\pgfsys@useobject{currentmarker}{}%
\end{pgfscope}%
\end{pgfscope}%
\begin{pgfscope}%
\pgfsetbuttcap%
\pgfsetroundjoin%
\definecolor{currentfill}{rgb}{0.000000,0.000000,0.000000}%
\pgfsetfillcolor{currentfill}%
\pgfsetlinewidth{0.602250pt}%
\definecolor{currentstroke}{rgb}{0.000000,0.000000,0.000000}%
\pgfsetstrokecolor{currentstroke}%
\pgfsetdash{}{0pt}%
\pgfsys@defobject{currentmarker}{\pgfqpoint{0.000000in}{-0.027778in}}{\pgfqpoint{0.000000in}{0.000000in}}{%
\pgfpathmoveto{\pgfqpoint{0.000000in}{0.000000in}}%
\pgfpathlineto{\pgfqpoint{0.000000in}{-0.027778in}}%
\pgfusepath{stroke,fill}%
}%
\begin{pgfscope}%
\pgfsys@transformshift{3.183951in}{0.331635in}%
\pgfsys@useobject{currentmarker}{}%
\end{pgfscope}%
\end{pgfscope}%
\begin{pgfscope}%
\pgfsetbuttcap%
\pgfsetroundjoin%
\definecolor{currentfill}{rgb}{0.000000,0.000000,0.000000}%
\pgfsetfillcolor{currentfill}%
\pgfsetlinewidth{0.602250pt}%
\definecolor{currentstroke}{rgb}{0.000000,0.000000,0.000000}%
\pgfsetstrokecolor{currentstroke}%
\pgfsetdash{}{0pt}%
\pgfsys@defobject{currentmarker}{\pgfqpoint{0.000000in}{-0.027778in}}{\pgfqpoint{0.000000in}{0.000000in}}{%
\pgfpathmoveto{\pgfqpoint{0.000000in}{0.000000in}}%
\pgfpathlineto{\pgfqpoint{0.000000in}{-0.027778in}}%
\pgfusepath{stroke,fill}%
}%
\begin{pgfscope}%
\pgfsys@transformshift{3.306125in}{0.331635in}%
\pgfsys@useobject{currentmarker}{}%
\end{pgfscope}%
\end{pgfscope}%
\begin{pgfscope}%
\pgfsetbuttcap%
\pgfsetroundjoin%
\definecolor{currentfill}{rgb}{0.000000,0.000000,0.000000}%
\pgfsetfillcolor{currentfill}%
\pgfsetlinewidth{0.602250pt}%
\definecolor{currentstroke}{rgb}{0.000000,0.000000,0.000000}%
\pgfsetstrokecolor{currentstroke}%
\pgfsetdash{}{0pt}%
\pgfsys@defobject{currentmarker}{\pgfqpoint{0.000000in}{-0.027778in}}{\pgfqpoint{0.000000in}{0.000000in}}{%
\pgfpathmoveto{\pgfqpoint{0.000000in}{0.000000in}}%
\pgfpathlineto{\pgfqpoint{0.000000in}{-0.027778in}}%
\pgfusepath{stroke,fill}%
}%
\begin{pgfscope}%
\pgfsys@transformshift{3.550473in}{0.331635in}%
\pgfsys@useobject{currentmarker}{}%
\end{pgfscope}%
\end{pgfscope}%
\begin{pgfscope}%
\pgfsetbuttcap%
\pgfsetroundjoin%
\definecolor{currentfill}{rgb}{0.000000,0.000000,0.000000}%
\pgfsetfillcolor{currentfill}%
\pgfsetlinewidth{0.602250pt}%
\definecolor{currentstroke}{rgb}{0.000000,0.000000,0.000000}%
\pgfsetstrokecolor{currentstroke}%
\pgfsetdash{}{0pt}%
\pgfsys@defobject{currentmarker}{\pgfqpoint{0.000000in}{-0.027778in}}{\pgfqpoint{0.000000in}{0.000000in}}{%
\pgfpathmoveto{\pgfqpoint{0.000000in}{0.000000in}}%
\pgfpathlineto{\pgfqpoint{0.000000in}{-0.027778in}}%
\pgfusepath{stroke,fill}%
}%
\begin{pgfscope}%
\pgfsys@transformshift{3.672647in}{0.331635in}%
\pgfsys@useobject{currentmarker}{}%
\end{pgfscope}%
\end{pgfscope}%
\begin{pgfscope}%
\pgfsetbuttcap%
\pgfsetroundjoin%
\definecolor{currentfill}{rgb}{0.000000,0.000000,0.000000}%
\pgfsetfillcolor{currentfill}%
\pgfsetlinewidth{0.602250pt}%
\definecolor{currentstroke}{rgb}{0.000000,0.000000,0.000000}%
\pgfsetstrokecolor{currentstroke}%
\pgfsetdash{}{0pt}%
\pgfsys@defobject{currentmarker}{\pgfqpoint{0.000000in}{-0.027778in}}{\pgfqpoint{0.000000in}{0.000000in}}{%
\pgfpathmoveto{\pgfqpoint{0.000000in}{0.000000in}}%
\pgfpathlineto{\pgfqpoint{0.000000in}{-0.027778in}}%
\pgfusepath{stroke,fill}%
}%
\begin{pgfscope}%
\pgfsys@transformshift{3.794821in}{0.331635in}%
\pgfsys@useobject{currentmarker}{}%
\end{pgfscope}%
\end{pgfscope}%
\begin{pgfscope}%
\pgfsetbuttcap%
\pgfsetroundjoin%
\definecolor{currentfill}{rgb}{0.000000,0.000000,0.000000}%
\pgfsetfillcolor{currentfill}%
\pgfsetlinewidth{0.602250pt}%
\definecolor{currentstroke}{rgb}{0.000000,0.000000,0.000000}%
\pgfsetstrokecolor{currentstroke}%
\pgfsetdash{}{0pt}%
\pgfsys@defobject{currentmarker}{\pgfqpoint{0.000000in}{-0.027778in}}{\pgfqpoint{0.000000in}{0.000000in}}{%
\pgfpathmoveto{\pgfqpoint{0.000000in}{0.000000in}}%
\pgfpathlineto{\pgfqpoint{0.000000in}{-0.027778in}}%
\pgfusepath{stroke,fill}%
}%
\begin{pgfscope}%
\pgfsys@transformshift{3.916995in}{0.331635in}%
\pgfsys@useobject{currentmarker}{}%
\end{pgfscope}%
\end{pgfscope}%
\begin{pgfscope}%
\pgfsetbuttcap%
\pgfsetroundjoin%
\definecolor{currentfill}{rgb}{0.000000,0.000000,0.000000}%
\pgfsetfillcolor{currentfill}%
\pgfsetlinewidth{0.602250pt}%
\definecolor{currentstroke}{rgb}{0.000000,0.000000,0.000000}%
\pgfsetstrokecolor{currentstroke}%
\pgfsetdash{}{0pt}%
\pgfsys@defobject{currentmarker}{\pgfqpoint{0.000000in}{-0.027778in}}{\pgfqpoint{0.000000in}{0.000000in}}{%
\pgfpathmoveto{\pgfqpoint{0.000000in}{0.000000in}}%
\pgfpathlineto{\pgfqpoint{0.000000in}{-0.027778in}}%
\pgfusepath{stroke,fill}%
}%
\begin{pgfscope}%
\pgfsys@transformshift{4.161342in}{0.331635in}%
\pgfsys@useobject{currentmarker}{}%
\end{pgfscope}%
\end{pgfscope}%
\begin{pgfscope}%
\pgfsetbuttcap%
\pgfsetroundjoin%
\definecolor{currentfill}{rgb}{0.000000,0.000000,0.000000}%
\pgfsetfillcolor{currentfill}%
\pgfsetlinewidth{0.602250pt}%
\definecolor{currentstroke}{rgb}{0.000000,0.000000,0.000000}%
\pgfsetstrokecolor{currentstroke}%
\pgfsetdash{}{0pt}%
\pgfsys@defobject{currentmarker}{\pgfqpoint{0.000000in}{-0.027778in}}{\pgfqpoint{0.000000in}{0.000000in}}{%
\pgfpathmoveto{\pgfqpoint{0.000000in}{0.000000in}}%
\pgfpathlineto{\pgfqpoint{0.000000in}{-0.027778in}}%
\pgfusepath{stroke,fill}%
}%
\begin{pgfscope}%
\pgfsys@transformshift{4.283516in}{0.331635in}%
\pgfsys@useobject{currentmarker}{}%
\end{pgfscope}%
\end{pgfscope}%
\begin{pgfscope}%
\pgfsetbuttcap%
\pgfsetroundjoin%
\definecolor{currentfill}{rgb}{0.000000,0.000000,0.000000}%
\pgfsetfillcolor{currentfill}%
\pgfsetlinewidth{0.602250pt}%
\definecolor{currentstroke}{rgb}{0.000000,0.000000,0.000000}%
\pgfsetstrokecolor{currentstroke}%
\pgfsetdash{}{0pt}%
\pgfsys@defobject{currentmarker}{\pgfqpoint{0.000000in}{-0.027778in}}{\pgfqpoint{0.000000in}{0.000000in}}{%
\pgfpathmoveto{\pgfqpoint{0.000000in}{0.000000in}}%
\pgfpathlineto{\pgfqpoint{0.000000in}{-0.027778in}}%
\pgfusepath{stroke,fill}%
}%
\begin{pgfscope}%
\pgfsys@transformshift{4.405690in}{0.331635in}%
\pgfsys@useobject{currentmarker}{}%
\end{pgfscope}%
\end{pgfscope}%
\begin{pgfscope}%
\pgfsetbuttcap%
\pgfsetroundjoin%
\definecolor{currentfill}{rgb}{0.000000,0.000000,0.000000}%
\pgfsetfillcolor{currentfill}%
\pgfsetlinewidth{0.602250pt}%
\definecolor{currentstroke}{rgb}{0.000000,0.000000,0.000000}%
\pgfsetstrokecolor{currentstroke}%
\pgfsetdash{}{0pt}%
\pgfsys@defobject{currentmarker}{\pgfqpoint{0.000000in}{-0.027778in}}{\pgfqpoint{0.000000in}{0.000000in}}{%
\pgfpathmoveto{\pgfqpoint{0.000000in}{0.000000in}}%
\pgfpathlineto{\pgfqpoint{0.000000in}{-0.027778in}}%
\pgfusepath{stroke,fill}%
}%
\begin{pgfscope}%
\pgfsys@transformshift{4.527864in}{0.331635in}%
\pgfsys@useobject{currentmarker}{}%
\end{pgfscope}%
\end{pgfscope}%
\begin{pgfscope}%
\pgfsetbuttcap%
\pgfsetroundjoin%
\definecolor{currentfill}{rgb}{0.000000,0.000000,0.000000}%
\pgfsetfillcolor{currentfill}%
\pgfsetlinewidth{0.803000pt}%
\definecolor{currentstroke}{rgb}{0.000000,0.000000,0.000000}%
\pgfsetstrokecolor{currentstroke}%
\pgfsetdash{}{0pt}%
\pgfsys@defobject{currentmarker}{\pgfqpoint{-0.048611in}{0.000000in}}{\pgfqpoint{-0.000000in}{0.000000in}}{%
\pgfpathmoveto{\pgfqpoint{-0.000000in}{0.000000in}}%
\pgfpathlineto{\pgfqpoint{-0.048611in}{0.000000in}}%
\pgfusepath{stroke,fill}%
}%
\begin{pgfscope}%
\pgfsys@transformshift{0.373953in}{0.977545in}%
\pgfsys@useobject{currentmarker}{}%
\end{pgfscope}%
\end{pgfscope}%
\begin{pgfscope}%
\definecolor{textcolor}{rgb}{0.000000,0.000000,0.000000}%
\pgfsetstrokecolor{textcolor}%
\pgfsetfillcolor{textcolor}%
\pgftext[x=0.100000in, y=0.924784in, left, base]{\color{textcolor}\sffamily\fontsize{10.000000}{12.000000}\selectfont 10}%
\end{pgfscope}%
\begin{pgfscope}%
\pgfsetbuttcap%
\pgfsetroundjoin%
\definecolor{currentfill}{rgb}{0.000000,0.000000,0.000000}%
\pgfsetfillcolor{currentfill}%
\pgfsetlinewidth{0.803000pt}%
\definecolor{currentstroke}{rgb}{0.000000,0.000000,0.000000}%
\pgfsetstrokecolor{currentstroke}%
\pgfsetdash{}{0pt}%
\pgfsys@defobject{currentmarker}{\pgfqpoint{-0.048611in}{0.000000in}}{\pgfqpoint{-0.000000in}{0.000000in}}{%
\pgfpathmoveto{\pgfqpoint{-0.000000in}{0.000000in}}%
\pgfpathlineto{\pgfqpoint{-0.048611in}{0.000000in}}%
\pgfusepath{stroke,fill}%
}%
\begin{pgfscope}%
\pgfsys@transformshift{0.373953in}{1.588415in}%
\pgfsys@useobject{currentmarker}{}%
\end{pgfscope}%
\end{pgfscope}%
\begin{pgfscope}%
\definecolor{textcolor}{rgb}{0.000000,0.000000,0.000000}%
\pgfsetstrokecolor{textcolor}%
\pgfsetfillcolor{textcolor}%
\pgftext[x=0.100000in, y=1.535653in, left, base]{\color{textcolor}\sffamily\fontsize{10.000000}{12.000000}\selectfont 20}%
\end{pgfscope}%
\begin{pgfscope}%
\pgfsetbuttcap%
\pgfsetroundjoin%
\definecolor{currentfill}{rgb}{0.000000,0.000000,0.000000}%
\pgfsetfillcolor{currentfill}%
\pgfsetlinewidth{0.803000pt}%
\definecolor{currentstroke}{rgb}{0.000000,0.000000,0.000000}%
\pgfsetstrokecolor{currentstroke}%
\pgfsetdash{}{0pt}%
\pgfsys@defobject{currentmarker}{\pgfqpoint{-0.048611in}{0.000000in}}{\pgfqpoint{-0.000000in}{0.000000in}}{%
\pgfpathmoveto{\pgfqpoint{-0.000000in}{0.000000in}}%
\pgfpathlineto{\pgfqpoint{-0.048611in}{0.000000in}}%
\pgfusepath{stroke,fill}%
}%
\begin{pgfscope}%
\pgfsys@transformshift{0.373953in}{2.199284in}%
\pgfsys@useobject{currentmarker}{}%
\end{pgfscope}%
\end{pgfscope}%
\begin{pgfscope}%
\definecolor{textcolor}{rgb}{0.000000,0.000000,0.000000}%
\pgfsetstrokecolor{textcolor}%
\pgfsetfillcolor{textcolor}%
\pgftext[x=0.100000in, y=2.146522in, left, base]{\color{textcolor}\sffamily\fontsize{10.000000}{12.000000}\selectfont 30}%
\end{pgfscope}%
\begin{pgfscope}%
\pgfsetbuttcap%
\pgfsetroundjoin%
\definecolor{currentfill}{rgb}{0.000000,0.000000,0.000000}%
\pgfsetfillcolor{currentfill}%
\pgfsetlinewidth{0.803000pt}%
\definecolor{currentstroke}{rgb}{0.000000,0.000000,0.000000}%
\pgfsetstrokecolor{currentstroke}%
\pgfsetdash{}{0pt}%
\pgfsys@defobject{currentmarker}{\pgfqpoint{-0.048611in}{0.000000in}}{\pgfqpoint{-0.000000in}{0.000000in}}{%
\pgfpathmoveto{\pgfqpoint{-0.000000in}{0.000000in}}%
\pgfpathlineto{\pgfqpoint{-0.048611in}{0.000000in}}%
\pgfusepath{stroke,fill}%
}%
\begin{pgfscope}%
\pgfsys@transformshift{0.373953in}{2.810153in}%
\pgfsys@useobject{currentmarker}{}%
\end{pgfscope}%
\end{pgfscope}%
\begin{pgfscope}%
\definecolor{textcolor}{rgb}{0.000000,0.000000,0.000000}%
\pgfsetstrokecolor{textcolor}%
\pgfsetfillcolor{textcolor}%
\pgftext[x=0.100000in, y=2.757392in, left, base]{\color{textcolor}\sffamily\fontsize{10.000000}{12.000000}\selectfont 40}%
\end{pgfscope}%
\begin{pgfscope}%
\pgfsetbuttcap%
\pgfsetroundjoin%
\definecolor{currentfill}{rgb}{0.000000,0.000000,0.000000}%
\pgfsetfillcolor{currentfill}%
\pgfsetlinewidth{0.803000pt}%
\definecolor{currentstroke}{rgb}{0.000000,0.000000,0.000000}%
\pgfsetstrokecolor{currentstroke}%
\pgfsetdash{}{0pt}%
\pgfsys@defobject{currentmarker}{\pgfqpoint{-0.048611in}{0.000000in}}{\pgfqpoint{-0.000000in}{0.000000in}}{%
\pgfpathmoveto{\pgfqpoint{-0.000000in}{0.000000in}}%
\pgfpathlineto{\pgfqpoint{-0.048611in}{0.000000in}}%
\pgfusepath{stroke,fill}%
}%
\begin{pgfscope}%
\pgfsys@transformshift{0.373953in}{3.421022in}%
\pgfsys@useobject{currentmarker}{}%
\end{pgfscope}%
\end{pgfscope}%
\begin{pgfscope}%
\definecolor{textcolor}{rgb}{0.000000,0.000000,0.000000}%
\pgfsetstrokecolor{textcolor}%
\pgfsetfillcolor{textcolor}%
\pgftext[x=0.100000in, y=3.368261in, left, base]{\color{textcolor}\sffamily\fontsize{10.000000}{12.000000}\selectfont 50}%
\end{pgfscope}%
\begin{pgfscope}%
\pgfsetbuttcap%
\pgfsetroundjoin%
\definecolor{currentfill}{rgb}{0.000000,0.000000,0.000000}%
\pgfsetfillcolor{currentfill}%
\pgfsetlinewidth{0.803000pt}%
\definecolor{currentstroke}{rgb}{0.000000,0.000000,0.000000}%
\pgfsetstrokecolor{currentstroke}%
\pgfsetdash{}{0pt}%
\pgfsys@defobject{currentmarker}{\pgfqpoint{-0.048611in}{0.000000in}}{\pgfqpoint{-0.000000in}{0.000000in}}{%
\pgfpathmoveto{\pgfqpoint{-0.000000in}{0.000000in}}%
\pgfpathlineto{\pgfqpoint{-0.048611in}{0.000000in}}%
\pgfusepath{stroke,fill}%
}%
\begin{pgfscope}%
\pgfsys@transformshift{0.373953in}{4.031892in}%
\pgfsys@useobject{currentmarker}{}%
\end{pgfscope}%
\end{pgfscope}%
\begin{pgfscope}%
\definecolor{textcolor}{rgb}{0.000000,0.000000,0.000000}%
\pgfsetstrokecolor{textcolor}%
\pgfsetfillcolor{textcolor}%
\pgftext[x=0.100000in, y=3.979130in, left, base]{\color{textcolor}\sffamily\fontsize{10.000000}{12.000000}\selectfont 60}%
\end{pgfscope}%
\begin{pgfscope}%
\pgfsetbuttcap%
\pgfsetroundjoin%
\definecolor{currentfill}{rgb}{0.000000,0.000000,0.000000}%
\pgfsetfillcolor{currentfill}%
\pgfsetlinewidth{0.602250pt}%
\definecolor{currentstroke}{rgb}{0.000000,0.000000,0.000000}%
\pgfsetstrokecolor{currentstroke}%
\pgfsetdash{}{0pt}%
\pgfsys@defobject{currentmarker}{\pgfqpoint{-0.027778in}{0.000000in}}{\pgfqpoint{-0.000000in}{0.000000in}}{%
\pgfpathmoveto{\pgfqpoint{-0.000000in}{0.000000in}}%
\pgfpathlineto{\pgfqpoint{-0.027778in}{0.000000in}}%
\pgfusepath{stroke,fill}%
}%
\begin{pgfscope}%
\pgfsys@transformshift{0.373953in}{0.488850in}%
\pgfsys@useobject{currentmarker}{}%
\end{pgfscope}%
\end{pgfscope}%
\begin{pgfscope}%
\pgfsetbuttcap%
\pgfsetroundjoin%
\definecolor{currentfill}{rgb}{0.000000,0.000000,0.000000}%
\pgfsetfillcolor{currentfill}%
\pgfsetlinewidth{0.602250pt}%
\definecolor{currentstroke}{rgb}{0.000000,0.000000,0.000000}%
\pgfsetstrokecolor{currentstroke}%
\pgfsetdash{}{0pt}%
\pgfsys@defobject{currentmarker}{\pgfqpoint{-0.027778in}{0.000000in}}{\pgfqpoint{-0.000000in}{0.000000in}}{%
\pgfpathmoveto{\pgfqpoint{-0.000000in}{0.000000in}}%
\pgfpathlineto{\pgfqpoint{-0.027778in}{0.000000in}}%
\pgfusepath{stroke,fill}%
}%
\begin{pgfscope}%
\pgfsys@transformshift{0.373953in}{0.611024in}%
\pgfsys@useobject{currentmarker}{}%
\end{pgfscope}%
\end{pgfscope}%
\begin{pgfscope}%
\pgfsetbuttcap%
\pgfsetroundjoin%
\definecolor{currentfill}{rgb}{0.000000,0.000000,0.000000}%
\pgfsetfillcolor{currentfill}%
\pgfsetlinewidth{0.602250pt}%
\definecolor{currentstroke}{rgb}{0.000000,0.000000,0.000000}%
\pgfsetstrokecolor{currentstroke}%
\pgfsetdash{}{0pt}%
\pgfsys@defobject{currentmarker}{\pgfqpoint{-0.027778in}{0.000000in}}{\pgfqpoint{-0.000000in}{0.000000in}}{%
\pgfpathmoveto{\pgfqpoint{-0.000000in}{0.000000in}}%
\pgfpathlineto{\pgfqpoint{-0.027778in}{0.000000in}}%
\pgfusepath{stroke,fill}%
}%
\begin{pgfscope}%
\pgfsys@transformshift{0.373953in}{0.733198in}%
\pgfsys@useobject{currentmarker}{}%
\end{pgfscope}%
\end{pgfscope}%
\begin{pgfscope}%
\pgfsetbuttcap%
\pgfsetroundjoin%
\definecolor{currentfill}{rgb}{0.000000,0.000000,0.000000}%
\pgfsetfillcolor{currentfill}%
\pgfsetlinewidth{0.602250pt}%
\definecolor{currentstroke}{rgb}{0.000000,0.000000,0.000000}%
\pgfsetstrokecolor{currentstroke}%
\pgfsetdash{}{0pt}%
\pgfsys@defobject{currentmarker}{\pgfqpoint{-0.027778in}{0.000000in}}{\pgfqpoint{-0.000000in}{0.000000in}}{%
\pgfpathmoveto{\pgfqpoint{-0.000000in}{0.000000in}}%
\pgfpathlineto{\pgfqpoint{-0.027778in}{0.000000in}}%
\pgfusepath{stroke,fill}%
}%
\begin{pgfscope}%
\pgfsys@transformshift{0.373953in}{0.855372in}%
\pgfsys@useobject{currentmarker}{}%
\end{pgfscope}%
\end{pgfscope}%
\begin{pgfscope}%
\pgfsetbuttcap%
\pgfsetroundjoin%
\definecolor{currentfill}{rgb}{0.000000,0.000000,0.000000}%
\pgfsetfillcolor{currentfill}%
\pgfsetlinewidth{0.602250pt}%
\definecolor{currentstroke}{rgb}{0.000000,0.000000,0.000000}%
\pgfsetstrokecolor{currentstroke}%
\pgfsetdash{}{0pt}%
\pgfsys@defobject{currentmarker}{\pgfqpoint{-0.027778in}{0.000000in}}{\pgfqpoint{-0.000000in}{0.000000in}}{%
\pgfpathmoveto{\pgfqpoint{-0.000000in}{0.000000in}}%
\pgfpathlineto{\pgfqpoint{-0.027778in}{0.000000in}}%
\pgfusepath{stroke,fill}%
}%
\begin{pgfscope}%
\pgfsys@transformshift{0.373953in}{1.099719in}%
\pgfsys@useobject{currentmarker}{}%
\end{pgfscope}%
\end{pgfscope}%
\begin{pgfscope}%
\pgfsetbuttcap%
\pgfsetroundjoin%
\definecolor{currentfill}{rgb}{0.000000,0.000000,0.000000}%
\pgfsetfillcolor{currentfill}%
\pgfsetlinewidth{0.602250pt}%
\definecolor{currentstroke}{rgb}{0.000000,0.000000,0.000000}%
\pgfsetstrokecolor{currentstroke}%
\pgfsetdash{}{0pt}%
\pgfsys@defobject{currentmarker}{\pgfqpoint{-0.027778in}{0.000000in}}{\pgfqpoint{-0.000000in}{0.000000in}}{%
\pgfpathmoveto{\pgfqpoint{-0.000000in}{0.000000in}}%
\pgfpathlineto{\pgfqpoint{-0.027778in}{0.000000in}}%
\pgfusepath{stroke,fill}%
}%
\begin{pgfscope}%
\pgfsys@transformshift{0.373953in}{1.221893in}%
\pgfsys@useobject{currentmarker}{}%
\end{pgfscope}%
\end{pgfscope}%
\begin{pgfscope}%
\pgfsetbuttcap%
\pgfsetroundjoin%
\definecolor{currentfill}{rgb}{0.000000,0.000000,0.000000}%
\pgfsetfillcolor{currentfill}%
\pgfsetlinewidth{0.602250pt}%
\definecolor{currentstroke}{rgb}{0.000000,0.000000,0.000000}%
\pgfsetstrokecolor{currentstroke}%
\pgfsetdash{}{0pt}%
\pgfsys@defobject{currentmarker}{\pgfqpoint{-0.027778in}{0.000000in}}{\pgfqpoint{-0.000000in}{0.000000in}}{%
\pgfpathmoveto{\pgfqpoint{-0.000000in}{0.000000in}}%
\pgfpathlineto{\pgfqpoint{-0.027778in}{0.000000in}}%
\pgfusepath{stroke,fill}%
}%
\begin{pgfscope}%
\pgfsys@transformshift{0.373953in}{1.344067in}%
\pgfsys@useobject{currentmarker}{}%
\end{pgfscope}%
\end{pgfscope}%
\begin{pgfscope}%
\pgfsetbuttcap%
\pgfsetroundjoin%
\definecolor{currentfill}{rgb}{0.000000,0.000000,0.000000}%
\pgfsetfillcolor{currentfill}%
\pgfsetlinewidth{0.602250pt}%
\definecolor{currentstroke}{rgb}{0.000000,0.000000,0.000000}%
\pgfsetstrokecolor{currentstroke}%
\pgfsetdash{}{0pt}%
\pgfsys@defobject{currentmarker}{\pgfqpoint{-0.027778in}{0.000000in}}{\pgfqpoint{-0.000000in}{0.000000in}}{%
\pgfpathmoveto{\pgfqpoint{-0.000000in}{0.000000in}}%
\pgfpathlineto{\pgfqpoint{-0.027778in}{0.000000in}}%
\pgfusepath{stroke,fill}%
}%
\begin{pgfscope}%
\pgfsys@transformshift{0.373953in}{1.466241in}%
\pgfsys@useobject{currentmarker}{}%
\end{pgfscope}%
\end{pgfscope}%
\begin{pgfscope}%
\pgfsetbuttcap%
\pgfsetroundjoin%
\definecolor{currentfill}{rgb}{0.000000,0.000000,0.000000}%
\pgfsetfillcolor{currentfill}%
\pgfsetlinewidth{0.602250pt}%
\definecolor{currentstroke}{rgb}{0.000000,0.000000,0.000000}%
\pgfsetstrokecolor{currentstroke}%
\pgfsetdash{}{0pt}%
\pgfsys@defobject{currentmarker}{\pgfqpoint{-0.027778in}{0.000000in}}{\pgfqpoint{-0.000000in}{0.000000in}}{%
\pgfpathmoveto{\pgfqpoint{-0.000000in}{0.000000in}}%
\pgfpathlineto{\pgfqpoint{-0.027778in}{0.000000in}}%
\pgfusepath{stroke,fill}%
}%
\begin{pgfscope}%
\pgfsys@transformshift{0.373953in}{1.710588in}%
\pgfsys@useobject{currentmarker}{}%
\end{pgfscope}%
\end{pgfscope}%
\begin{pgfscope}%
\pgfsetbuttcap%
\pgfsetroundjoin%
\definecolor{currentfill}{rgb}{0.000000,0.000000,0.000000}%
\pgfsetfillcolor{currentfill}%
\pgfsetlinewidth{0.602250pt}%
\definecolor{currentstroke}{rgb}{0.000000,0.000000,0.000000}%
\pgfsetstrokecolor{currentstroke}%
\pgfsetdash{}{0pt}%
\pgfsys@defobject{currentmarker}{\pgfqpoint{-0.027778in}{0.000000in}}{\pgfqpoint{-0.000000in}{0.000000in}}{%
\pgfpathmoveto{\pgfqpoint{-0.000000in}{0.000000in}}%
\pgfpathlineto{\pgfqpoint{-0.027778in}{0.000000in}}%
\pgfusepath{stroke,fill}%
}%
\begin{pgfscope}%
\pgfsys@transformshift{0.373953in}{1.832762in}%
\pgfsys@useobject{currentmarker}{}%
\end{pgfscope}%
\end{pgfscope}%
\begin{pgfscope}%
\pgfsetbuttcap%
\pgfsetroundjoin%
\definecolor{currentfill}{rgb}{0.000000,0.000000,0.000000}%
\pgfsetfillcolor{currentfill}%
\pgfsetlinewidth{0.602250pt}%
\definecolor{currentstroke}{rgb}{0.000000,0.000000,0.000000}%
\pgfsetstrokecolor{currentstroke}%
\pgfsetdash{}{0pt}%
\pgfsys@defobject{currentmarker}{\pgfqpoint{-0.027778in}{0.000000in}}{\pgfqpoint{-0.000000in}{0.000000in}}{%
\pgfpathmoveto{\pgfqpoint{-0.000000in}{0.000000in}}%
\pgfpathlineto{\pgfqpoint{-0.027778in}{0.000000in}}%
\pgfusepath{stroke,fill}%
}%
\begin{pgfscope}%
\pgfsys@transformshift{0.373953in}{1.954936in}%
\pgfsys@useobject{currentmarker}{}%
\end{pgfscope}%
\end{pgfscope}%
\begin{pgfscope}%
\pgfsetbuttcap%
\pgfsetroundjoin%
\definecolor{currentfill}{rgb}{0.000000,0.000000,0.000000}%
\pgfsetfillcolor{currentfill}%
\pgfsetlinewidth{0.602250pt}%
\definecolor{currentstroke}{rgb}{0.000000,0.000000,0.000000}%
\pgfsetstrokecolor{currentstroke}%
\pgfsetdash{}{0pt}%
\pgfsys@defobject{currentmarker}{\pgfqpoint{-0.027778in}{0.000000in}}{\pgfqpoint{-0.000000in}{0.000000in}}{%
\pgfpathmoveto{\pgfqpoint{-0.000000in}{0.000000in}}%
\pgfpathlineto{\pgfqpoint{-0.027778in}{0.000000in}}%
\pgfusepath{stroke,fill}%
}%
\begin{pgfscope}%
\pgfsys@transformshift{0.373953in}{2.077110in}%
\pgfsys@useobject{currentmarker}{}%
\end{pgfscope}%
\end{pgfscope}%
\begin{pgfscope}%
\pgfsetbuttcap%
\pgfsetroundjoin%
\definecolor{currentfill}{rgb}{0.000000,0.000000,0.000000}%
\pgfsetfillcolor{currentfill}%
\pgfsetlinewidth{0.602250pt}%
\definecolor{currentstroke}{rgb}{0.000000,0.000000,0.000000}%
\pgfsetstrokecolor{currentstroke}%
\pgfsetdash{}{0pt}%
\pgfsys@defobject{currentmarker}{\pgfqpoint{-0.027778in}{0.000000in}}{\pgfqpoint{-0.000000in}{0.000000in}}{%
\pgfpathmoveto{\pgfqpoint{-0.000000in}{0.000000in}}%
\pgfpathlineto{\pgfqpoint{-0.027778in}{0.000000in}}%
\pgfusepath{stroke,fill}%
}%
\begin{pgfscope}%
\pgfsys@transformshift{0.373953in}{2.321458in}%
\pgfsys@useobject{currentmarker}{}%
\end{pgfscope}%
\end{pgfscope}%
\begin{pgfscope}%
\pgfsetbuttcap%
\pgfsetroundjoin%
\definecolor{currentfill}{rgb}{0.000000,0.000000,0.000000}%
\pgfsetfillcolor{currentfill}%
\pgfsetlinewidth{0.602250pt}%
\definecolor{currentstroke}{rgb}{0.000000,0.000000,0.000000}%
\pgfsetstrokecolor{currentstroke}%
\pgfsetdash{}{0pt}%
\pgfsys@defobject{currentmarker}{\pgfqpoint{-0.027778in}{0.000000in}}{\pgfqpoint{-0.000000in}{0.000000in}}{%
\pgfpathmoveto{\pgfqpoint{-0.000000in}{0.000000in}}%
\pgfpathlineto{\pgfqpoint{-0.027778in}{0.000000in}}%
\pgfusepath{stroke,fill}%
}%
\begin{pgfscope}%
\pgfsys@transformshift{0.373953in}{2.443632in}%
\pgfsys@useobject{currentmarker}{}%
\end{pgfscope}%
\end{pgfscope}%
\begin{pgfscope}%
\pgfsetbuttcap%
\pgfsetroundjoin%
\definecolor{currentfill}{rgb}{0.000000,0.000000,0.000000}%
\pgfsetfillcolor{currentfill}%
\pgfsetlinewidth{0.602250pt}%
\definecolor{currentstroke}{rgb}{0.000000,0.000000,0.000000}%
\pgfsetstrokecolor{currentstroke}%
\pgfsetdash{}{0pt}%
\pgfsys@defobject{currentmarker}{\pgfqpoint{-0.027778in}{0.000000in}}{\pgfqpoint{-0.000000in}{0.000000in}}{%
\pgfpathmoveto{\pgfqpoint{-0.000000in}{0.000000in}}%
\pgfpathlineto{\pgfqpoint{-0.027778in}{0.000000in}}%
\pgfusepath{stroke,fill}%
}%
\begin{pgfscope}%
\pgfsys@transformshift{0.373953in}{2.565805in}%
\pgfsys@useobject{currentmarker}{}%
\end{pgfscope}%
\end{pgfscope}%
\begin{pgfscope}%
\pgfsetbuttcap%
\pgfsetroundjoin%
\definecolor{currentfill}{rgb}{0.000000,0.000000,0.000000}%
\pgfsetfillcolor{currentfill}%
\pgfsetlinewidth{0.602250pt}%
\definecolor{currentstroke}{rgb}{0.000000,0.000000,0.000000}%
\pgfsetstrokecolor{currentstroke}%
\pgfsetdash{}{0pt}%
\pgfsys@defobject{currentmarker}{\pgfqpoint{-0.027778in}{0.000000in}}{\pgfqpoint{-0.000000in}{0.000000in}}{%
\pgfpathmoveto{\pgfqpoint{-0.000000in}{0.000000in}}%
\pgfpathlineto{\pgfqpoint{-0.027778in}{0.000000in}}%
\pgfusepath{stroke,fill}%
}%
\begin{pgfscope}%
\pgfsys@transformshift{0.373953in}{2.687979in}%
\pgfsys@useobject{currentmarker}{}%
\end{pgfscope}%
\end{pgfscope}%
\begin{pgfscope}%
\pgfsetbuttcap%
\pgfsetroundjoin%
\definecolor{currentfill}{rgb}{0.000000,0.000000,0.000000}%
\pgfsetfillcolor{currentfill}%
\pgfsetlinewidth{0.602250pt}%
\definecolor{currentstroke}{rgb}{0.000000,0.000000,0.000000}%
\pgfsetstrokecolor{currentstroke}%
\pgfsetdash{}{0pt}%
\pgfsys@defobject{currentmarker}{\pgfqpoint{-0.027778in}{0.000000in}}{\pgfqpoint{-0.000000in}{0.000000in}}{%
\pgfpathmoveto{\pgfqpoint{-0.000000in}{0.000000in}}%
\pgfpathlineto{\pgfqpoint{-0.027778in}{0.000000in}}%
\pgfusepath{stroke,fill}%
}%
\begin{pgfscope}%
\pgfsys@transformshift{0.373953in}{2.932327in}%
\pgfsys@useobject{currentmarker}{}%
\end{pgfscope}%
\end{pgfscope}%
\begin{pgfscope}%
\pgfsetbuttcap%
\pgfsetroundjoin%
\definecolor{currentfill}{rgb}{0.000000,0.000000,0.000000}%
\pgfsetfillcolor{currentfill}%
\pgfsetlinewidth{0.602250pt}%
\definecolor{currentstroke}{rgb}{0.000000,0.000000,0.000000}%
\pgfsetstrokecolor{currentstroke}%
\pgfsetdash{}{0pt}%
\pgfsys@defobject{currentmarker}{\pgfqpoint{-0.027778in}{0.000000in}}{\pgfqpoint{-0.000000in}{0.000000in}}{%
\pgfpathmoveto{\pgfqpoint{-0.000000in}{0.000000in}}%
\pgfpathlineto{\pgfqpoint{-0.027778in}{0.000000in}}%
\pgfusepath{stroke,fill}%
}%
\begin{pgfscope}%
\pgfsys@transformshift{0.373953in}{3.054501in}%
\pgfsys@useobject{currentmarker}{}%
\end{pgfscope}%
\end{pgfscope}%
\begin{pgfscope}%
\pgfsetbuttcap%
\pgfsetroundjoin%
\definecolor{currentfill}{rgb}{0.000000,0.000000,0.000000}%
\pgfsetfillcolor{currentfill}%
\pgfsetlinewidth{0.602250pt}%
\definecolor{currentstroke}{rgb}{0.000000,0.000000,0.000000}%
\pgfsetstrokecolor{currentstroke}%
\pgfsetdash{}{0pt}%
\pgfsys@defobject{currentmarker}{\pgfqpoint{-0.027778in}{0.000000in}}{\pgfqpoint{-0.000000in}{0.000000in}}{%
\pgfpathmoveto{\pgfqpoint{-0.000000in}{0.000000in}}%
\pgfpathlineto{\pgfqpoint{-0.027778in}{0.000000in}}%
\pgfusepath{stroke,fill}%
}%
\begin{pgfscope}%
\pgfsys@transformshift{0.373953in}{3.176675in}%
\pgfsys@useobject{currentmarker}{}%
\end{pgfscope}%
\end{pgfscope}%
\begin{pgfscope}%
\pgfsetbuttcap%
\pgfsetroundjoin%
\definecolor{currentfill}{rgb}{0.000000,0.000000,0.000000}%
\pgfsetfillcolor{currentfill}%
\pgfsetlinewidth{0.602250pt}%
\definecolor{currentstroke}{rgb}{0.000000,0.000000,0.000000}%
\pgfsetstrokecolor{currentstroke}%
\pgfsetdash{}{0pt}%
\pgfsys@defobject{currentmarker}{\pgfqpoint{-0.027778in}{0.000000in}}{\pgfqpoint{-0.000000in}{0.000000in}}{%
\pgfpathmoveto{\pgfqpoint{-0.000000in}{0.000000in}}%
\pgfpathlineto{\pgfqpoint{-0.027778in}{0.000000in}}%
\pgfusepath{stroke,fill}%
}%
\begin{pgfscope}%
\pgfsys@transformshift{0.373953in}{3.298849in}%
\pgfsys@useobject{currentmarker}{}%
\end{pgfscope}%
\end{pgfscope}%
\begin{pgfscope}%
\pgfsetbuttcap%
\pgfsetroundjoin%
\definecolor{currentfill}{rgb}{0.000000,0.000000,0.000000}%
\pgfsetfillcolor{currentfill}%
\pgfsetlinewidth{0.602250pt}%
\definecolor{currentstroke}{rgb}{0.000000,0.000000,0.000000}%
\pgfsetstrokecolor{currentstroke}%
\pgfsetdash{}{0pt}%
\pgfsys@defobject{currentmarker}{\pgfqpoint{-0.027778in}{0.000000in}}{\pgfqpoint{-0.000000in}{0.000000in}}{%
\pgfpathmoveto{\pgfqpoint{-0.000000in}{0.000000in}}%
\pgfpathlineto{\pgfqpoint{-0.027778in}{0.000000in}}%
\pgfusepath{stroke,fill}%
}%
\begin{pgfscope}%
\pgfsys@transformshift{0.373953in}{3.543196in}%
\pgfsys@useobject{currentmarker}{}%
\end{pgfscope}%
\end{pgfscope}%
\begin{pgfscope}%
\pgfsetbuttcap%
\pgfsetroundjoin%
\definecolor{currentfill}{rgb}{0.000000,0.000000,0.000000}%
\pgfsetfillcolor{currentfill}%
\pgfsetlinewidth{0.602250pt}%
\definecolor{currentstroke}{rgb}{0.000000,0.000000,0.000000}%
\pgfsetstrokecolor{currentstroke}%
\pgfsetdash{}{0pt}%
\pgfsys@defobject{currentmarker}{\pgfqpoint{-0.027778in}{0.000000in}}{\pgfqpoint{-0.000000in}{0.000000in}}{%
\pgfpathmoveto{\pgfqpoint{-0.000000in}{0.000000in}}%
\pgfpathlineto{\pgfqpoint{-0.027778in}{0.000000in}}%
\pgfusepath{stroke,fill}%
}%
\begin{pgfscope}%
\pgfsys@transformshift{0.373953in}{3.665370in}%
\pgfsys@useobject{currentmarker}{}%
\end{pgfscope}%
\end{pgfscope}%
\begin{pgfscope}%
\pgfsetbuttcap%
\pgfsetroundjoin%
\definecolor{currentfill}{rgb}{0.000000,0.000000,0.000000}%
\pgfsetfillcolor{currentfill}%
\pgfsetlinewidth{0.602250pt}%
\definecolor{currentstroke}{rgb}{0.000000,0.000000,0.000000}%
\pgfsetstrokecolor{currentstroke}%
\pgfsetdash{}{0pt}%
\pgfsys@defobject{currentmarker}{\pgfqpoint{-0.027778in}{0.000000in}}{\pgfqpoint{-0.000000in}{0.000000in}}{%
\pgfpathmoveto{\pgfqpoint{-0.000000in}{0.000000in}}%
\pgfpathlineto{\pgfqpoint{-0.027778in}{0.000000in}}%
\pgfusepath{stroke,fill}%
}%
\begin{pgfscope}%
\pgfsys@transformshift{0.373953in}{3.787544in}%
\pgfsys@useobject{currentmarker}{}%
\end{pgfscope}%
\end{pgfscope}%
\begin{pgfscope}%
\pgfsetbuttcap%
\pgfsetroundjoin%
\definecolor{currentfill}{rgb}{0.000000,0.000000,0.000000}%
\pgfsetfillcolor{currentfill}%
\pgfsetlinewidth{0.602250pt}%
\definecolor{currentstroke}{rgb}{0.000000,0.000000,0.000000}%
\pgfsetstrokecolor{currentstroke}%
\pgfsetdash{}{0pt}%
\pgfsys@defobject{currentmarker}{\pgfqpoint{-0.027778in}{0.000000in}}{\pgfqpoint{-0.000000in}{0.000000in}}{%
\pgfpathmoveto{\pgfqpoint{-0.000000in}{0.000000in}}%
\pgfpathlineto{\pgfqpoint{-0.027778in}{0.000000in}}%
\pgfusepath{stroke,fill}%
}%
\begin{pgfscope}%
\pgfsys@transformshift{0.373953in}{3.909718in}%
\pgfsys@useobject{currentmarker}{}%
\end{pgfscope}%
\end{pgfscope}%
\begin{pgfscope}%
\pgfsetbuttcap%
\pgfsetroundjoin%
\definecolor{currentfill}{rgb}{0.000000,0.000000,0.000000}%
\pgfsetfillcolor{currentfill}%
\pgfsetlinewidth{0.602250pt}%
\definecolor{currentstroke}{rgb}{0.000000,0.000000,0.000000}%
\pgfsetstrokecolor{currentstroke}%
\pgfsetdash{}{0pt}%
\pgfsys@defobject{currentmarker}{\pgfqpoint{-0.027778in}{0.000000in}}{\pgfqpoint{-0.000000in}{0.000000in}}{%
\pgfpathmoveto{\pgfqpoint{-0.000000in}{0.000000in}}%
\pgfpathlineto{\pgfqpoint{-0.027778in}{0.000000in}}%
\pgfusepath{stroke,fill}%
}%
\begin{pgfscope}%
\pgfsys@transformshift{0.373953in}{4.154065in}%
\pgfsys@useobject{currentmarker}{}%
\end{pgfscope}%
\end{pgfscope}%
\begin{pgfscope}%
\pgfsetbuttcap%
\pgfsetroundjoin%
\definecolor{currentfill}{rgb}{0.000000,0.000000,0.000000}%
\pgfsetfillcolor{currentfill}%
\pgfsetlinewidth{0.602250pt}%
\definecolor{currentstroke}{rgb}{0.000000,0.000000,0.000000}%
\pgfsetstrokecolor{currentstroke}%
\pgfsetdash{}{0pt}%
\pgfsys@defobject{currentmarker}{\pgfqpoint{-0.027778in}{0.000000in}}{\pgfqpoint{-0.000000in}{0.000000in}}{%
\pgfpathmoveto{\pgfqpoint{-0.000000in}{0.000000in}}%
\pgfpathlineto{\pgfqpoint{-0.027778in}{0.000000in}}%
\pgfusepath{stroke,fill}%
}%
\begin{pgfscope}%
\pgfsys@transformshift{0.373953in}{4.276239in}%
\pgfsys@useobject{currentmarker}{}%
\end{pgfscope}%
\end{pgfscope}%
\begin{pgfscope}%
\pgfsetbuttcap%
\pgfsetroundjoin%
\definecolor{currentfill}{rgb}{0.000000,0.000000,0.000000}%
\pgfsetfillcolor{currentfill}%
\pgfsetlinewidth{0.602250pt}%
\definecolor{currentstroke}{rgb}{0.000000,0.000000,0.000000}%
\pgfsetstrokecolor{currentstroke}%
\pgfsetdash{}{0pt}%
\pgfsys@defobject{currentmarker}{\pgfqpoint{-0.027778in}{0.000000in}}{\pgfqpoint{-0.000000in}{0.000000in}}{%
\pgfpathmoveto{\pgfqpoint{-0.000000in}{0.000000in}}%
\pgfpathlineto{\pgfqpoint{-0.027778in}{0.000000in}}%
\pgfusepath{stroke,fill}%
}%
\begin{pgfscope}%
\pgfsys@transformshift{0.373953in}{4.398413in}%
\pgfsys@useobject{currentmarker}{}%
\end{pgfscope}%
\end{pgfscope}%
\begin{pgfscope}%
\pgfsetbuttcap%
\pgfsetroundjoin%
\definecolor{currentfill}{rgb}{0.000000,0.000000,0.000000}%
\pgfsetfillcolor{currentfill}%
\pgfsetlinewidth{0.602250pt}%
\definecolor{currentstroke}{rgb}{0.000000,0.000000,0.000000}%
\pgfsetstrokecolor{currentstroke}%
\pgfsetdash{}{0pt}%
\pgfsys@defobject{currentmarker}{\pgfqpoint{-0.027778in}{0.000000in}}{\pgfqpoint{-0.000000in}{0.000000in}}{%
\pgfpathmoveto{\pgfqpoint{-0.000000in}{0.000000in}}%
\pgfpathlineto{\pgfqpoint{-0.027778in}{0.000000in}}%
\pgfusepath{stroke,fill}%
}%
\begin{pgfscope}%
\pgfsys@transformshift{0.373953in}{4.520587in}%
\pgfsys@useobject{currentmarker}{}%
\end{pgfscope}%
\end{pgfscope}%
\begin{pgfscope}%
\pgfsetrectcap%
\pgfsetmiterjoin%
\pgfsetlinewidth{0.803000pt}%
\definecolor{currentstroke}{rgb}{0.000000,0.000000,0.000000}%
\pgfsetstrokecolor{currentstroke}%
\pgfsetdash{}{0pt}%
\pgfpathmoveto{\pgfqpoint{0.373953in}{0.470524in}}%
\pgfpathlineto{\pgfqpoint{0.373953in}{4.600000in}}%
\pgfusepath{stroke}%
\end{pgfscope}%
\begin{pgfscope}%
\pgfsetrectcap%
\pgfsetmiterjoin%
\pgfsetlinewidth{0.803000pt}%
\definecolor{currentstroke}{rgb}{0.000000,0.000000,0.000000}%
\pgfsetstrokecolor{currentstroke}%
\pgfsetdash{}{0pt}%
\pgfpathmoveto{\pgfqpoint{0.288431in}{0.331635in}}%
\pgfpathlineto{\pgfqpoint{4.735559in}{0.331635in}}%
\pgfusepath{stroke}%
\end{pgfscope}%
\end{pgfpicture}%
\makeatother%
\endgroup%
}
      \end{figure}
    \end{columns}
  \end{frame}

  \begin{frame}
    \frametitle{유한체}
    \framesubtitle{유한체 \(\Ftwom\)}

    나중에 시간이 나면 다룰 내용
  \end{frame}

  \begin{frame}
    \frametitle{타원곡선에서 정의된 군}
    \framesubtitle{\(\Fp\)의 경우: \(E(\Fp)\)}

    Point at infinity: \(\mathcal{O}\)
    \begin{itemize}
      \item 자기 자신 더하기: \(\mathcal{O} + \mathcal{O} = \mathcal{O}\)
      \item 다른 점 더하기: \((x,y) + \mathcal{O} = \mathcal{O} + (x,y) = (x,y)\quad\forall (x,y)\in E(\Fp)\)
    \end{itemize}
    \pause
    두 점의 덧셈
    \begin{itemize}
      \item 덧셈 역원: \((x,y) + (x,-y) = \mathcal{O}\quad\forall (x,y)\in E(\Fp)\)
      \item \((x_3,y_3) := (x_1,y_1) + (x_2,y_2)\quad(x_1\not=x_2)\)
      \pause
      \begin{itemize}
        \item \(\lambda := \frac{y_2-y_1}{x_2-x_1}\)
        \item \(x_3 := \lambda^2 - x_1 - x_2\)
        \item \(y_3 := \lambda(x_1 - x_3) - y_1\)
      \end{itemize}
    \end{itemize}
  \end{frame}

  \begin{frame}
    \frametitle{타원곡선에서 정의된 군}
    \framesubtitle{\(\Fp\)의 경우: \(E(\Fp)\)}

    두 점의 덧셈
    \begin{itemize}
      \item \((x_3,y_3) := (x_1,y_1) + (x_1,y_1)\)
      \pause
      \begin{itemize}
        \item \(\lambda := \frac{3x_1^2+a}{2y_1}\)
        \item \(x_3 := \lambda^2 - 2x_1\)
        \item \(y_3 := \lambda(x_1 - x_3) - y_1\)
      \end{itemize}
      \pause
      \item \(n(x,y)=\underbrace{(x,y) + (x,y) + \dots + (x,y)}_{n\text{개 더함}}\)
    \end{itemize}
  \end{frame}

  \begin{frame}
    \frametitle{타원곡선에서 정의된 군}
    \framesubtitle{\(\Ftwom\)의 경우}

    나중에 시간이 나면 다룰 내용
  \end{frame}

  \begin{frame}
    \frametitle{타원곡선에서 정의된 군}
    \framesubtitle{Double-and-Add Algorithm}

    Elliptic Curve Cryptography에서 \(n\)회 덧셈은 많이 쓰이는데, 여기서 \(n\)은 엄청 클 수 있음.
    \pause
    \(O(n)\)인 naive한 알고리즘은 너무 느리고, 실제 구현에서는 \(O(\log_2 n)\) 알고리즘을 사용.
    \begin{enumerate}
      \item \(n\in\Fp,\,P\in E(\Fp)\)에 대해, \(nP\)를 \(Q\)에 저장한다고 하자. 초기에 \(Q = \mathcal{O}\).
      \item \(n\)의 most significant bit부터 시작, least significant bit까지 반복.
      \pause
      \begin{enumerate}
        \item \(2Q\to Q\)
        \item 지금 보고 있는 비트를 \texttt{b}라고 할 때, \texttt{b == 1}이면 \(Q + P\to Q\)
      \end{enumerate}
    \end{enumerate}
    \pause
    이 알고리즘을 실제 코드로 구현할 때에는 side-channel attack을 막기 위한 처리가 필요
  \end{frame}

  \begin{frame}
    \frametitle{타원곡선의 응용}
    \framesubtitle{Elliptic Curve Domain Parameters}

    암호화, 복호화를 하기 위한 타원곡선, 유한체의 조건을 미리 정해놓아야 함
    \[T=(p,a,b,G,n,h)\]
    \begin{itemize}
      \item \(p\): 유한체 \(\Fp\)의 order
      \item \(a, b\): 타원곡선 방정식 \(y^2 = x^3 + ax + b\)의 파라메터
      \item \(G\): 타원곡선 위의 base point
      \item \(n\): \(G\)의 order
      \item \(h\): cofactor \(\#E(\Fp)/n\)
    \end{itemize}
  \end{frame}

  \begin{frame}
    \frametitle{타원곡선의 응용}
    \framesubtitle{ECDH}

    key pair는 무작위로 고른 정수 \(d\in [1,n-1]\)에 대해 \(Q=dG\)를 계산, \((d,Q)\).
    \begin{itemize}
      \item 여기서 \(d\)가 비밀키, \(Q\)가 공개키.
    \end{itemize}
    \pause
    Alice, Bob이 각각 key pair \((d_A,Q_A), (d_B,Q_B)\)를 갖고 있다고 하자. 두 사람이 공통의 기밀 정보 \(z\)를 공유하고 싶다고 하자.
    \begin{enumerate}
      \item Alice는 Bob에게 \(Q_A\)를, Bob은 Alice에게 \(Q_B\)를 보냄
      \item Alice는 \(z = d_A Q_B\), Bob은 \(z = d_B Q_A\)를 계산
    \end{enumerate}
    \pause
    \(z = d_A Q_B = d_B Q_A = d_A d_B G\)이므로, Alice와 Bob은 공통의 기밀 정보를 가지게 된다.
  \end{frame}

  \begin{frame}
    \frametitle{이산 로그}
    \framesubtitle{Computational Complexity}

    Alice, Bob의 통신 과정을 Eve가 도청했다고 가정
    \pause
    \begin{itemize}
      \item Eve가 얻은 \(Q_A, Q_B\)를 이용하여 \(z\)를 계산하기 위해서는 \(d_A\) 혹은 \(d_B\)를 계산해야 함
      \item 이는 이산 로그 문제로, 군의 order \(n\)에 대해 적어도 \(O(\sqrt{n})\)의 시간 복잡도를 가짐
      \pause
      \begin{itemize}
        \item \(O(\sqrt{n})\)보다 효율적인 알고리즘이 있는지는 미해결 문제
      \end{itemize}
      \pause
      \item Eve가 양자 컴퓨터를 가지고 있다면? Shor's Algorithm
    \end{itemize}
    \pause
    양자 컴퓨터의 상용화 이후를 대비하는 post-quantum cryptography로 타원 곡선 사이의 isogeny에 기반한 암호체계가 제안된 바 있음
  \end{frame}

  \begin{frame}
    \frametitle{새내기의 발표 들어주셔서 감사합니다!}
    \framesubtitle{참고 자료 + 참고하면 좋았을... 자료}

    이인석. {\it 대수학}. 서울: 서울대학교출판부, 2008. Print. 학부 대수학 강의; 2.
    \begin{itemize}
      \item 체, 군 등 대수적 구조 관련 참고
    \end{itemize}
    Certicom Research. {\it SEC 1: Elliptic Curve Cryptography}. Certicom Corp, 2009.
    \begin{itemize}
      \item 타원곡선 암호체계의 표준 문서
    \end{itemize}
    Beltrametti, Mauro. et. al. {\it Lectures on Curves, Surfaces and Projective Varieties : A Classical View of Algebraic Geometry}. Zürich: European Mathematical Society, 2009. Print. EMS Textbooks in Mathematics.
    \begin{itemize}
      \item 도서관에서 빌렸는데 첫페이지 첫단어부터 몰라서 바로 덮었습니다
    \end{itemize}
    Luca De Feo. {\it Mathematics of Isogeny Based Cryptography}. arXiv:1711.04062 [cs.CR]
    \begin{itemize}
      \item Isogeny based Cryptography까지 설명되어 있는 강의노트. 최근 읽어보는 중입니다.
    \end{itemize}
  \end{frame}
\end{document}
